% -*- latex -*-
%%%%%%%%%%%%%%%%%%%%%%%%%%%%%%%%%%%%%%%%%%%%%%%%%%%%%%%%%%%%%%%%
%%%%%%%%%%%%%%%%%%%%%%%%%%%%%%%%%%%%%%%%%%%%%%%%%%%%%%%%%%%%%%%%
%%%%
%%%% This text file is part of the source of 
%%%% `Introduction to High-Performance Scientific Computing'
%%%% by Victor Eijkhout, copyright 2012-2023
%%%%
%%%% This book is distributed under a Creative Commons Attribution 3.0
%%%% Unported (CC BY 3.0) license and made possible by funding from
%%%% The Saylor Foundation \url{http://www.saylor.org}.
%%%%
%%%%
%%%%%%%%%%%%%%%%%%%%%%%%%%%%%%%%%%%%%%%%%%%%%%%%%%%%%%%%%%%%%%%%
%%%%%%%%%%%%%%%%%%%%%%%%%%%%%%%%%%%%%%%%%%%%%%%%%%%%%%%%%%%%%%%%

%%
%% slide/book switches
%%
\newif\ifInBook\InBooktrue
\excludecomment{fullsources}
\let\slidebreak\relax

%%%%
%%%% Page layout
%%%%

\usepackage{geometry}
\addtolength{\textwidth}{.5in}
\addtolength{\textheight}{.5in}
\addtolength{\evensidemargin}{-.5in}
\renewcommand\topfraction{.95}
\renewcommand\floatpagefraction{.75}
\emergencystretch=1.5in

\usepackage{fancyhdr}
\pagestyle{fancy}\fancyhead{}\fancyfoot{}
% remove uppercase from fancy defs
\makeatletter
\def\chaptermark#1{\markboth {{\ifnum \c@secnumdepth>\m@ne
 \thechapter. \ \fi #1}}{}}
\def\sectionmark#1{\markright{{\ifnum \c@secnumdepth >\z@
 \thesection. \ \fi #1}}}
\makeatother
% now the fancy specs
%\fancyhead[LE]{\thepage \hskip.5\unitindent/\hskip.5\unitindent \leftmark}
%\fancyhead[RO]{\rightmark \hskip.5\unitindent/\hskip.5\unitindent \thepage}
\fancyhead[LE]{\leftmark}
\fancyfoot[LE]{\thepage}
\fancyhead[RO]{\rightmark}
\fancyfoot[RO]{\thepage}
\fancyfoot[RE]{\footnotesize\sl Parallel Computing -- r\revision}
\begin{lulu}
\fancyfoot[RE]{\footnotesize\sl Parallel Computing}
\end{lulu}
\fancyfoot[LO]{\footnotesize\sl Victor Eijkhout}
\setlength{\headheight}{14pt}
\addtolength{\topmargin}{-1.6pt}

%% graphics stuff
\usepackage{graphicx,outliner}
\usepackage{wrapfig}
% plotting with tikz
\usepackage{pgfplots}  
\usepgfplotslibrary{external}
\tikzexternalize
\tikzsetexternalprefix{plots/}
\pgfplotsset{width=5in,compat=1.7}  

% fancy text stuff
\usepackage{fontspec}
\setmainfont[
  Extension=.otf,
  UprightFont={*-Regular},
  BoldFont={*-Bold},
  ItalicFont={*-Italic},
  BoldItalicFont={*-BoldItalic}
]{LibertinusSerif}
\usepackage{unicode-math}
\setmathfont{LibertinusMath-Regular.otf}

%%%%
%%%% Outlining
%%%%
\OutlineLevelStart0{\chapter{#1}}
\OutlineLevelStart1{\section{#1}}
\OutlineLevelCont1{\section{#1}}
\OutlineLevelStart2{\subsection{#1}}
\OutlineLevelStart3{\subsubsection{#1}}
\setcounter{secnumdepth}{4}
\OutlineLevelStart4{\paragraph{\textit{#1}}}

%%%%
%%%% stuff
%%%%
\newcommand\furtherreading{\Level 0 {Further Reading}\label{sec:furtherreading-\chapshortname}}
\newcommand\heading[1]{\paragraph*{\textbf{#1}}}

{\catcode`\^^I=13 \globaldefs=1
 \newcommand\listing[2]{\begingroup\small\par\vspace{1ex}
  \catcode`\^^I=13 \def^^I{\leavevmode\hspace{40pt}}
  \noindent\fbox{#1}
  \verbatiminput{#2}\endgroup}
 \newcommand\codelisting[1]{\begingroup\small\par\vspace{1ex}
  \catcode`\^^I=13 \def^^I{\leavevmode\hspace{40pt}}
  \noindent\fbox{#1}
  \verbatiminput{#1}\endgroup}
}

%%
%% refer to sections in the HPC book
%%
\usepackage{xr-hyper}
\externaldocument[HPSC-]{scicompbook}
\newcommand\HPSCref[1]{HPC book, section-\ref{HPSC-#1}}
\externaldocument[CARP-]{scicomptutorials}
\newcommand\CARPref[1]{Tutorials book, section-\ref{CARP-#1}}
\externaldocument[ISP-]{ispbook}
\newcommand\ISPref[1]{Programming book, section-\ref{ISP-#1}}
%%
%% hyperref
%%
\begin{notlulu}
  \usepackage[xetex,colorlinks,pagebackref=true]{hyperref}
  \hypersetup{bookmarksopen=true}
  \renewcommand*{\backref}[1]{}
  \renewcommand*{\backrefalt}[4]{[{\tiny%
      \ifcase #1 Not cited.%
      \or Cited on page~#2.%
      \else Cited on pages #2.%
      \fi%
  }]}
  \usepackage[all]{hypcap}
\end{notlulu}
\begin{lulu}
  \usepackage{url}
\end{lulu}

\usepackage{etoolbox}

\let\exref\ref

\newtheorem{remark}{Remark}
\expandafter\ifx\csname definition\endcsname\relax
    \newtheorem{definition}{Definition}
\fi
\expandafter\ifx\csname theorem\endcsname\relax
    \newtheorem{theorem}{Theorem}
\fi
\expandafter\ifx\csname lemma\endcsname\relax
    \newtheorem{lemma}{Lemma}
\fi

\newcommand\skeleton[1]{\par
    (There is a skeleton for this exercise under the name \texttt{#1}.)}

%%%%
%%%% Verbatim source handling
%%%%

% each chapter has a list of sources
\newcommand\addchaptersource[1]{
  \ifinlistcs{#1}{\chapshortname:sourcelist}
             {}
             {\message{adding source #1}
               %\globaldefs=1
               \listcsadd{\chapshortname:sourcelist}{#1}
               %\globaldefs=0
             }
}
\newcounter{source}
\newcommand\listchaptersources[1]{
  \setcounter{source}{0}
  \renewcommand*\do[1]{\message{counting <<##1>>}\stepcounter{source}}%
  \dolistcsloop{#1:sourcelist}
  \expandafter\ifnum\value{source}>0
    \immediate\message{Sources: \arabic{source}}
    \Level 0 {Sources used in this chapter}
    \renewcommand*\do[1]{\ListOneSource{##1}}
    \dolistcsloop{#1:sourcelist}
  \fi
}
\newcommand\ListOneSource[1]{
  \immediate\message{sourcing <<#1>>}
  \Level 1 {Listing of code #1}
  \label{lst:#1}
  %% \ListStrippedSource{#1}
  \LinkRepoSource{#1}
  \par
}
\def\SourceRepoRoot{https://github.com/VictorEijkhout/TheArtOfHPC_vol2_parallelprogramming/tree/main}
\newcommand\LinkRepoSource[1]{
  \expandafter\url\expandafter{\SourceRepoRoot /#1}
}
%% \newcommand\ListStrippedSource[1]{
%%   \begingroup \footnotesize
%%   \immediate \write 18 { ./stripsource #1 }
%%   \verbatiminput{input.cut}
%%   \endgroup
%% }
\def\LSR{\LSR}
\def\ChapterSourceHeader#1\LSR{
  \def\test{#1\LSR}
  \ifx\test\LSR
  \else
    \Level 0 {Sources used in this chapter}
  \fi
}
\def\ListSourcesRecursively#1{
  \def\test{#1}
  \ifx\test\LSR
  \else
  % list the file
  \ListOneSource{#1}
  % continue
  \expandafter\ListSourcesRecursively
  \fi
}

\newenvironment{raggedlist}{\begingroup\rightskip=0pt plus 2in}{\par\endgroup}
\newenvironment{question}{\begin{quotation}\textbf{Question.\ }}{\end{quotation}}
\newenvironment
    {example}
    {\par\begin{raggedright}
      \advance\leftskip by \unitindent
      \noindent \hbox{\kern-\unitindent \textsl{Example.}}\kern1em\ignorespaces
    }
    {\par\end{raggedright}}
\newenvironment{specialnote}[1]
               {\par\begin{raggedright}
                 \advance\leftskip by \unitindent
                 \noindent \hbox{\kern-\unitindent \textsl{#1.}}\kern1em\ignorespaces
               }
               {\par\end{raggedright}}

%% list of python notes and such
\usepackage{tocloft}

\newcommand\deflanguagenote[4]{
  %% #1 : note name
  %% #2 : list id
  %% #3 : listing language
  %% #4 : header
  \argcomment
    {#1}
    {\par\advance\leftskip by \unitindent
      \refstepcounter{#1}
      {\lccode32=45 \lccode47=45
       \edef\tmp{\lowercase{\gdef\noexpand\commentarg{\CommentArg}}}\tmp}
      \def\CommentCutFile{snippets/#1-\commentarg.cut}
      \noindent \hbox
                {\kern-\unitindent \textsl{#4 note \arabic{#1}:\ \CommentArg.}}\ %
      \addcontentsline{#2}{#1}{\protect\numberline{\csname the#1\endcsname}\CommentArg}
      %\index[fortran]{\CommentArg}%
      \lstset{language=#3}%
    }
    {\par\advance\leftskip by -\unitindent
      \lstset{language=C}
      \relax}
}

\begin{comment}
\argcomment
    {mplnote}
    {\par\advance\leftskip by \unitindent
      \refstepcounter{mplnote}

      {\lccode32=45 \lccode47=45
        \edef\tmp{\lowercase{\gdef\noexpand\commentarg{\CommentArg}}}\tmp}
      \def\CommentCutFile{snippets/mplnote-\commentarg.cut}

      %\index[mpl]{\CommentArg}%
      \noindent \hbox{\kern-\unitindent \textsl{MPL note \arabic{mplnote}.}}\ %
      \addcontentsline{lompl}{mplnote}{\protect\numberline{\themplnote}\CommentArg}
      \lstset{language=C++}%
    }
    {\par \noindent\textsl{End of MPL note}\par
      \advance\leftskip by -\unitindent
      \lstset{language=C}
      \relax}
\end{comment}


\newcommand\listftnnote{List of Fortran notes}
\newlistof{fortrannote}{loftn}\listftnnote
\renewcommand\cftloftntitlefont{\slshape}
\deflanguagenote{fortrannote}{loftn}{Fortran}{Fortran}

\newcommand\listcppnote{List of C++ notes}
\newlistof{cppnote}{locpp}\listcppnote
\renewcommand\cftlocpptitlefont{\slshape}
\deflanguagenote{cppnote}{locpp}{C++}{C++}

\newcommand\listpynote{List of Python notes}
\newlistof{pythonnote}{lopy}\listpynote
\renewcommand\cftlopytitlefont{\slshape}
\deflanguagenote{pythonnote}{lopy}{Python}{Python}

%\newcounter{mplnote}
\newcommand\listmplnote{List of Mpl notes}
\newlistof{mplnote}{lompl}\listmplnote
\renewcommand\cftlompltitlefont{\slshape}
\deflanguagenote{mplnote}{lompl}{C++}{MPL}

\newenvironment
    {mplimpl}
    {\begin{quotation}\noindent\textsl{Implementation note:}\ }
    {\end{quotation}}

\newenvironment{taccnote}
  {\begin{specialnote}{TACC note}}
  {\end{specialnote}}

\newenvironment{intelnote}
    {\begin{specialnote}{Intel note}}
    {\end{specialnote}
}

\newenvironment{dpcppnote}
    {\begin{specialnote}{Intel DPC++ extension}}
    {\end{specialnote}
}

\newcommand\listmpifournote{List of MPI-4 notes}
\newlistof{mpifournote}{lomfour}\listmpifournote
\renewcommand\cftlomfourtitlefont{\slshape}
\argcomment
    {mpifournote}
    {\refstepcounter{mpifournote}
      \begingroup\def\ProcessCutFile{}\par}
    {\par \smallskip \hrule
      \hbox{\textsl{The following material is for the recently released MPI-4 standard and may not be supported yet.}}
      \nobreak
      \addcontentsline{lomfour}{mpifournote}{\protect\numberline{\thempifournote}\CommentArg}
      \input{\CommentCutFile}
      \par
      \hbox{\textsl{End of MPI-4 material}}
      \hrule
      \endgroup
    }

%% \newenvironment{highermath}
%%     {\bigskip\begin{quotation}\noindent\emph{MMM}}
%%     {\end{quotation}\bigskip\noindent\ignorespaces}


\def\chaptertitle{\csname\chaptername title\endcsname}
\def\chaptershorttitle{\csname\chaptername shorttitle\endcsname}

\newcommand\mpiparmtextsize{1.75in}
