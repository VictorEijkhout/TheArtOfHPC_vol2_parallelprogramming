% -*- latex -*-
%%%%%%%%%%%%%%%%%%%%%%%%%%%%%%%%%%%%%%%%%%%%%%%%%%%%%%%%%%%%%%%%
%%%%%%%%%%%%%%%%%%%%%%%%%%%%%%%%%%%%%%%%%%%%%%%%%%%%%%%%%%%%%%%%
%%%%
%%%% This LaTeX file is part of the source of 
%%%% `Parallel Computing'
%%%% by Victor Eijkhout, copyright 2012-2024
%%%%
%%%% idxmacs.tex : index-related macros.
%%%%
%%%%%%%%%%%%%%%%%%%%%%%%%%%%%%%%%%%%%%%%%%%%%%%%%%%%%%%%%%%%%%%%
%%%%%%%%%%%%%%%%%%%%%%%%%%%%%%%%%%%%%%%%%%%%%%%%%%%%%%%%%%%%%%%%

%% \indexsetup{noclearpage}
%% \makeindex[columns=1]
%% \usepackage{morewrites}

%% \newcommand{\indextermp}[1]{\emph{#1s}\index{#1}}
%% \newcommand{\indextermsub}[2]{\emph{#1 #2}\index{#2!#1}}
\newcommand{\indextermsubh}[2]{\emph{#1-#2}\index{#2!#1}}
%% \newcommand{\indextermsubdef}[2]{\emph{#1 #2}\index{#2!#1|textbf}}
\newcommand{\indextermsubdefh}[2]{\emph{#1-#2}\index{#2!#1|textbf}}
%% \newcommand{\indextermsubp}[2]{\emph{#1 #2s}\index{#2!#1}}
%% \newcommand{\indextermbus}[2]{\emph{#1 #2}\index{#1!#2}}
%% \newcommand{\indextermbusp}[2]{\emph{#1 #2s}\index{#1!#2}}
%% \newcommand{\indextermbusdef}[2]{\emph{#1 #2}\index{#1!#2|textbf}}
%% \newcommand{\indextermstart}[1]{\emph{#1}\index{#1|(}}
%% \newcommand{\indextermend}[1]{\index{#1|)}}
%% \newcommand{\indexstart}[1]{\index{#1|(}}
%% \newcommand{\indexend}[1]{\index{#1|)}}

%% \makeatletter
%% \newcommand\indexac[1]{\emph{\ac{#1}}%
%%   %\tracingmacros=2 \tracingcommands=2
%%   \edef\tmp{\noexpand\index{%
%%     \expandafter\expandafter\expandafter
%%         \@secondoftwo\csname fn@#1\endcsname%
%%     @\acl{#1} (#1)}}\tmp}
%% \newcommand\indexacp[1]{\emph{\ac{#1}}%
%%   %\tracingmacros=2 \tracingcommands=2
%%   \edef\tmp{\noexpand\index{%
%%     \expandafter\expandafter\expandafter
%%         \@secondoftwo\csname fn@#1\endcsname%
%%     @\acl{#1} (#1)}}\tmp}
%% \newcommand\indexacf[1]{\emph{\acf{#1}}%
%%   \edef\tmp{\noexpand\index{%
%%     \expandafter\expandafter\expandafter
%%         \@secondoftwo\csname fn@#1\endcsname
%%     @\acl{#1} (#1)}}\tmp}
%% \newcommand\indexacstart[1]{%
%%   \edef\tmp{\noexpand\index{%
%%     \expandafter\expandafter\expandafter
%%         \@secondoftwo\csname fn@#1\endcsname
%%     @\acl{#1} (#1)|(}}\tmp}
%% \newcommand\indexacend[1]{%
%%   \edef\tmp{\noexpand\index{%
%%     \expandafter\expandafter\expandafter
%%         \@secondoftwo\csname fn@#1\endcsname
%%     @\acl{#1} (#1)|)}}\tmp}
%% \makeatother


%%%%
%%%% Stuff to index
%%%%
\def\indexgenshow#1#{\bgroup
    \InnocentChars \ShowRoutinetrue
    \afterassignment#1\edef\indexedmpi
}
% mpi
\newcommand\mpitoindex{\gentoindex{mpi}{textrm}}
\newcommand\mpitoindexbf{\gentoindex{mpi}{textbf}}
\newcommand\mpitoindexit{\gentoindex{mpi}{textsl}}
% mpl
\newcommand\mpltoindex{\gentoindex{mpl}{textrm}}
\newcommand\mpltoindexbf{\gentoindex{mpl}{textbf}}
%
\newcommand\omptoindex{\gentoindex{omp}{textrm}}
\newcommand\omptoindexbf{\gentoindex{omp}{textbf}}
\newcommand\omptoindexit{\gentoindex{omp}{textsl}}
% basic routine
\newcommand\gentoindex[2]{%% 1: index name, 2: font cs
  \edef\tmp{%
    \noexpand\ifShowRoutine
        \noexpand\lstinline+\indexedmpi+\noexpand\nobreak
    \noexpand\fi
    \noexpand\index[#1]{\indexedmpi@{\catcode95=12 \noexpand\texttt{\indexedmpi}}|#2}%
  }%
  \tmp
  \egroup\nobreak
}

\def\mpitoindexf{
  \edef\tmp{%
    \noexpand\ifShowRoutine
        \noexpand\lstinline+\indexedmpi+\noexpand\nobreak
    \noexpand\fi
    \noexpand\index[mpi]{\indexedmpi@{\catcode95=12
        \noexpand\texttt{\indexedmpi}}!in Fortran}%
  }%
  \tmp
  \egroup\nobreak
}
\def\mpitoindexp{
  \edef\tmp{%
    \noexpand\ifShowRoutine
        \noexpand\lstinline+\indexedmpi+\noexpand\nobreak
    \noexpand\fi
    \noexpand\index[mpi]{\indexedmpi@{\catcode95=12
        \noexpand\texttt{\indexedmpi}}!in Python}%
  }%
  \tmp
  \egroup\nobreak
}
%% \def\mpitoindexit{%\tracingmacros=2
%%   \edef\tmp{%
%%     \noexpand\n{\indexedmpi}%
%%     \noexpand\index[mpi]{\indexedmpi@{\catcode95=12 \noexpand\texttt{\indexedmpi}}|textsl}}%
%%   \tmp
%%   \egroup\nobreak
%% }
\def\mpitoindexsub{%\tracingmacros=2
  \edef\tmp{%
    \noexpand\lstinline+\indexedmpisubone+ {\indexedmpisubtwo}%
    \noexpand\index[mpi]{\indexedmpisubone@{\catcode95=12 \noexpand\texttt{\indexedmpisubone}}!\indexedmpisubtwo@{\indexedmpisubtwo}}}%
  \tmp
  \egroup\nobreak
}


%%%%
%%%% OpenMP to index
%%%%

%%%%
%%%% MPI Routine Ref
%%%%

%%%%
%%%% PETSc Routine Ref
%%%%
{ \catcode`\_=13
  \gdef\indexpetsc#{\bgroup
    \InnocentChars
    \ShowRoutinefalse
    \tt \afterassignment\petsctoindex\edef\indexedpetsc}
  \gdef\indexpetscoption#{\bgroup
    \InnocentChars
    \ShowRoutinetrue
    \tt \afterassignment\petscoptiontoindex\edef\indexedpetsc}
  \global\let\indexpetscfile\indexpetscshow
}
\def\petsctoindex{%\tracingmacros=2
  \edef\tmp{%
    \noexpand\lstinline+\indexedpetsc+%
    \noexpand\index[petsc]{\indexedpetsc@{\catcode95=12 \noexpand\texttt{\indexedpetsc}}}}%
  \tmp
  \egroup
}
\def\petscoptiontoindex{%\tracingmacros=2
  \edef\tmp{%
    \hbox{\noexpand\lstinline+-\indexedpetsc+}%
    \noexpand\index[petsc]{-\indexedpetsc@{\catcode95=12 \noexpand\texttt{-\indexedpetsc}}}}%
  \tmp
  \egroup
}
\def\petsctoindexbf{%\tracingmacros=2
  \edef\tmp{%
    \noexpand\lstinline+\indexedpetsc+%
    \noexpand\index[petsc]{\indexedpetsc@{\catcode95=12 \noexpand\texttt{\indexedpetsc}}|textbf}%
  }%
  \tmp
  \egroup
}


%%%%
%%%% routine ref
%%%%

%%
%% Write `indexedroutine' to index #1 with #2 font
%% manpage float is done in RoutineRefDisplay
%%
\def\RoutineShowAndIndex#1#2{% write `indexedroutine' to index #1 with #2 font
  \def\ifont{#2}\def\bfont{textbf}%
  \edef\tmp{%
    \noexpand\lstinline+\indexedroutine+%
    \ifx\ifont\bfont\noexpand\label{def:\indexedroutine}\fi%
    \noexpand\index[#1]{%
      \indexedroutine@{\catcode95=12 \noexpand\texttt{\indexedroutine}}|#2}\ \relax
    (figure\noexpand~\noexpand\ref{ref:\referencedroutine})%
  }%
  \tmp%
}

\newcommand\RoutineRefStyle{
  \catcode`\_=12 % 13 \def_{\char95\discretionary{}{}{}}
  \catcode`\>=12 \catcode`\<=12
  \catcode`\&=12 \catcode`\^=12 \catcode`\~=12 \def\\{\char`\\}\relax
}
\newcommand\RoutineIndexStyle{
  \catcode`\_=12 % 13 \def_{\char95\discretionary{}{}{}}
  \catcode`\>=12 \catcode`\<=12
  \catcode`\&=12 \catcode`\^=12 \catcode`\~=12 \def\\{\char`\\}\relax
}
{ \catcode`\_=12
\gdef\underscore{_}
}

\usepackage{newfloat} % ,caption}
\DeclareFloatingEnvironment[fileext=man,placement={tp},name=Manpage]{manpage}
%\captionsetup[manpage]{justification=justified,labelformat=empty}

%\newcounter{manpage}
{ %\catcode`\_=13
  \makeatother
  \gdef\RoutineRefDisplay{%
    \begin{manpage}%
      % counter and reference to counter
      \refstepcounter{manpage}%
      \edef\labeltext{ref:\indexedroutine}
      \edef\tmp{\noexpand\label{\labeltext}}\tmp

      % routine name
      \ifx\indexedroutine\referencedroutine
      \else \par \large \ref{\labeltext}\ \textbf{\texttt{\referencedroutine}}
      \fi
      { \RoutineRefStyle %\catcode`\_=13 \def_{\underscore}
        Figure \ref{\labeltext}\ \textbf{\texttt{\indexedroutine}}
      }

      % mpi standard macros
      \def\MPI/{MPI}\def\mpi/{MPI}\def\RMA/{RMA}
      \def\mpifunc##1{\texttt{##1}}
      \let\mpiarg\mpifunc \let\mpicode\mpifunc
      \let\const\mpifunc  \let\constskip\mpifunc

      % routine display
      {\footnotesize
       \def\verbatim@startline{\verbatim@line{\leavevmode\relax}}
        \def\mpifam{mpi}
        \ifx\mpifam\routinefam
            \edef\tmp{\lowercase{\def\noexpand\standardroutine{\referencedroutine}}}\tmp
            \IfFileExists
                {standard/\standardroutine.tex}
                {\input{standard/\standardroutine}}
                {% if no standard file, then maybe handwritten file
                  \IfFileExists
                      {mpireference/\referencedroutine.tex}
                      {\verbatiminput{mpireference/\referencedroutine}}{}}
            \IfFileExists
                  {mplreference/\standardroutine.tex}
                  {MPL:\\ \verbatiminput{mplreference/\standardroutine.tex} \kern10pt \hrule}{}
            \IfFileExists
                {standardp/\standardroutine.tex}
                {Python:\\ \lstinputlisting[language=python]{standardp/\standardroutine.tex}}
                {% no python reference from cython definition
                 \IfFileExists
                     {pyreference/\referencedroutine}
                     {Python:\\ \verbatiminput{pyreference/\referencedroutine.tex}}{}
                }
        \else
            \edef\tmp{\noexpand\verbatiminput{\referencedroutine}}\tmp
        \fi
      }
      \vskip10pt
      \hrule
    \end{manpage}%
  }
}

\newcommand\pylist[1]{Class #1:\verbatiminput{standardp/class_#1.tex}}

\def\boldtt#1{\textbf{\texttt{#1}}}

%% \newcommand{\indexterm}[1]{\emph{#1}\index{#1}}
%% \def\indextermdef#1{\emph{#1}\index{#1|textbf}}
%% \def\indextermdefp#1{\emph{#1s}\index{#1|textbf}}
%% { \catcode`\_=13
%%   \gdef\tttoindex{%\tracingmacros=2
%%     \edef\tmp{%
%%       \noexpand\n{\indexedtt}%
%%       \noexpand\index{\indexedtt@{%
%%           \catcode95=12 \noexpand\texttt{\indexedtt}}}%
%%     }%
%%     \tmp
%%     \egroup
%%   }

%% % used for unix, environment variables, stuff
%%   \gdef\indextermtt#{\bgroup \InnocentChars
%%     \tt \afterassignment\tttoindex\edef\indexedtt}

%%   \gdef\ttnotetoindex{%\tracingmacros=2
%%     \edef\tmp{%
%%       \noexpand\n{\indexedttnote}%
%%       \noexpand\index{\indexedttnote@{%
%%           \catcode95=12 \noexpand\texttt{\indexedttnote}}}%
%%     }%
%%     \tmp
%%     \egroup
%%   }
%%   \gdef\indextermttnote#{\bgroup \InnocentChars
%%     \tt \afterassignment\ttnotetoindex\edef\indexedttnote}
%% }

%% \let\indextermttdef\indextermtt
%% \let\indexcommand     \indextermtt
%% \let\indextermunix    \indextermtt
%% \let\indextermfunction\indextermtt
