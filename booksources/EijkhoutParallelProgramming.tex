% -*- latex -*-
%%%%%%%%%%%%%%%%%%%%%%%%%%%%%%%%%%%%%%%%%%%%%%%%%%%%%%%%%%%%%%%%
%%%%%%%%%%%%%%%%%%%%%%%%%%%%%%%%%%%%%%%%%%%%%%%%%%%%%%%%%%%%%%%%
%%%%
%%%% This text file is part of the source of 
%%%% `Parallel Computing'
%%%% by Victor Eijkhout, copyright 2012-2025
%%%%
%%%% parcompbook.tex : master file for the book
%%%%
%%%%%%%%%%%%%%%%%%%%%%%%%%%%%%%%%%%%%%%%%%%%%%%%%%%%%%%%%%%%%%%%
%%%%%%%%%%%%%%%%%%%%%%%%%%%%%%%%%%%%%%%%%%%%%%%%%%%%%%%%%%%%%%%%

\documentclass[11pt,letterpaper,twoside,openany]{boek3}
%\documentclass{book}

\usepackage{verbatim}

\input macros/comment.sty % our own prerelease

\specialcomment{tacc}{\def\CommentCutFile{tacc.cut}}{}
\newif\ifIncludeAnswers
\IncludeAnswersfalse
\input inex
\includecomment{gpu}
\includecomment{review}
\includecomment{book}

% fancy text stuff
\usepackage{fontspec}
\setmainfont[
  Extension=.otf,
  UprightFont={*-Regular},
  BoldFont={*-Bold},
  ItalicFont={*-Italic},
  BoldItalicFont={*-BoldItalic}
]{LibertinusSerif}
\usepackage{unicode-math}
\setmathfont{LibertinusMath-Regular.otf}
%%%%%%%%%%%%%%%%%%%
\usepackage{dirtree} % ,times
% table stuff
\usepackage{booktabs,multicol,multirow}

% AMS math
%\usepackage[fleqn]{amsmath}
%\usepackage{amssymb}

\def\svnrev{428}
% dashed lines; this may interfere with other table packages
% \usepackage{arydshln}

\edef\revision{\svnrev}
\def\lulurevision{}

\newif\ifVolumeOne
\VolumeOnefalse
\newif\ifFormatAsBook \FormatAsBooktrue
\newcommand\codesnippetsdir{snippets}
\newcommand\latexengine{,xetex}
\input commonmacs
\input commonbookmacs
\input pcsecommonmacs
\input acromacs
\input blockmacs
\input exmacs
\input standardmacs
\input tutmacs
%%
%% indexing
%%
\usepackage{morewrites}
\usepackage[original,nonewpage]{imakeidx}
\makeindex % default index
\input idxmacs
\input alsoidxmacs
\input pcseidxmacs

\input bookmacs
\input pcsebookmacs
\input listingmacs
\input pcselistingmacs
\input snippetmacs
\input pcsesnippetmacs

\ifincludesources
  \def\chaptersourcelisting#1{
    \newpage
    \listchaptersources{#1}
  }
\else
  \def\chaptersourcelisting#1{
    %% \newpage
    %% \Level 0 {Full source code of examples}
    %% Please see the repository:
    %% \expandafter\url\expandafter{\SourceRepoRoot}.
    %% Examples are sorted first by programming system, then by language.
  }
\fi

%% \end{notlulu}

%%
%% Title and front matter
%%
\def\publicdraft{{\bf\normalsize \relax Public draft - open for comments}}
\def\revdate{2nd edition 2022, formatted \today\\
  \small
  Book and slides download: \url{https://tinyurl.com/vle335course}\\
  Public repository:
      \url{https://github.com/VictorEijkhout/TheArtOfHPC_vol2_parallelprogramming}\\
  HTML version: \url{https://theartofhpc.com/pcse/}\\ [20pt]
  This book is published under the CC-BY 4.0 license.
}
\begin{lulu}
  \def\publicdraft{}
  \def\revdate{2nd edition 2022\\
    Series reference: \url{https://theartofhpc.com/}
  }
\end{lulu}

\newwrite\chapterlist \openout\chapterlist=chapternames.tex

\begin{document}

\author{Victor Eijkhout}
\title{Parallel Programming in MPI and OpenMP\\
  \small The Art of HPC, volume 2}
\expandafter\date\expandafter{\revdate}
\maketitle

\input introduction
\vfill\pagebreak 

{\setcounter{tocdepth}{1}
\tableofcontents
\setcounter{tocdepth}{2}
}

\acresetall
\part{MPI}
\label{part:MPI}
\addcontentsline{locpp}{cppnote}{MPI}
\addcontentsline{loftn}{fortrannote}{MPI}
\addcontentsline{lopy}{pythonnote}{MPI}

\input chapters/mpi-competency
\CHAPTER{Getting started with MPI}{mpi-started}
\CHAPTER{MPI topic: Functional parallelism}{mpi-functional}
\CHAPTER{MPI topic: Collectives}{mpi-collective}
\CHAPTER{MPI topic: Point-to-point}{mpi-ptp}
%\end{document}
\CHAPTER{MPI topic: Communication modes}{mpi-persist}
\CHAPTER{MPI topic: Data types}{mpi-data}
\CHAPTER{MPI topic: Communicators}{mpi-comm}
\CHAPTER{MPI topic: Process management}{mpi-proc}
\CHAPTER{MPI topic: One-sided communication}{mpi-onesided}
\CHAPTER{MPI topic: File I/O}{mpi-io}
\CHAPTER{MPI topic: Topologies}{mpi-topo}
\CHAPTER{MPI topic: Shared memory}{mpi-shared}
\CHAPTER{MPI topic: Hybrid computing}{mpi-hybrid}
\CHAPTER{MPI topic: Tools interface}{mpi-tools}
\CHAPTER{MPI leftover topics}{mpi}
\CHAPTER{MPI Examples}{mpi-examples}
%\CHAPTER{MPI Review}{mpireview}

\acresetall
\part{OpenMP}
\label{part:OpenMP}
\addcontentsline{locpp}{cppnote}{OpenMP}
\addcontentsline{loftn}{fortrannote}{OpenMP}
\addcontentsline{lopy}{pythonnote}{OpenMP}

\input chapters/omp-competency
\CHAPTER{Getting started with OpenMP}{omp-basics}
\CHAPTER{OpenMP topic: Parallel regions}{omp-parallel}
\CHAPTER{OpenMP topic: Loop parallelism}{omp-loop}
\CHAPTER{OpenMP topic: Reductions}{omp-reduction}
\CHAPTER{OpenMP topic: Work sharing}{omp-share}
\CHAPTER{OpenMP topic: Controlling thread data}{omp-data}
\CHAPTER{OpenMP topic: Synchronization}{omp-sync}
\CHAPTER{OpenMP topic: Tasks}{omp-task}
\CHAPTER{OpenMP topic: Affinity}{omp-affinity}
%% \CHAPTER{OpenMP topic: Memory model}{omp-memory}
\CHAPTER{OpenMP topic: SIMD processing}{omp-simd}
\CHAPTER{OpenMP topic: Offloading}{omp-gpu}

\CHAPTER{OpenMP remaining topics}{openmp}
%%\CHAPTER{OpenMP Reference}{ompref}
%% \CHAPTER{OpenMP Review}{ompreview}
\CHAPTER{OpenMP Exercises and examples}{omp-examples}

\part{PETSc}
\label{part:petsc}
\addcontentsline{locpp}{cppnote}{PETSc}
\addcontentsline{loftn}{fortrannote}{PETSc}
\addcontentsline{lopy}{pythonnote}{PETSc}

\CHAPTER{PETSc basics}{petsc-design}
\CHAPTER{PETSc objects}{petsc-objects}
\CHAPTER{Grid support}{petsc-dmda}
\CHAPTER{Finite Elements support}{petsc-fem}
\CHAPTER{PETSc solvers}{petsc-solver}
\CHAPTER{PETSC nonlinear solvers}{petsc-nonlinear}
\CHAPTER{PETSc GPU support}{petsc-gpu}
\CHAPTER{PETSc tools}{petsc-tools}
\CHAPTER{PETSc topics}{petsc}

\part{Other programming models}
\addcontentsline{loftn}{fortrannote}{Other}
\addcontentsline{lopy}{fortrannote}{Other}

\input otherblurb

\lstset{language=Fortran}
\CHAPTER{Co-array Fortran}{caf}
\lstset{language=kokkos}
\CHAPTER{Kokkos}{kokkos}
\lstset{language=sycl}
\CHAPTER{Sycl, OneAPI, DPC++}{dpcpp}
\lstset{language=cuda}
\CHAPTER{CUDA}{cuda}
\lstset{language=kokkos}
\CHAPTER{Python multiprocessing}{multiprocessing}

\part{The Rest}

%\CHAPTER{Ruminations on parallelism}{patterns}
\CHAPTER{Exploring computer architecture}{architecture}
%% merged into next \CHAPTER{Process and thread affinity}{affinity}
\CHAPTER{Hybrid computing}{hybrid}
%% \CHAPTER{Parallel I/O}{io}
\CHAPTER{Support libraries}{libraries}

\part{Class projects}

\PROJECT{A Style Guide to Project Submissions}{projectstyle}
\PROJECT{Warmup Exercises}{warmup}
\PROJECT{Mandelbrot set}{mandelbrot}
\PROJECT{Data parallel grids}{grid}
\PROJECT{N-body problems}{nbody}

%pyskipbegin
\part{Didactics}

\CHAPTER{Teaching guide}{mpi-course}
\CHAPTER{Teaching from mental models}{mpi-mental}
%\CHAPTER{Parallel Programming Explained through Conway's Game Of Life}{conwaysection}
%pyskipend



\part {Bibliography, index, and list of acronyms}

\Level 0 {Bibliography}

\bibliography{vle}
\bibliographystyle{plain}
\vfill\pagebreak

\Level 0 {List of acronyms}

\def\acitem#1#2{\item[#1] #2}
\def\acitemi#1#2#3{\item[#1]{#2}\index{#1|see{#3}}}

\begin{multicols}{2}
\begin{description}
\input acronyms
\end{description}
\end{multicols}

\Level 0 {General Index}

\index{parallel!prefix|see{prefix}}

\printindex

\Level 0 {Lists of notes}

\Level 1 {MPI-4 notes}

\listofmpifournote
\vfill\hbox{}

\Level 1 {Fortran notes}

\listoffortrannote
\vfill\hbox{}

\Level 1 {C++ notes}

\listofcppnote
\vfill\hbox{}

\Level 1 {The MPL C++ interface}
\label{sec:idx:mpl}

\listofmplnote
\vfill\hbox{}

\Level 1 {Python notes}

\listofpythonnote

\Level 0 {Index of MPI commands and keywords}

\begin{multicols*}{2}
\printindex[mpi]
\end{multicols*}

\Level 1 {From the standard document}

This is an automatically generated list of every
function, type, and constant in the MPI standard document.
Where these appear in this book, a page reference is given.

\Level 2 {List of all functions}
\input{standard/standard-functions}
\Level 2 {List of all dtypes}
%%%% empty list?
%\input{standard/standard-dtypes}
\Level 2 {List of all ctypes}
\input{standard/standard-ctypes}
\Level 2 {List of all ftypes}
\input{standard/standard-ftypes}
\Level 2 {List of all constants}
\input{standard/standard-consts}
\Level 2 {List of all callbacks}
\input{standard/standard-callbacks}

\Level 2 {Missing routines from this book}
\indexmpidef{MPI_Address}
\indexmpidef{MPI_Alloc_mem_cptr}
\indexmpidef{MPI_Alltoallw}
\indexmpidef{MPI_Attr_delete}
\indexmpidef{MPI_Attr_put}
\indexmpidef{MPI_Buffer_iflush}
\indexmpidef{MPI_Cart_shift}
\indexmpidef{MPI_Comm_call_errhandler}
\indexmpidef{MPI_Comm_create_from_group}
\indexmpidef{MPI_Comm_get_name}
\indexmpidef{MPI_Comm_iflush_buffer}
\indexmpidef{MPI_Dist_graph_create_adjacent}
\indexmpidef{MPI_Errhandler_get}
\indexmpidef{MPI_Errhandler_set}
\indexmpidef{MPI_Error_class}
\indexmpidef{MPI_File_create_errhandler}
\indexmpidef{MPI_File_get_amode}
\indexmpidef{MPI_File_get_atomicity}
\indexmpidef{MPI_File_get_byte_offset}
\indexmpidef{MPI_File_get_errhandler}
\indexmpidef{MPI_File_get_group}
\indexmpidef{MPI_File_get_info}
\indexmpidef{MPI_File_get_position}
\indexmpidef{MPI_File_get_position_shared}
\indexmpidef{MPI_File_get_type_extent}
\indexmpidef{MPI_File_read_at_all_begin}
\indexmpidef{MPI_File_read_at_all_end}
\indexmpidef{MPI_File_read_ordered_begin}
\indexmpidef{MPI_File_read_ordered_end}
\indexmpidef{MPI_File_set_info}
\indexmpidef{MPI_File_write_at_all_begin}
\indexmpidef{MPI_File_write_at_all_end}
\indexmpidef{MPI_File_write_ordered_begin}
\indexmpidef{MPI_File_write_ordered_end}
\indexmpidef{MPI_F_sync_reg}
\indexmpidef{MPI_Grequest_complete}
\indexmpidef{MPI_Grequest_start}
\indexmpidef{MPI_Group_compare}
\indexmpidef{MPI_Group_free}
\indexmpidef{MPI_Group_from_session_pset}
\indexmpidef{MPI_Group_range_excl}
\indexmpidef{MPI_Group_range_incl}
\indexmpidef{MPI_Group_rank}
\indexmpidef{MPI_Group_size}
\indexmpidef{MPI_Group_translate_ranks}
\indexmpidef{MPI_Get_elements_c}
\indexmpidef{MPI_Ibsend}
\indexmpidef{MPI_Improbe}
\indexmpidef{MPI_Imrecv}
\indexmpidef{MPI_Ineighbor_allgather}
\indexmpidef{MPI_Ineighbor_allgatherv}
\indexmpidef{MPI_Ineighbor_alltoall}
\indexmpidef{MPI_Ineighbor_alltoallv}
\indexmpidef{MPI_Ineighbor_alltoallw}
\indexmpidef{MPI_Info_create_env}
\indexmpidef{MPI_Info_get_valuelen}
\indexmpidef{MPI_Intercomm_create_from_groups}
\indexmpidef{MPI_Irsend}
\indexmpidef{MPI_Keyval_create}
\indexmpidef{MPI_Keyval_free}
\indexmpidef{MPI_Lookup_name}
\indexmpidef{MPI_Neighbor_allgatherv}
\indexmpidef{MPI_Neighbor_alltoall}
\indexmpidef{MPI_Neighbor_alltoallv}
\indexmpidef{MPI_Neighbor_alltoallw}
\indexmpidef{MPI_Pack_external}
\indexmpidef{MPI_Pcontrol}
\indexmpidef{MPI_Raccumulate}
\indexmpidef{MPI_Register_datarep}
\indexmpidef{MPI_Remove_error_class}
\indexmpidef{MPI_Remove_error_code}
\indexmpidef{MPI_Remove_error_string}
\indexmpidef{MPI_Rget}
\indexmpidef{MPI_Rget_accumulate}
\indexmpidef{MPI_Rsend}
\indexmpidef{MPI_Session_iflush_buffer}
\indexmpidef{MPI_Status_get_error}
\indexmpidef{MPI_Status_get_source}
\indexmpidef{MPI_Status_get_tag}
\indexmpidef{MPI_Status_set_cancelled}
\indexmpidef{MPI_Status_set_elements}
\indexmpidef{MPI_Status_set_elements_x}
\indexmpidef{MPI_Status_set_error}
\indexmpidef{MPI_Status_set_source}
\indexmpidef{MPI_Status_set_tag}
\indexmpidef{MPI_Status_set_elements_c}
\indexmpidef{MPI_Test_cancelled}
\indexmpidef{MPI_Type_create_darray}
\indexmpidef{MPI_Type_create_hindexed}
\indexmpidef{MPI_Type_create_hvector}
\indexmpidef{MPI_Type_create_indexed_block}
\indexmpidef{MPI_Type_dup}
\indexmpidef{MPI_Type_extent}
\indexmpidef{MPI_Type_get_attr}
\indexmpidef{MPI_Type_get_name}
\indexmpidef{MPI_Type_get_value_index}
\indexmpidef{MPI_Type_hindexed}
\indexmpidef{MPI_Type_hvector}
\indexmpidef{MPI_Type_lb}
\indexmpidef{MPI_Type_set_name}
\indexmpidef{MPI_Type_size_x}
\indexmpidef{MPI_Type_struct}
\indexmpidef{MPI_Type_ub}
\indexmpidef{MPI_T_category_get_events}
\indexmpidef{MPI_T_category_get_num_events}
\indexmpidef{MPI_T_cvar_handle_alloc}
\indexmpidef{MPI_T_enum_get_info}
\indexmpidef{MPI_T_enum_get_item}
\indexmpidef{MPI_T_event_callback_get_info}
\indexmpidef{MPI_T_event_callback_set_info}
\indexmpidef{MPI_T_event_copy}
\indexmpidef{MPI_T_event_get_index}
\indexmpidef{MPI_T_event_get_info}
\indexmpidef{MPI_T_event_get_source}
\indexmpidef{MPI_T_event_get_timestamp}
\indexmpidef{MPI_T_event_handle_alloc}
\indexmpidef{MPI_T_event_handle_free}
\indexmpidef{MPI_T_event_handle_get_info}
\indexmpidef{MPI_T_event_handle_set_info}
\indexmpidef{MPI_T_event_read}
\indexmpidef{MPI_T_event_register_callback}
\indexmpidef{MPI_T_event_set_dropped_handler}
\indexmpidef{MPI_T_pvar_readreset}
\indexmpidef{MPI_T_pvar_reset}
\indexmpidef{MPI_T_pvar_write}
\indexmpidef{MPI_T_source_get_info}
\indexmpidef{MPI_T_source_get_num}
\indexmpidef{MPI_T_source_get_timestamp}
\indexmpidef{MPI_Type_get_extent_c}
\indexmpidef{MPI_Type_get_true_extent_c}
\indexmpidef{MPI_Type_size_c}
\indexmpidef{MPI_Unpack_external}
\indexmpidef{MPI_Waitsome}
\indexmpidef{MPI_Win_allocate_cptr}
\indexmpidef{MPI_Win_allocate_shared_cptr}
\indexmpidef{MPI_Win_create_errhandler}
\indexmpidef{MPI_Win_flush_all}
\indexmpidef{MPI_Win_flush_local_all}
\indexmpidef{MPI_Win_get_group}
\indexmpidef{MPI_Win_get_name}
\indexmpidef{MPI_Win_post}
\indexmpidef{MPI_Win_shared_query_cptr}
\indexmpidef{MPI_User_function}
\indexmpidef{MPI_ASYNC_PROTECTS_NONBLOCKING}
\indexmpidef{MPI_BOTTOM}
\indexmpidef{MPI_COMBINER_CONTIGUOUS}
\indexmpidef{MPI_COMBINER_DARRAY}
\indexmpidef{MPI_COMBINER_DUP}
\indexmpidef{MPI_COMBINER_HINDEXED}
\indexmpidef{MPI_COMBINER_HINDEXED_BLOCK}
\indexmpidef{MPI_COMBINER_HINDEXED_INTEGER}
\indexmpidef{MPI_COMBINER_HVECTOR}
\indexmpidef{MPI_COMBINER_HVECTOR_INTEGER}
\indexmpidef{MPI_COMBINER_INDEXED}
\indexmpidef{MPI_COMBINER_INDEXED_BLOCK}
\indexmpidef{MPI_COMBINER_NAMED}
\indexmpidef{MPI_COMBINER_RESIZED}
\indexmpidef{MPI_COMBINER_STRUCT}
\indexmpidef{MPI_COMBINER_STRUCT_INTEGER}
\indexmpidef{MPI_COMBINER_SUBARRAY}
\indexmpidef{MPI_COMBINER_VALUE_INDEX}
\indexmpidef{MPI_COMM_NULL}
\indexmpidef{MPI_COMM_SELF}
\indexmpidef{MPI_COMM_TYPE_HW_GUIDED}
\indexmpidef{MPI_COMM_TYPE_HW_UNGUIDED}
\indexmpidef{MPI_COMM_TYPE_RESOURCE_GUIDED}
\indexmpidef{MPI_COMM_TYPE_SHARED}
\indexmpidef{MPI_COMM_WORLD}
\indexmpidef{MPI_COUNT_KIND}
\indexmpidef{MPI_ERRHANDLER_NULL}
\indexmpidef{MPI_FILE_NULL}
\indexmpidef{MPI_F_ERROR}
\indexmpidef{MPI_F_SOURCE}
\indexmpidef{MPI_F_STATUSES_IGNORE}
\indexmpidef{MPI_F_STATUS_IGNORE}
\indexmpidef{MPI_F_STATUS_SIZE}
\indexmpidef{MPI_F_TAG}
\indexmpidef{MPI_HOST}
\indexmpidef{MPI_IDENT}
\indexmpidef{MPI_INFO_NULL}
\indexmpidef{MPI_INTEGER_KIND}
\indexmpidef{MPI_IO}
\indexmpidef{MPI_KEYVAL_INVALID}
\indexmpidef{MPI_LOCK_EXCLUSIVE}
\indexmpidef{MPI_LOCK_SHARED}
\indexmpidef{MPI_MAX_DATAREP_STRING}
\indexmpidef{MPI_MAX_INFO_VAL}
\indexmpidef{MPI_MAX_OBJECT_NAME}
\indexmpidef{MPI_MAX_PROCESSOR_NAME}
\indexmpidef{MPI_MAX_PSET_NAME_LEN}
\indexmpidef{MPI_MAX_STRINGTAG_LEN}
\indexmpidef{MPI_MESSAGE_NO_PROC}
\indexmpidef{MPI_MESSAGE_NULL}
\indexmpidef{MPI_OP_NULL}
\indexmpidef{MPI_REQUEST_NULL}
\indexmpidef{MPI_ROOT}
\indexmpidef{MPI_SESSION_NULL}
\indexmpidef{MPI_STATUSES_IGNORE}
\indexmpidef{MPI_STATUS_SIZE}
\indexmpidef{MPI_TYPECLASS_COMPLEX}
\indexmpidef{MPI_TYPECLASS_INTEGER}
\indexmpidef{MPI_TYPECLASS_REAL}
\indexmpidef{MPI_T_BIND_MPI_COMM}
\indexmpidef{MPI_T_BIND_MPI_DATATYPE}
\indexmpidef{MPI_T_BIND_MPI_ERRHANDLER}
\indexmpidef{MPI_T_BIND_MPI_FILE}
\indexmpidef{MPI_T_BIND_MPI_GROUP}
\indexmpidef{MPI_T_BIND_MPI_INFO}
\indexmpidef{MPI_T_BIND_MPI_MESSAGE}
\indexmpidef{MPI_T_BIND_MPI_OP}
\indexmpidef{MPI_T_BIND_MPI_REQUEST}
\indexmpidef{MPI_T_BIND_MPI_SESSION}
\indexmpidef{MPI_T_BIND_MPI_WIN}
\indexmpidef{MPI_T_CB_REQUIRE_ASYNC_SIGNAL_SAFE}
\indexmpidef{MPI_T_CB_REQUIRE_MPI_RESTRICTED}
\indexmpidef{MPI_T_CB_REQUIRE_NONE}
\indexmpidef{MPI_T_CB_REQUIRE_THREAD_SAFE}
\indexmpidef{MPI_T_CVAR_HANDLE_NULL}
\indexmpidef{MPI_T_ENUM_NULL}
\indexmpidef{MPI_T_SCOPE_ALL}
\indexmpidef{MPI_T_SCOPE_ALL_EQ}
\indexmpidef{MPI_T_SCOPE_CONSTANT}
\indexmpidef{MPI_T_SCOPE_GROUP}
\indexmpidef{MPI_T_SCOPE_GROUP_EQ}
\indexmpidef{MPI_T_SCOPE_LOCAL}
\indexmpidef{MPI_T_SCOPE_READONLY}
\indexmpidef{MPI_T_SOURCE_ORDERED}
\indexmpidef{MPI_T_SOURCE_UNORDERED}
\indexmpidef{MPI_T_VERBOSITY_MPIDEV_ALL}
\indexmpidef{MPI_T_VERBOSITY_MPIDEV_BASIC}
\indexmpidef{MPI_T_VERBOSITY_MPIDEV_DETAIL}
\indexmpidef{MPI_T_VERBOSITY_TUNER_ALL}
\indexmpidef{MPI_T_VERBOSITY_TUNER_BASIC}
\indexmpidef{MPI_T_VERBOSITY_TUNER_DETAIL}
\indexmpidef{MPI_T_VERBOSITY_USER_ALL}
\indexmpidef{MPI_T_VERBOSITY_USER_BASIC}
\indexmpidef{MPI_T_VERBOSITY_USER_DETAIL}
\indexmpidef{MPI_UNDEFINED}
\indexmpidef{MPI_UNWEIGHTED}
\indexmpidef{MPI_WIN_BASE}
\indexmpidef{MPI_WIN_CREATE_FLAVOR}
\indexmpidef{MPI_WIN_DISP_UNIT}
\indexmpidef{MPI_WIN_FLAVOR_ALLOCATE}
\indexmpidef{MPI_WIN_FLAVOR_CREATE}
\indexmpidef{MPI_WIN_FLAVOR_DYNAMIC}
\indexmpidef{MPI_WIN_FLAVOR_SHARED}
\indexmpidef{MPI_WIN_NULL}
\indexmpidef{MPI_WIN_SEPARATE}
\indexmpidef{MPI_WIN_SIZE}
\indexmpidef{MPI_WIN_UNIFIED}
\indexmpidef{MPI_WTIME_IS_GLOBAL}
\indexmpidef{MPI_Comm_copy_attr_function}
\indexmpidef{MPI_Comm_delete_attr_function}
\indexmpidef{MPI_Comm_errhandler_function}
\indexmpidef{MPI_Copy_function}
\indexmpidef{MPI_Datarep_conversion_function}
\indexmpidef{MPI_Datarep_conversion_function_c}
\indexmpidef{MPI_Delete_function}
\indexmpidef{MPI_File_errhandler_function}
\indexmpidef{MPI_Handler_function}
\indexmpidef{MPI_Session_errhandler_function}
\indexmpidef{MPI_Type_delete_attr_function}
\indexmpidef{MPI_User_function}
\indexmpidef{MPI_User_function_c}
\indexmpidef{MPI_Win_errhandler_function}


\Level 1 {MPI for Python}

\Level 2 {Buffer specifications}

{\small
%\begin{multicols}{2}
  \verbatiminput{standardp/bufspec.tex}
  \verbatiminput{standardp/bufspecb.tex}
  \verbatiminput{standardp/bufspecv.tex}
  \verbatiminput{standardp/bufspecw.tex}
%\end{multicols}
}

\Level 2 {Listing of python routines}

\begin{multicols}{3}
\small
\pylist{Comm}
\pylist{Cartcomm}
\pylist{Distgraphcomm}
\pylist{Graphcomm}
\pylist{Intercomm}
\pylist{Intracomm}
\pylist{Topocomm}
\pylist{Group}

\pylist{Request}
\pylist{Grequest}
\pylist{Prequest}
\pylist{Status}

\pylist{Win}

\pylist{Datatype}
\pylist{File}
\pylist{Info}
\pylist{Op}

\pylist{Errhandler}
\pylist{Message}
%\pylist{_InPlace}
%\pylist{Exception}
%\pylist{_Pickle}
\end{multicols}

\Level 0 {Index of OpenMP keywords}

\begin{multicols*}{2}
\printindex[omp]
\end{multicols*}

\Level 0 {Index of PETSc keywords}

\begin{multicols*}{2}
\printindex[petsc]
\end{multicols*}

\Level 0 {Index of KOKKOS keywords}

\begin{multicols*}{2}
\printindex[kokkos]
\end{multicols*}

\Level 0 {Index of SYCL keywords}

\begin{multicols*}{2}
\printindex[sycl]
\end{multicols*}

\hbox{}\vfill
\includegraphics{isbnbarcode}

\closeout\chapterlist
\end{document}
