% -*- latex -*-
%%%%%%%%%%%%%%%%%%%%%%%%%%%%%%%%%%%%%%%%%%%%%%%%%%%%%%%%%%%%%%%%
%%%%%%%%%%%%%%%%%%%%%%%%%%%%%%%%%%%%%%%%%%%%%%%%%%%%%%%%%%%%%%%%
%%%%
%%%% This text file is part of the source of 
%%%% `Parallel Computing'
%%%% by Victor Eijkhout, copyright 2012-2024
%%%%
%%%% parcompbook.tex : master file for the book
%%%%
%%%%%%%%%%%%%%%%%%%%%%%%%%%%%%%%%%%%%%%%%%%%%%%%%%%%%%%%%%%%%%%%
%%%%%%%%%%%%%%%%%%%%%%%%%%%%%%%%%%%%%%%%%%%%%%%%%%%%%%%%%%%%%%%%

\documentclass[11pt,letterpaper,twoside,openany]{boek3}
%\documentclass{book}

\usepackage{verbatim}

\input macros/comment.sty % our own prerelease
%\usepackage{comment}

\specialcomment{tacc}{\def\CommentCutFile{tacc.cut}}{}
\newif\ifIncludeAnswers
\IncludeAnswersfalse
\input inex
\includecomment{gpu}
\includecomment{review}
\includecomment{book}

%% arbitrary numbers of output streams
\usepackage{morewrites}

% fancy text stuff
\usepackage{fontspec}
\setmainfont[
  Extension=.otf,
  UprightFont={*-Regular},
  BoldFont={*-Bold},
  ItalicFont={*-Italic},
  BoldItalicFont={*-BoldItalic}
]{LibertinusSerif}
\usepackage{unicode-math}
\setmathfont{LibertinusMath-Regular.otf}
%%%%%%%%%%%%%%%%%%%
\usepackage{dirtree} % ,times
% table stuff
\usepackage{booktabs,multicol,multirow}

% AMS math
%\usepackage[fleqn]{amsmath}
%\usepackage{amssymb}

\def\svnrev{428}
% dashed lines; this may interfere with other table packages
% \usepackage{arydshln}

\edef\revision{\svnrev}
\def\lulurevision{}

\input commonmacs
\input acromacs
\input blockmacs
\input exmacs
\input tutmacs
\input idxmacs
\input idxpkgmacs
%% \makeindex
\input bookmacs
\input snippetmacs

\ifincludesources
  \def\chaptersourcelisting#1{
    \newpage
    \listchaptersources{#1}
  }
\else
  \def\chaptersourcelisting#1{
    %% \newpage
    %% \Level 0 {Full source code of examples}
    %% Please see the repository:
    %% \expandafter\url\expandafter{\SourceRepoRoot}.
    %% Examples are sorted first by programming system, then by language.
  }
\fi

%% \end{notlulu}

%%
%% Title and front matter
%%
\def\publicdraft{{\bf\normalsize \relax Public draft - open for comments}}
\def\revdate{2nd edition 2022, formatted \today\\
  \small
  Book and slides download: \url{https://tinyurl.com/vle335course}\\
  Public repository:
      \url{https://github.com/VictorEijkhout/TheArtOfHPC_vol2_parallelprogramming}\\
  HTML version: \url{https://theartofhpc.com/pcse/}\\ [20pt]
  This book is published under the CC-BY 4.0 license.
}
\begin{lulu}
  \def\publicdraft{}
  \def\revdate{2nd edition 2022\\
    Series reference: \url{https://theartofhpc.com/}
  }
\end{lulu}

\newwrite\chapterlist \openout\chapterlist=chapternames.tex

\begin{document}

\author{Victor Eijkhout}
\title{Parallel Programming in MPI and OpenMP\\
  \small The Art of HPC, volume 2}
\expandafter\date\expandafter{\revdate}
\maketitle

\input introduction
\vfill\pagebreak 

{\setcounter{tocdepth}{1}
\tableofcontents
\setcounter{tocdepth}{2}
}

\acresetall
\part{MPI}
\label{part:MPI}
\addcontentsline{locpp}{cppnote}{MPI}
\addcontentsline{loftn}{fortrannote}{MPI}
\addcontentsline{lopy}{pythonnote}{MPI}

\input chapters/mpi-competency
\CHAPTER{Getting started with MPI}{mpi-started}
\CHAPTER{MPI topic: Functional parallelism}{mpi-functional}
\CHAPTER{MPI topic: Collectives}{mpi-collective}
\CHAPTER{MPI topic: Point-to-point}{mpi-ptp}
%\end{document}
\CHAPTER{MPI topic: Communication modes}{mpi-persist}
\CHAPTER{MPI topic: Data types}{mpi-data}
\CHAPTER{MPI topic: Communicators}{mpi-comm}
\CHAPTER{MPI topic: Process management}{mpi-proc}
\CHAPTER{MPI topic: One-sided communication}{mpi-onesided}
\CHAPTER{MPI topic: File I/O}{mpi-io}
\CHAPTER{MPI topic: Topologies}{mpi-topo}
\CHAPTER{MPI topic: Shared memory}{mpi-shared}
\CHAPTER{MPI topic: Hybrid computing}{mpi-hybrid}
\CHAPTER{MPI topic: Tools interface}{mpi-tools}
\CHAPTER{MPI leftover topics}{mpi}
\CHAPTER{MPI Examples}{mpi-examples}
%\CHAPTER{MPI Review}{mpireview}

\acresetall
\part{OpenMP}
\label{part:OpenMP}
\addcontentsline{locpp}{cppnote}{OpenMP}
\addcontentsline{loftn}{fortrannote}{OpenMP}
\addcontentsline{lopy}{pythonnote}{OpenMP}

\input chapters/omp-competency
\CHAPTER{Getting started with OpenMP}{omp-basics}
\CHAPTER{OpenMP topic: Parallel regions}{omp-parallel}
\CHAPTER{OpenMP topic: Loop parallelism}{omp-loop}
\CHAPTER{OpenMP topic: Reductions}{omp-reduction}
\CHAPTER{OpenMP topic: Work sharing}{omp-share}
\CHAPTER{OpenMP topic: Controlling thread data}{omp-data}
\CHAPTER{OpenMP topic: Synchronization}{omp-sync}
\CHAPTER{OpenMP topic: Tasks}{omp-task}
\CHAPTER{OpenMP topic: Affinity}{omp-affinity}
%% \CHAPTER{OpenMP topic: Memory model}{omp-memory}
\CHAPTER{OpenMP topic: SIMD processing}{omp-simd}
\CHAPTER{OpenMP topic: Offloading}{omp-gpu}

\CHAPTER{OpenMP remaining topics}{openmp}
%%\CHAPTER{OpenMP Reference}{ompref}
\CHAPTER{OpenMP Review}{ompreview}
\CHAPTER{OpenMP Examples}{omp-examples}

\part{PETSc}
\label{part:petsc}
\addcontentsline{locpp}{cppnote}{PETSc}
\addcontentsline{loftn}{fortrannote}{PETSc}
\addcontentsline{lopy}{pythonnote}{PETSc}

\CHAPTER{PETSc basics}{petsc-design}
\CHAPTER{PETSc objects}{petsc-objects}
\CHAPTER{Grid support}{petsc-dmda}
\CHAPTER{Finite Elements support}{petsc-fem}
\CHAPTER{PETSc solvers}{petsc-solver}
\CHAPTER{PETSC nonlinear solvers}{petsc-nonlinear}
\CHAPTER{PETSc GPU support}{petsc-gpu}
\CHAPTER{PETSc tools}{petsc-tools}
\CHAPTER{PETSc topics}{petsc}

\part{Other programming models}
\addcontentsline{loftn}{fortrannote}{Other}
\addcontentsline{lopy}{fortrannote}{Other}

\input otherblurb

\lstset{language=Fortran}
\CHAPTER{Co-array Fortran}{caf}
\lstset{language=C++}
\CHAPTER{Kokkos}{kokkos}
\CHAPTER{Sycl, OneAPI, DPC++}{dpcpp}
\CHAPTER{Python multiprocessing}{multiprocessing}

\part{The Rest}

%\CHAPTER{Ruminations on parallelism}{patterns}
\CHAPTER{Exploring computer architecture}{architecture}
%% merged into next \CHAPTER{Process and thread affinity}{affinity}
\CHAPTER{Hybrid computing}{hybrid}
%% \CHAPTER{Parallel I/O}{io}
\CHAPTER{Support libraries}{libraries}

\part{Class projects}

\PROJECT{A Style Guide to Project Submissions}{projectstyle}
\PROJECT{Warmup Exercises}{warmup}
\PROJECT{Mandelbrot set}{mandelbrot}
\PROJECT{Data parallel grids}{grid}
\PROJECT{N-body problems}{nbody}

%pyskipbegin
\part{Didactics}

\CHAPTER{Teaching guide}{mpi-course}
\CHAPTER{Teaching from mental models}{mpi-mental}
%\CHAPTER{Parallel Programming Explained through Conway's Game Of Life}{conwaysection}
%pyskipend



\part {Bibliography, index, and list of acronyms}

\Level 0 {Bibliography}

\bibliography{vle}
\bibliographystyle{plain}
\vfill\pagebreak

\Level 0 {List of acronyms}

\def\acitem#1#2{\item[#1] #2}
\def\acitemi#1#2#3{\item[#1]{#2}\index{#1|see{#3}}}

\begin{multicols}{2}
\begin{description}
\input acronyms
\end{description}
\end{multicols}

\Level 0 {General Index}

\index{parallel!prefix|see{prefix}}

\printindex

\Level 0 {Lists of notes}

\Level 1 {MPI-4 notes}

\listofmpifournote
\vfill\hbox{}

\Level 1 {Fortran notes}

\listoffortrannote
\vfill\hbox{}

\Level 1 {C++ notes}

\listofcppnote
\vfill\hbox{}

\Level 1 {The MPL C++ interface}
\label{sec:idx:mpl}

\listofmplnote
\vfill\hbox{}

\Level 1 {Python notes}

\listofpythonnote

\Level 0 {Index of MPI commands and keywords}

\begin{multicols*}{2}
\printindex[mpi]
\end{multicols*}

\Level 1 {From the standard document}

This is an automatically generated list of every
function, type, and constant in the MPI standard document.
Where these appear in this book, a page reference is given.

\Level 2 {List of all functions}
\begin{multicols}{3}
\catcode`\_=12
\footnotesize
\begin{itemize}
\end{itemize}
\end{multicols}

\Level 2 {List of all dtypes}
%%%% empty list?
%\input{standard/standard-dtypes}
\Level 2 {List of all ctypes}
\begin{multicols}{3}
\catcode`\_=12
\footnotesize
\begin{itemize}
\item \texttt{Datatype}~\pageref{def:Datatype}
\item \texttt{Group}~\pageref{def:Group}
\item \texttt{MPI_Aint}~\pageref{def:MPI_Aint}
\item \texttt{MPI_Comm_copy_attr_function}~\pageref{def:MPI_Comm_copy_attr_function}
\item \texttt{MPI_Comm_delete_attr_function}~\pageref{def:MPI_Comm_delete_attr_function}
\item \texttt{MPI_Copy_function}~\pageref{def:MPI_Copy_function}
\item \texttt{MPI_Count}~\pageref{def:MPI_Count}
\item \texttt{MPI_Datarep_conversion_function}~\pageref{def:MPI_Datarep_conversion_function}
\item \texttt{MPI_Datarep_conversion_function_c}~\pageref{def:MPI_Datarep_conversion_function_c}
\item \texttt{MPI_Delete_function}~\pageref{def:MPI_Delete_function}
\item \texttt{MPI_Offset}~\pageref{def:MPI_Offset}
\item \texttt{MPI_Type_copy_attr_function}~\pageref{def:MPI_Type_copy_attr_function}
\item \texttt{MPI_Type_delete_attr_function}~\pageref{def:MPI_Type_delete_attr_function}
\item \texttt{MPI_Win_copy_attr_function}~\pageref{def:MPI_Win_copy_attr_function}
\item \texttt{MPI_Win_delete_attr_function}~\pageref{def:MPI_Win_delete_attr_function}
\item \texttt{Status}~\pageref{def:Status}
\item \texttt{_Bool}~\pageref{def:_Bool}
\item \texttt{bool}~\pageref{def:bool}
\item \texttt{char}~\pageref{def:char}
\item \texttt{class}~\pageref{def:class}
\item \texttt{double}~\pageref{def:double}
\item \texttt{enum}~\pageref{def:enum}
\item \texttt{float}~\pageref{def:float}
\item \texttt{int}~\pageref{def:int}
\item \texttt{long}~\pageref{def:long}
\item \texttt{short}~\pageref{def:short}
\item \texttt{wchar_t}~\pageref{def:wchar_t}
\end{itemize}
\end{multicols}

\Level 2 {List of all ftypes}
\begin{multicols}{3}
\catcode`\_=12
\footnotesize
\begin{itemize}
\item \texttt{ALLOCATABLE}~\pageref{def:ALLOCATABLE}
\item \texttt{ASYNCHRONOUS}~\pageref{def:ASYNCHRONOUS}
\item \texttt{BLOCK}~\pageref{def:BLOCK}
\item \texttt{CHARACTER}~\pageref{def:CHARACTER}
\item \texttt{COMMON}~\pageref{def:COMMON}
\item \texttt{COMM_COPY_ATTR_FUNCTION}~\pageref{def:COMM_COPY_ATTR_FUNCTION}
\item \texttt{COMM_DELETE_ATTR_FUNCTION}~\pageref{def:COMM_DELETE_ATTR_FUNCTION}
\item \texttt{COMPLEX}~\pageref{def:COMPLEX}
\item \texttt{CONTAINS}~\pageref{def:CONTAINS}
\item \texttt{CONTIGUOUS}~\pageref{def:CONTIGUOUS}
\item \texttt{COPY_FUNCTION}~\pageref{def:COPY_FUNCTION}
\item \texttt{C_F_POINTER}~\pageref{def:C_F_POINTER}
\item \texttt{C_PTR}~\pageref{def:C_PTR}
\item \texttt{DATAREP_CONVERSION_FUNCTION}~\pageref{def:DATAREP_CONVERSION_FUNCTION}
\item \texttt{DELETE_FUNCTION}~\pageref{def:DELETE_FUNCTION}
\item \texttt{EXTERNAL}~\pageref{def:EXTERNAL}
\item \texttt{FUNCTION}~\pageref{def:FUNCTION}
\item \texttt{IN}~\pageref{def:IN}
\item \texttt{INCLUDE}~\pageref{def:INCLUDE}
\item \texttt{INOUT}~\pageref{def:INOUT}
\item \texttt{INTEGER}~\pageref{def:INTEGER}
\item \texttt{INTENT}~\pageref{def:INTENT}
\item \texttt{INTERFACE}~\pageref{def:INTERFACE}
\item \texttt{ISO_C_BINDING}~\pageref{def:ISO_C_BINDING}
\item \texttt{ISO_FORTRAN_ENV}~\pageref{def:ISO_FORTRAN_ENV}
\item \texttt{KIND}~\pageref{def:KIND}
\item \texttt{LOGICAL}~\pageref{def:LOGICAL}
\item \texttt{MODULE}~\pageref{def:MODULE}
\item \texttt{MPI_Send}~\pageref{def:MPI_Send}
\item \texttt{MPI_Status}~\pageref{def:MPI_Status}
\item \texttt{MPI_User_function}~\pageref{def:MPI_User_function}
\item \texttt{MPI_Waitall}~\pageref{def:MPI_Waitall}
\item \texttt{OPTIONAL}~\pageref{def:OPTIONAL}
\item \texttt{OUT}~\pageref{def:OUT}
\item \texttt{POINTER}~\pageref{def:POINTER}
\item \texttt{PROCEDURE}~\pageref{def:PROCEDURE}
\item \texttt{REAL}~\pageref{def:REAL}
\item \texttt{SEQUENCE}~\pageref{def:SEQUENCE}
\item \texttt{TARGET}~\pageref{def:TARGET}
\item \texttt{TYPE}~\pageref{def:TYPE}
\item \texttt{TYPE_COPY_ATTR_FUNCTION}~\pageref{def:TYPE_COPY_ATTR_FUNCTION}
\item \texttt{TYPE_DELETE_ATTR_FUNCTION}~\pageref{def:TYPE_DELETE_ATTR_FUNCTION}
\item \texttt{USER_FUNCTION}~\pageref{def:USER_FUNCTION}
\item \texttt{VOLATILE}~\pageref{def:VOLATILE}
\item \texttt{WIN_COPY_ATTR_FUNCTION}~\pageref{def:WIN_COPY_ATTR_FUNCTION}
\item \texttt{WIN_DELETE_ATTR_FUNCTION}~\pageref{def:WIN_DELETE_ATTR_FUNCTION}
\item \texttt{base}~\pageref{def:base}
\item \texttt{foo}~\pageref{def:foo}
\item \texttt{int}~\pageref{def:int}
\item \texttt{separated_sections}~\pageref{def:separated_sections}
\end{itemize}
\end{multicols}

\Level 2 {List of all constants}
\begin{multicols}{3}
\catcode`\_=12
\footnotesize
\begin{itemize}
\item \texttt{MPI_ADDRESS_KIND}~\pageref{def:MPI_ADDRESS_KIND}
\item \texttt{MPI_ANY_SOURCE}~\pageref{def:MPI_ANY_SOURCE}
\item \texttt{MPI_ANY_TAG}~\pageref{def:MPI_ANY_TAG}
\item \texttt{MPI_APPNUM}~\pageref{def:MPI_APPNUM}
\item \texttt{MPI_ARGVS_NULL}~\pageref{def:MPI_ARGVS_NULL}
\item \texttt{MPI_ARGV_NULL}~\pageref{def:MPI_ARGV_NULL}
\item \texttt{MPI_ASYNC_PROTECTS_NONBLOCKING}~\pageref{def:MPI_ASYNC_PROTECTS_NONBLOCKING}
\item \texttt{MPI_Aint}~\pageref{def:MPI_Aint}
\item \texttt{MPI_BAND}~\pageref{def:MPI_BAND}
\item \texttt{MPI_BOR}~\pageref{def:MPI_BOR}
\item \texttt{MPI_BOTTOM}~\pageref{def:MPI_BOTTOM}
\item \texttt{MPI_BSEND_OVERHEAD}~\pageref{def:MPI_BSEND_OVERHEAD}
\item \texttt{MPI_BXOR}~\pageref{def:MPI_BXOR}
\item \texttt{MPI_CART}~\pageref{def:MPI_CART}
\item \texttt{MPI_COMBINER_CONTIGUOUS}~\pageref{def:MPI_COMBINER_CONTIGUOUS}
\item \texttt{MPI_COMBINER_DARRAY}~\pageref{def:MPI_COMBINER_DARRAY}
\item \texttt{MPI_COMBINER_DUP}~\pageref{def:MPI_COMBINER_DUP}
\item \texttt{MPI_COMBINER_HINDEXED}~\pageref{def:MPI_COMBINER_HINDEXED}
\item \texttt{MPI_COMBINER_HINDEXED_BLOCK}~\pageref{def:MPI_COMBINER_HINDEXED_BLOCK}
\item \texttt{MPI_COMBINER_HINDEXED_INTEGER}~\pageref{def:MPI_COMBINER_HINDEXED_INTEGER}
\item \texttt{MPI_COMBINER_HVECTOR}~\pageref{def:MPI_COMBINER_HVECTOR}
\item \texttt{MPI_COMBINER_HVECTOR_INTEGER}~\pageref{def:MPI_COMBINER_HVECTOR_INTEGER}
\item \texttt{MPI_COMBINER_INDEXED}~\pageref{def:MPI_COMBINER_INDEXED}
\item \texttt{MPI_COMBINER_INDEXED_BLOCK}~\pageref{def:MPI_COMBINER_INDEXED_BLOCK}
\item \texttt{MPI_COMBINER_NAMED}~\pageref{def:MPI_COMBINER_NAMED}
\item \texttt{MPI_COMBINER_RESIZED}~\pageref{def:MPI_COMBINER_RESIZED}
\item \texttt{MPI_COMBINER_STRUCT}~\pageref{def:MPI_COMBINER_STRUCT}
\item \texttt{MPI_COMBINER_STRUCT_INTEGER}~\pageref{def:MPI_COMBINER_STRUCT_INTEGER}
\item \texttt{MPI_COMBINER_SUBARRAY}~\pageref{def:MPI_COMBINER_SUBARRAY}
\item \texttt{MPI_COMBINER_VECTOR}~\pageref{def:MPI_COMBINER_VECTOR}
\item \texttt{MPI_COMM_NULL}~\pageref{def:MPI_COMM_NULL}
\item \texttt{MPI_COMM_SELF}~\pageref{def:MPI_COMM_SELF}
\item \texttt{MPI_COMM_TYPE_HW_GUIDED}~\pageref{def:MPI_COMM_TYPE_HW_GUIDED}
\item \texttt{MPI_COMM_TYPE_HW_UNGUIDED}~\pageref{def:MPI_COMM_TYPE_HW_UNGUIDED}
\item \texttt{MPI_COMM_TYPE_SHARED}~\pageref{def:MPI_COMM_TYPE_SHARED}
\item \texttt{MPI_COMM_WORLD}~\pageref{def:MPI_COMM_WORLD}
\item \texttt{MPI_CONGRUENT}~\pageref{def:MPI_CONGRUENT}
\item \texttt{MPI_COUNT_KIND}~\pageref{def:MPI_COUNT_KIND}
\item \texttt{MPI_Comm}~\pageref{def:MPI_Comm}
\item \texttt{MPI_Count}~\pageref{def:MPI_Count}
\item \texttt{MPI_DATATYPE_NULL}~\pageref{def:MPI_DATATYPE_NULL}
\item \texttt{MPI_DISPLACEMENT_CURRENT}~\pageref{def:MPI_DISPLACEMENT_CURRENT}
\item \texttt{MPI_DISTRIBUTE_BLOCK}~\pageref{def:MPI_DISTRIBUTE_BLOCK}
\item \texttt{MPI_DISTRIBUTE_CYCLIC}~\pageref{def:MPI_DISTRIBUTE_CYCLIC}
\item \texttt{MPI_DISTRIBUTE_DFLT_DARG}~\pageref{def:MPI_DISTRIBUTE_DFLT_DARG}
\item \texttt{MPI_DISTRIBUTE_NONE}~\pageref{def:MPI_DISTRIBUTE_NONE}
\item \texttt{MPI_DIST_GRAPH}~\pageref{def:MPI_DIST_GRAPH}
\item \texttt{MPI_Datatype}~\pageref{def:MPI_Datatype}
\item \texttt{MPI_ERRCODES_IGNORE}~\pageref{def:MPI_ERRCODES_IGNORE}
\item \texttt{MPI_ERRHANDLER_NULL}~\pageref{def:MPI_ERRHANDLER_NULL}
\item \texttt{MPI_ERROR}~\pageref{def:MPI_ERROR}
\item \texttt{MPI_ERRORS_ABORT}~\pageref{def:MPI_ERRORS_ABORT}
\item \texttt{MPI_ERRORS_ARE_FATAL}~\pageref{def:MPI_ERRORS_ARE_FATAL}
\item \texttt{MPI_ERRORS_RETURN}~\pageref{def:MPI_ERRORS_RETURN}
\item \texttt{MPI_ERR_ARG}~\pageref{def:MPI_ERR_ARG}
\item \texttt{MPI_ERR_BUFFER}~\pageref{def:MPI_ERR_BUFFER}
\item \texttt{MPI_ERR_COMM}~\pageref{def:MPI_ERR_COMM}
\item \texttt{MPI_ERR_COUNT}~\pageref{def:MPI_ERR_COUNT}
\item \texttt{MPI_ERR_DIMS}~\pageref{def:MPI_ERR_DIMS}
\item \texttt{MPI_ERR_GROUP}~\pageref{def:MPI_ERR_GROUP}
\item \texttt{MPI_ERR_INTERN}~\pageref{def:MPI_ERR_INTERN}
\item \texttt{MPI_ERR_IN_STATUS}~\pageref{def:MPI_ERR_IN_STATUS}
\item \texttt{MPI_ERR_LASTCODE}~\pageref{def:MPI_ERR_LASTCODE}
\item \texttt{MPI_ERR_OP}~\pageref{def:MPI_ERR_OP}
\item \texttt{MPI_ERR_OTHER}~\pageref{def:MPI_ERR_OTHER}
\item \texttt{MPI_ERR_PENDING}~\pageref{def:MPI_ERR_PENDING}
\item \texttt{MPI_ERR_RANK}~\pageref{def:MPI_ERR_RANK}
\item \texttt{MPI_ERR_REQUEST}~\pageref{def:MPI_ERR_REQUEST}
\item \texttt{MPI_ERR_RMA_ATTACH}~\pageref{def:MPI_ERR_RMA_ATTACH}
\item \texttt{MPI_ERR_RMA_FLAVOR}~\pageref{def:MPI_ERR_RMA_FLAVOR}
\item \texttt{MPI_ERR_RMA_RANGE}~\pageref{def:MPI_ERR_RMA_RANGE}
\item \texttt{MPI_ERR_RMA_SHARED}~\pageref{def:MPI_ERR_RMA_SHARED}
\item \texttt{MPI_ERR_ROOT}~\pageref{def:MPI_ERR_ROOT}
\item \texttt{MPI_ERR_SESSION}~\pageref{def:MPI_ERR_SESSION}
\item \texttt{MPI_ERR_TAG}~\pageref{def:MPI_ERR_TAG}
\item \texttt{MPI_ERR_TOPOLOGY}~\pageref{def:MPI_ERR_TOPOLOGY}
\item \texttt{MPI_ERR_TRUNCATE}~\pageref{def:MPI_ERR_TRUNCATE}
\item \texttt{MPI_ERR_TYPE}~\pageref{def:MPI_ERR_TYPE}
\item \texttt{MPI_ERR_UNKNOWN}~\pageref{def:MPI_ERR_UNKNOWN}
\item \texttt{MPI_ERR_VALUE_TOO_LARGE}~\pageref{def:MPI_ERR_VALUE_TOO_LARGE}
\item \texttt{MPI_Errhandler}~\pageref{def:MPI_Errhandler}
\item \texttt{MPI_FILE_NULL}~\pageref{def:MPI_FILE_NULL}
\item \texttt{MPI_FLOAT_INT}~\pageref{def:MPI_FLOAT_INT}
\item \texttt{MPI_F_ERROR}~\pageref{def:MPI_F_ERROR}
\item \texttt{MPI_F_SOURCE}~\pageref{def:MPI_F_SOURCE}
\item \texttt{MPI_F_STATUSES_IGNORE}~\pageref{def:MPI_F_STATUSES_IGNORE}
\item \texttt{MPI_F_STATUS_IGNORE}~\pageref{def:MPI_F_STATUS_IGNORE}
\item \texttt{MPI_F_STATUS_SIZE}~\pageref{def:MPI_F_STATUS_SIZE}
\item \texttt{MPI_F_TAG}~\pageref{def:MPI_F_TAG}
\item \texttt{MPI_File}~\pageref{def:MPI_File}
\item \texttt{MPI_Fint}~\pageref{def:MPI_Fint}
\item \texttt{MPI_GRAPH}~\pageref{def:MPI_GRAPH}
\item \texttt{MPI_GROUP_EMPTY}~\pageref{def:MPI_GROUP_EMPTY}
\item \texttt{MPI_GROUP_NULL}~\pageref{def:MPI_GROUP_NULL}
\item \texttt{MPI_Group}~\pageref{def:MPI_Group}
\item \texttt{MPI_HOST}~\pageref{def:MPI_HOST}
\item \texttt{MPI_IDENT}~\pageref{def:MPI_IDENT}
\item \texttt{MPI_INFO_ENV}~\pageref{def:MPI_INFO_ENV}
\item \texttt{MPI_INFO_NULL}~\pageref{def:MPI_INFO_NULL}
\item \texttt{MPI_INTEGER_KIND}~\pageref{def:MPI_INTEGER_KIND}
\item \texttt{MPI_IN_PLACE}~\pageref{def:MPI_IN_PLACE}
\item \texttt{MPI_IO}~\pageref{def:MPI_IO}
\item \texttt{MPI_Info}~\pageref{def:MPI_Info}
\item \texttt{MPI_KEYVAL_INVALID}~\pageref{def:MPI_KEYVAL_INVALID}
\item \texttt{MPI_LAND}~\pageref{def:MPI_LAND}
\item \texttt{MPI_LASTUSEDCODE}~\pageref{def:MPI_LASTUSEDCODE}
\item \texttt{MPI_LB}~\pageref{def:MPI_LB}
\item \texttt{MPI_LOCK_EXCLUSIVE}~\pageref{def:MPI_LOCK_EXCLUSIVE}
\item \texttt{MPI_LOCK_SHARED}~\pageref{def:MPI_LOCK_SHARED}
\item \texttt{MPI_LOR}~\pageref{def:MPI_LOR}
\item \texttt{MPI_LXOR}~\pageref{def:MPI_LXOR}
\item \texttt{MPI_MAX}~\pageref{def:MPI_MAX}
\item \texttt{MPI_MAXLOC}~\pageref{def:MPI_MAXLOC}
\item \texttt{MPI_MAX_DATAREP_STRING}~\pageref{def:MPI_MAX_DATAREP_STRING}
\item \texttt{MPI_MAX_ERROR_STRING}~\pageref{def:MPI_MAX_ERROR_STRING}
\item \texttt{MPI_MAX_INFO_KEY}~\pageref{def:MPI_MAX_INFO_KEY}
\item \texttt{MPI_MAX_INFO_VAL}~\pageref{def:MPI_MAX_INFO_VAL}
\item \texttt{MPI_MAX_LIBRARY_VERSION_STRING}~\pageref{def:MPI_MAX_LIBRARY_VERSION_STRING}
\item \texttt{MPI_MAX_OBJECT_NAME}~\pageref{def:MPI_MAX_OBJECT_NAME}
\item \texttt{MPI_MAX_PORT_NAME}~\pageref{def:MPI_MAX_PORT_NAME}
\item \texttt{MPI_MAX_PROCESSOR_NAME}~\pageref{def:MPI_MAX_PROCESSOR_NAME}
\item \texttt{MPI_MAX_PSET_NAME_LEN}~\pageref{def:MPI_MAX_PSET_NAME_LEN}
\item \texttt{MPI_MAX_STRINGTAG_LEN}~\pageref{def:MPI_MAX_STRINGTAG_LEN}
\item \texttt{MPI_MESSAGE_NO_PROC}~\pageref{def:MPI_MESSAGE_NO_PROC}
\item \texttt{MPI_MESSAGE_NULL}~\pageref{def:MPI_MESSAGE_NULL}
\item \texttt{MPI_MIN}~\pageref{def:MPI_MIN}
\item \texttt{MPI_MINLOC}~\pageref{def:MPI_MINLOC}
\item \texttt{MPI_MODE_APPEND}~\pageref{def:MPI_MODE_APPEND}
\item \texttt{MPI_MODE_CREATE}~\pageref{def:MPI_MODE_CREATE}
\item \texttt{MPI_MODE_DELETE_ON_CLOSE}~\pageref{def:MPI_MODE_DELETE_ON_CLOSE}
\item \texttt{MPI_MODE_EXCL}~\pageref{def:MPI_MODE_EXCL}
\item \texttt{MPI_MODE_NOCHECK}~\pageref{def:MPI_MODE_NOCHECK}
\item \texttt{MPI_MODE_NOPRECEDE}~\pageref{def:MPI_MODE_NOPRECEDE}
\item \texttt{MPI_MODE_NOPUT}~\pageref{def:MPI_MODE_NOPUT}
\item \texttt{MPI_MODE_NOSTORE}~\pageref{def:MPI_MODE_NOSTORE}
\item \texttt{MPI_MODE_NOSUCCEED}~\pageref{def:MPI_MODE_NOSUCCEED}
\item \texttt{MPI_MODE_RDONLY}~\pageref{def:MPI_MODE_RDONLY}
\item \texttt{MPI_MODE_RDWR}~\pageref{def:MPI_MODE_RDWR}
\item \texttt{MPI_MODE_SEQUENTIAL}~\pageref{def:MPI_MODE_SEQUENTIAL}
\item \texttt{MPI_MODE_UNIQUE_OPEN}~\pageref{def:MPI_MODE_UNIQUE_OPEN}
\item \texttt{MPI_MODE_WRONLY}~\pageref{def:MPI_MODE_WRONLY}
\item \texttt{MPI_Message}~\pageref{def:MPI_Message}
\item \texttt{MPI_NO_OP}~\pageref{def:MPI_NO_OP}
\item \texttt{MPI_OFFSET_KIND}~\pageref{def:MPI_OFFSET_KIND}
\item \texttt{MPI_OP_NULL}~\pageref{def:MPI_OP_NULL}
\item \texttt{MPI_ORDER_C}~\pageref{def:MPI_ORDER_C}
\item \texttt{MPI_ORDER_FORTRAN}~\pageref{def:MPI_ORDER_FORTRAN}
\item \texttt{MPI_Offset}~\pageref{def:MPI_Offset}
\item \texttt{MPI_Op}~\pageref{def:MPI_Op}
\item \texttt{MPI_PROC_NULL}~\pageref{def:MPI_PROC_NULL}
\item \texttt{MPI_PROD}~\pageref{def:MPI_PROD}
\item \texttt{MPI_REPLACE}~\pageref{def:MPI_REPLACE}
\item \texttt{MPI_REQUEST_NULL}~\pageref{def:MPI_REQUEST_NULL}
\item \texttt{MPI_ROOT}~\pageref{def:MPI_ROOT}
\item \texttt{MPI_Request}~\pageref{def:MPI_Request}
\item \texttt{MPI_SEEK_CUR}~\pageref{def:MPI_SEEK_CUR}
\item \texttt{MPI_SEEK_END}~\pageref{def:MPI_SEEK_END}
\item \texttt{MPI_SEEK_SET}~\pageref{def:MPI_SEEK_SET}
\item \texttt{MPI_SESSION_NULL}~\pageref{def:MPI_SESSION_NULL}
\item \texttt{MPI_SHORT_INT}~\pageref{def:MPI_SHORT_INT}
\item \texttt{MPI_SIMILAR}~\pageref{def:MPI_SIMILAR}
\item \texttt{MPI_SOURCE}~\pageref{def:MPI_SOURCE}
\item \texttt{MPI_STATUSES_IGNORE}~\pageref{def:MPI_STATUSES_IGNORE}
\item \texttt{MPI_STATUS_IGNORE}~\pageref{def:MPI_STATUS_IGNORE}
\item \texttt{MPI_STATUS_SIZE}~\pageref{def:MPI_STATUS_SIZE}
\item \texttt{MPI_SUBARRAYS_SUPPORTED}~\pageref{def:MPI_SUBARRAYS_SUPPORTED}
\item \texttt{MPI_SUBVERSION}~\pageref{def:MPI_SUBVERSION}
\item \texttt{MPI_SUCCESS}~\pageref{def:MPI_SUCCESS}
\item \texttt{MPI_SUM}~\pageref{def:MPI_SUM}
\item \texttt{MPI_Session}~\pageref{def:MPI_Session}
\item \texttt{MPI_Status}~\pageref{def:MPI_Status}
\item \texttt{MPI_TAG}~\pageref{def:MPI_TAG}
\item \texttt{MPI_TAG_UB}~\pageref{def:MPI_TAG_UB}
\item \texttt{MPI_THREAD_FUNNELED}~\pageref{def:MPI_THREAD_FUNNELED}
\item \texttt{MPI_THREAD_MULTIPLE}~\pageref{def:MPI_THREAD_MULTIPLE}
\item \texttt{MPI_THREAD_SERIALIZED}~\pageref{def:MPI_THREAD_SERIALIZED}
\item \texttt{MPI_THREAD_SINGLE}~\pageref{def:MPI_THREAD_SINGLE}
\item \texttt{MPI_TYPECLASS_COMPLEX}~\pageref{def:MPI_TYPECLASS_COMPLEX}
\item \texttt{MPI_TYPECLASS_INTEGER}~\pageref{def:MPI_TYPECLASS_INTEGER}
\item \texttt{MPI_TYPECLASS_REAL}~\pageref{def:MPI_TYPECLASS_REAL}
\item \texttt{MPI_T_BIND_MPI_COMM}~\pageref{def:MPI_T_BIND_MPI_COMM}
\item \texttt{MPI_T_BIND_MPI_DATATYPE}~\pageref{def:MPI_T_BIND_MPI_DATATYPE}
\item \texttt{MPI_T_BIND_MPI_ERRHANDLER}~\pageref{def:MPI_T_BIND_MPI_ERRHANDLER}
\item \texttt{MPI_T_BIND_MPI_FILE}~\pageref{def:MPI_T_BIND_MPI_FILE}
\item \texttt{MPI_T_BIND_MPI_GROUP}~\pageref{def:MPI_T_BIND_MPI_GROUP}
\item \texttt{MPI_T_BIND_MPI_INFO}~\pageref{def:MPI_T_BIND_MPI_INFO}
\item \texttt{MPI_T_BIND_MPI_MESSAGE}~\pageref{def:MPI_T_BIND_MPI_MESSAGE}
\item \texttt{MPI_T_BIND_MPI_OP}~\pageref{def:MPI_T_BIND_MPI_OP}
\item \texttt{MPI_T_BIND_MPI_REQUEST}~\pageref{def:MPI_T_BIND_MPI_REQUEST}
\item \texttt{MPI_T_BIND_MPI_SESSION}~\pageref{def:MPI_T_BIND_MPI_SESSION}
\item \texttt{MPI_T_BIND_MPI_WIN}~\pageref{def:MPI_T_BIND_MPI_WIN}
\item \texttt{MPI_T_BIND_NO_OBJECT}~\pageref{def:MPI_T_BIND_NO_OBJECT}
\item \texttt{MPI_T_CB_REQUIRE_ASYNC_SIGNAL_SAFE}~\pageref{def:MPI_T_CB_REQUIRE_ASYNC_SIGNAL_SAFE}
\item \texttt{MPI_T_CB_REQUIRE_MPI_RESTRICTED}~\pageref{def:MPI_T_CB_REQUIRE_MPI_RESTRICTED}
\item \texttt{MPI_T_CB_REQUIRE_NONE}~\pageref{def:MPI_T_CB_REQUIRE_NONE}
\item \texttt{MPI_T_CB_REQUIRE_THREAD_SAFE}~\pageref{def:MPI_T_CB_REQUIRE_THREAD_SAFE}
\item \texttt{MPI_T_CVAR_HANDLE_NULL}~\pageref{def:MPI_T_CVAR_HANDLE_NULL}
\item \texttt{MPI_T_ENUM_NULL}~\pageref{def:MPI_T_ENUM_NULL}
\item \texttt{MPI_T_ERR_CANNOT_INIT}~\pageref{def:MPI_T_ERR_CANNOT_INIT}
\item \texttt{MPI_T_ERR_CVAR_SET_NEVER}~\pageref{def:MPI_T_ERR_CVAR_SET_NEVER}
\item \texttt{MPI_T_ERR_CVAR_SET_NOT_NOW}~\pageref{def:MPI_T_ERR_CVAR_SET_NOT_NOW}
\item \texttt{MPI_T_ERR_INVALID}~\pageref{def:MPI_T_ERR_INVALID}
\item \texttt{MPI_T_ERR_INVALID_HANDLE}~\pageref{def:MPI_T_ERR_INVALID_HANDLE}
\item \texttt{MPI_T_ERR_INVALID_INDEX}~\pageref{def:MPI_T_ERR_INVALID_INDEX}
\item \texttt{MPI_T_ERR_INVALID_ITEM}~\pageref{def:MPI_T_ERR_INVALID_ITEM}
\item \texttt{MPI_T_ERR_INVALID_NAME}~\pageref{def:MPI_T_ERR_INVALID_NAME}
\item \texttt{MPI_T_ERR_INVALID_SESSION}~\pageref{def:MPI_T_ERR_INVALID_SESSION}
\item \texttt{MPI_T_ERR_MEMORY}~\pageref{def:MPI_T_ERR_MEMORY}
\item \texttt{MPI_T_ERR_NOT_ACCESSIBLE}~\pageref{def:MPI_T_ERR_NOT_ACCESSIBLE}
\item \texttt{MPI_T_ERR_NOT_INITIALIZED}~\pageref{def:MPI_T_ERR_NOT_INITIALIZED}
\item \texttt{MPI_T_ERR_NOT_SUPPORTED}~\pageref{def:MPI_T_ERR_NOT_SUPPORTED}
\item \texttt{MPI_T_ERR_OUT_OF_HANDLES}~\pageref{def:MPI_T_ERR_OUT_OF_HANDLES}
\item \texttt{MPI_T_ERR_OUT_OF_SESSIONS}~\pageref{def:MPI_T_ERR_OUT_OF_SESSIONS}
\item \texttt{MPI_T_ERR_PVAR_NO_ATOMIC}~\pageref{def:MPI_T_ERR_PVAR_NO_ATOMIC}
\item \texttt{MPI_T_ERR_PVAR_NO_STARTSTOP}~\pageref{def:MPI_T_ERR_PVAR_NO_STARTSTOP}
\item \texttt{MPI_T_ERR_PVAR_NO_WRITE}~\pageref{def:MPI_T_ERR_PVAR_NO_WRITE}
\item \texttt{MPI_T_PVAR_ALL_HANDLES}~\pageref{def:MPI_T_PVAR_ALL_HANDLES}
\item \texttt{MPI_T_PVAR_CLASS_AGGREGATE}~\pageref{def:MPI_T_PVAR_CLASS_AGGREGATE}
\item \texttt{MPI_T_PVAR_CLASS_COUNTER}~\pageref{def:MPI_T_PVAR_CLASS_COUNTER}
\item \texttt{MPI_T_PVAR_CLASS_GENERIC}~\pageref{def:MPI_T_PVAR_CLASS_GENERIC}
\item \texttt{MPI_T_PVAR_CLASS_HIGHWATERMARK}~\pageref{def:MPI_T_PVAR_CLASS_HIGHWATERMARK}
\item \texttt{MPI_T_PVAR_CLASS_LEVEL}~\pageref{def:MPI_T_PVAR_CLASS_LEVEL}
\item \texttt{MPI_T_PVAR_CLASS_LOWWATERMARK}~\pageref{def:MPI_T_PVAR_CLASS_LOWWATERMARK}
\item \texttt{MPI_T_PVAR_CLASS_PERCENTAGE}~\pageref{def:MPI_T_PVAR_CLASS_PERCENTAGE}
\item \texttt{MPI_T_PVAR_CLASS_SIZE}~\pageref{def:MPI_T_PVAR_CLASS_SIZE}
\item \texttt{MPI_T_PVAR_CLASS_STATE}~\pageref{def:MPI_T_PVAR_CLASS_STATE}
\item \texttt{MPI_T_PVAR_CLASS_TIMER}~\pageref{def:MPI_T_PVAR_CLASS_TIMER}
\item \texttt{MPI_T_PVAR_HANDLE_NULL}~\pageref{def:MPI_T_PVAR_HANDLE_NULL}
\item \texttt{MPI_T_PVAR_SESSION_NULL}~\pageref{def:MPI_T_PVAR_SESSION_NULL}
\item \texttt{MPI_T_SCOPE_ALL}~\pageref{def:MPI_T_SCOPE_ALL}
\item \texttt{MPI_T_SCOPE_ALL_EQ}~\pageref{def:MPI_T_SCOPE_ALL_EQ}
\item \texttt{MPI_T_SCOPE_CONSTANT}~\pageref{def:MPI_T_SCOPE_CONSTANT}
\item \texttt{MPI_T_SCOPE_GROUP}~\pageref{def:MPI_T_SCOPE_GROUP}
\item \texttt{MPI_T_SCOPE_GROUP_EQ}~\pageref{def:MPI_T_SCOPE_GROUP_EQ}
\item \texttt{MPI_T_SCOPE_LOCAL}~\pageref{def:MPI_T_SCOPE_LOCAL}
\item \texttt{MPI_T_SCOPE_READONLY}~\pageref{def:MPI_T_SCOPE_READONLY}
\item \texttt{MPI_T_SOURCE_ORDERED}~\pageref{def:MPI_T_SOURCE_ORDERED}
\item \texttt{MPI_T_SOURCE_UNORDERED}~\pageref{def:MPI_T_SOURCE_UNORDERED}
\item \texttt{MPI_T_VERBOSITY_MPIDEV_ALL}~\pageref{def:MPI_T_VERBOSITY_MPIDEV_ALL}
\item \texttt{MPI_T_VERBOSITY_MPIDEV_BASIC}~\pageref{def:MPI_T_VERBOSITY_MPIDEV_BASIC}
\item \texttt{MPI_T_VERBOSITY_MPIDEV_DETAIL}~\pageref{def:MPI_T_VERBOSITY_MPIDEV_DETAIL}
\item \texttt{MPI_T_VERBOSITY_TUNER_ALL}~\pageref{def:MPI_T_VERBOSITY_TUNER_ALL}
\item \texttt{MPI_T_VERBOSITY_TUNER_BASIC}~\pageref{def:MPI_T_VERBOSITY_TUNER_BASIC}
\item \texttt{MPI_T_VERBOSITY_TUNER_DETAIL}~\pageref{def:MPI_T_VERBOSITY_TUNER_DETAIL}
\item \texttt{MPI_T_VERBOSITY_USER_ALL}~\pageref{def:MPI_T_VERBOSITY_USER_ALL}
\item \texttt{MPI_T_VERBOSITY_USER_BASIC}~\pageref{def:MPI_T_VERBOSITY_USER_BASIC}
\item \texttt{MPI_T_VERBOSITY_USER_DETAIL}~\pageref{def:MPI_T_VERBOSITY_USER_DETAIL}
\item \texttt{MPI_T_cb_safety}~\pageref{def:MPI_T_cb_safety}
\item \texttt{MPI_T_cvar_handle}~\pageref{def:MPI_T_cvar_handle}
\item \texttt{MPI_T_enum}~\pageref{def:MPI_T_enum}
\item \texttt{MPI_T_event_instance}~\pageref{def:MPI_T_event_instance}
\item \texttt{MPI_T_event_registration}~\pageref{def:MPI_T_event_registration}
\item \texttt{MPI_T_pvar_handle}~\pageref{def:MPI_T_pvar_handle}
\item \texttt{MPI_T_pvar_session}~\pageref{def:MPI_T_pvar_session}
\item \texttt{MPI_T_source_order}~\pageref{def:MPI_T_source_order}
\item \texttt{MPI_UB}~\pageref{def:MPI_UB}
\item \texttt{MPI_UNDEFINED}~\pageref{def:MPI_UNDEFINED}
\item \texttt{MPI_UNEQUAL}~\pageref{def:MPI_UNEQUAL}
\item \texttt{MPI_UNIVERSE_SIZE}~\pageref{def:MPI_UNIVERSE_SIZE}
\item \texttt{MPI_UNWEIGHTED}~\pageref{def:MPI_UNWEIGHTED}
\item \texttt{MPI_VAL}~\pageref{def:MPI_VAL}
\item \texttt{MPI_VERSION}~\pageref{def:MPI_VERSION}
\item \texttt{MPI_WEIGHTS_EMPTY}~\pageref{def:MPI_WEIGHTS_EMPTY}
\item \texttt{MPI_WIN_BASE}~\pageref{def:MPI_WIN_BASE}
\item \texttt{MPI_WIN_CREATE_FLAVOR}~\pageref{def:MPI_WIN_CREATE_FLAVOR}
\item \texttt{MPI_WIN_DISP_UNIT}~\pageref{def:MPI_WIN_DISP_UNIT}
\item \texttt{MPI_WIN_FLAVOR_ALLOCATE}~\pageref{def:MPI_WIN_FLAVOR_ALLOCATE}
\item \texttt{MPI_WIN_FLAVOR_CREATE}~\pageref{def:MPI_WIN_FLAVOR_CREATE}
\item \texttt{MPI_WIN_FLAVOR_DYNAMIC}~\pageref{def:MPI_WIN_FLAVOR_DYNAMIC}
\item \texttt{MPI_WIN_FLAVOR_SHARED}~\pageref{def:MPI_WIN_FLAVOR_SHARED}
\item \texttt{MPI_WIN_MODEL}~\pageref{def:MPI_WIN_MODEL}
\item \texttt{MPI_WIN_NULL}~\pageref{def:MPI_WIN_NULL}
\item \texttt{MPI_WIN_SEPARATE}~\pageref{def:MPI_WIN_SEPARATE}
\item \texttt{MPI_WIN_SIZE}~\pageref{def:MPI_WIN_SIZE}
\item \texttt{MPI_WIN_UNIFIED}~\pageref{def:MPI_WIN_UNIFIED}
\item \texttt{MPI_WTIME_IS_GLOBAL}~\pageref{def:MPI_WTIME_IS_GLOBAL}
\item \texttt{MPI_Win}~\pageref{def:MPI_Win}
\end{itemize}
\end{multicols}

\Level 2 {List of all callbacks}
\begin{multicols}{3}
\catcode`\_=12
\footnotesize
\begin{itemize}
\item \texttt{COMM_COPY_ATTR_FUNCTION}~\pageref{def:COMM_COPY_ATTR_FUNCTION}
\item \texttt{COMM_DELETE_ATTR_FUNCTION}~\pageref{def:COMM_DELETE_ATTR_FUNCTION}
\item \texttt{COPY_FUNCTION}~\pageref{def:COPY_FUNCTION}
\item \texttt{DELETE_FUNCTION}~\pageref{def:DELETE_FUNCTION}
\item \texttt{MPI_Comm_copy_attr_function}~\pageref{def:MPI_Comm_copy_attr_function}
\item \texttt{MPI_Comm_delete_attr_function}~\pageref{def:MPI_Comm_delete_attr_function}
\item \texttt{MPI_Comm_errhandler_function}~\pageref{def:MPI_Comm_errhandler_function}
\item \texttt{MPI_Copy_function}~\pageref{def:MPI_Copy_function}
\item \texttt{MPI_Datarep_conversion_function}~\pageref{def:MPI_Datarep_conversion_function}
\item \texttt{MPI_Datarep_conversion_function_c}~\pageref{def:MPI_Datarep_conversion_function_c}
\item \texttt{MPI_Delete_function}~\pageref{def:MPI_Delete_function}
\item \texttt{MPI_File_errhandler_function}~\pageref{def:MPI_File_errhandler_function}
\item \texttt{MPI_Handler_function}~\pageref{def:MPI_Handler_function}
\item \texttt{MPI_Session_errhandler_function}~\pageref{def:MPI_Session_errhandler_function}
\item \texttt{MPI_Type_delete_attr_function}~\pageref{def:MPI_Type_delete_attr_function}
\item \texttt{MPI_User_function}~\pageref{def:MPI_User_function}
\item \texttt{MPI_User_function_c}~\pageref{def:MPI_User_function_c}
\item \texttt{MPI_Win_errhandler_function}~\pageref{def:MPI_Win_errhandler_function}
\end{itemize}
\end{multicols}


\Level 1 {MPI for Python}

\Level 2 {Buffer specifications}

{\small
%\begin{multicols}{2}
  \verbatiminput{standardp/bufspec.tex}
  \verbatiminput{standardp/bufspecb.tex}
  \verbatiminput{standardp/bufspecv.tex}
  \verbatiminput{standardp/bufspecw.tex}
%\end{multicols}
}

\Level 2 {Listing of python routines}

\begin{multicols}{3}
\small
\pylist{Comm}
\pylist{Cartcomm}
\pylist{Distgraphcomm}
\pylist{Graphcomm}
\pylist{Intercomm}
\pylist{Intracomm}
\pylist{Topocomm}
\pylist{Group}

\pylist{Request}
\pylist{Grequest}
\pylist{Prequest}
\pylist{Status}

\pylist{Win}

\pylist{Datatype}
\pylist{File}
\pylist{Info}
\pylist{Op}

\pylist{Errhandler}
\pylist{Message}
%\pylist{_InPlace}
%\pylist{Exception}
%\pylist{_Pickle}
\end{multicols}

\Level 0 {Index of OpenMP keywords}

\begin{multicols*}{2}
\printindex[omp]
\end{multicols*}

\Level 0 {Index of PETSc keywords}

\begin{multicols*}{2}
\printindex[petsc]
\end{multicols*}

\Level 0 {Index of KOKKOS keywords}

\begin{multicols*}{2}
\printindex[kokkos]
\end{multicols*}

\Level 0 {Index of SYCL keywords}

\begin{multicols*}{2}
\printindex[sycl]
\end{multicols*}

\hbox{}\vfill
\includegraphics{isbnbarcode}

\closeout\chapterlist
\end{document}
