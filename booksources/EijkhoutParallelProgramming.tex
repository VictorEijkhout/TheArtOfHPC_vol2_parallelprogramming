% -*- latex -*-
%%%%%%%%%%%%%%%%%%%%%%%%%%%%%%%%%%%%%%%%%%%%%%%%%%%%%%%%%%%%%%%%
%%%%%%%%%%%%%%%%%%%%%%%%%%%%%%%%%%%%%%%%%%%%%%%%%%%%%%%%%%%%%%%%
%%%%
%%%% This text file is part of the source of 
%%%% `Parallel Computing'
%%%% by Victor Eijkhout, copyright 2012-2025
%%%%
%%%% parcompbook.tex : master file for the book
%%%%
%%%%%%%%%%%%%%%%%%%%%%%%%%%%%%%%%%%%%%%%%%%%%%%%%%%%%%%%%%%%%%%%
%%%%%%%%%%%%%%%%%%%%%%%%%%%%%%%%%%%%%%%%%%%%%%%%%%%%%%%%%%%%%%%%

\documentclass[11pt,letterpaper,twoside,openany]{boek3}
%\documentclass{book}

\usepackage{verbatim}

\input macros/comment.sty % our own prerelease
%\usepackage{comment}

\specialcomment{tacc}{\def\CommentCutFile{tacc.cut}}{}
\newif\ifIncludeAnswers
\IncludeAnswersfalse
\input inex
\includecomment{gpu}
\includecomment{review}
\includecomment{book}

%% arbitrary numbers of output streams
\usepackage{morewrites}

% fancy text stuff
\usepackage{fontspec}
\setmainfont[
  Extension=.otf,
  UprightFont={*-Regular},
  BoldFont={*-Bold},
  ItalicFont={*-Italic},
  BoldItalicFont={*-BoldItalic}
]{LibertinusSerif}
\usepackage{unicode-math}
\setmathfont{LibertinusMath-Regular.otf}
%%%%%%%%%%%%%%%%%%%
\usepackage{dirtree} % ,times
% table stuff
\usepackage{booktabs,multicol,multirow}

% AMS math
%\usepackage[fleqn]{amsmath}
%\usepackage{amssymb}

\def\svnrev{428}
% dashed lines; this may interfere with other table packages
% \usepackage{arydshln}

\edef\revision{\svnrev}
\def\lulurevision{}

\newcommand\codesnippetsdir{snippets}
\newcommand\latexengine{,xetex}
\input commonmacs
\input commonbookmacs
\input pcsecommonmacs
\input acromacs
\input blockmacs
\input exmacs
\input standardmacs
\input tutmacs
%%
%% indexing
%%
\usepackage{morewrites}
\usepackage[original,nonewpage]{imakeidx}
\makeindex % default index
\input idxmacs
\input alsoidxmacs
\input pcseidxmacs

\input bookmacs
\input pcsebookmacs
\input listingmacs
\input pcselistingmacs
\def\snippetstyle{booksnippetcode}
\lstset{style=\snippetstyle}
\input snippetmacs
\input pcsesnippetmacs

\ifincludesources
  \def\chaptersourcelisting#1{
    \newpage
    \listchaptersources{#1}
  }
\else
  \def\chaptersourcelisting#1{
    %% \newpage
    %% \Level 0 {Full source code of examples}
    %% Please see the repository:
    %% \expandafter\url\expandafter{\SourceRepoRoot}.
    %% Examples are sorted first by programming system, then by language.
  }
\fi

%% \end{notlulu}

%%
%% Title and front matter
%%
\def\publicdraft{{\bf\normalsize \relax Public draft - open for comments}}
\def\revdate{2nd edition 2022, formatted \today\\
  \small
  Book and slides download: \url{https://tinyurl.com/vle335course}\\
  Public repository:
      \url{https://github.com/VictorEijkhout/TheArtOfHPC_vol2_parallelprogramming}\\
  HTML version: \url{https://theartofhpc.com/pcse/}\\ [20pt]
  This book is published under the CC-BY 4.0 license.
}
\begin{lulu}
  \def\publicdraft{}
  \def\revdate{2nd edition 2022\\
    Series reference: \url{https://theartofhpc.com/}
  }
\end{lulu}

\newwrite\chapterlist \openout\chapterlist=chapternames.tex

\begin{document}

\author{Victor Eijkhout}
\title{Parallel Programming in MPI and OpenMP\\
  \small The Art of HPC, volume 2}
\expandafter\date\expandafter{\revdate}
\maketitle

\input introduction
\vfill\pagebreak 

{\setcounter{tocdepth}{1}
\tableofcontents
\setcounter{tocdepth}{2}
}

\acresetall
\part{MPI}
\label{part:MPI}
\addcontentsline{locpp}{cppnote}{MPI}
\addcontentsline{loftn}{fortrannote}{MPI}
\addcontentsline{lopy}{pythonnote}{MPI}

\input chapters/mpi-competency
\CHAPTER{Getting started with MPI}{mpi-started}
\CHAPTER{MPI topic: Functional parallelism}{mpi-functional}
\CHAPTER{MPI topic: Collectives}{mpi-collective}
\CHAPTER{MPI topic: Point-to-point}{mpi-ptp}
%\end{document}
\CHAPTER{MPI topic: Communication modes}{mpi-persist}
\CHAPTER{MPI topic: Data types}{mpi-data}
\CHAPTER{MPI topic: Communicators}{mpi-comm}
\CHAPTER{MPI topic: Process management}{mpi-proc}
\CHAPTER{MPI topic: One-sided communication}{mpi-onesided}
\CHAPTER{MPI topic: File I/O}{mpi-io}
\CHAPTER{MPI topic: Topologies}{mpi-topo}
\CHAPTER{MPI topic: Shared memory}{mpi-shared}
\CHAPTER{MPI topic: Hybrid computing}{mpi-hybrid}
\CHAPTER{MPI topic: Tools interface}{mpi-tools}
\CHAPTER{MPI leftover topics}{mpi}
\CHAPTER{MPI Examples}{mpi-examples}
%\CHAPTER{MPI Review}{mpireview}

\acresetall
\part{OpenMP}
\label{part:OpenMP}
\addcontentsline{locpp}{cppnote}{OpenMP}
\addcontentsline{loftn}{fortrannote}{OpenMP}
\addcontentsline{lopy}{pythonnote}{OpenMP}

\input chapters/omp-competency
\CHAPTER{Getting started with OpenMP}{omp-basics}
\CHAPTER{OpenMP topic: Parallel regions}{omp-parallel}
\CHAPTER{OpenMP topic: Loop parallelism}{omp-loop}
\CHAPTER{OpenMP topic: Reductions}{omp-reduction}
\CHAPTER{OpenMP topic: Work sharing}{omp-share}
\CHAPTER{OpenMP topic: Controlling thread data}{omp-data}
\CHAPTER{OpenMP topic: Synchronization}{omp-sync}
\CHAPTER{OpenMP topic: Tasks}{omp-task}
\CHAPTER{OpenMP topic: Affinity}{omp-affinity}
%% \CHAPTER{OpenMP topic: Memory model}{omp-memory}
\CHAPTER{OpenMP topic: SIMD processing}{omp-simd}
\CHAPTER{OpenMP topic: Offloading}{omp-gpu}

\CHAPTER{OpenMP remaining topics}{openmp}
%%\CHAPTER{OpenMP Reference}{ompref}
%% \CHAPTER{OpenMP Review}{ompreview}
\CHAPTER{OpenMP Exercises and examples}{omp-examples}

\part{PETSc}
\label{part:petsc}
\addcontentsline{locpp}{cppnote}{PETSc}
\addcontentsline{loftn}{fortrannote}{PETSc}
\addcontentsline{lopy}{pythonnote}{PETSc}

\CHAPTER{PETSc basics}{petsc-design}
\CHAPTER{PETSc objects}{petsc-objects}
\CHAPTER{Grid support}{petsc-dmda}
\CHAPTER{Finite Elements support}{petsc-fem}
\CHAPTER{PETSc solvers}{petsc-solver}
\CHAPTER{PETSC nonlinear solvers}{petsc-nonlinear}
\CHAPTER{PETSc GPU support}{petsc-gpu}
\CHAPTER{PETSc tools}{petsc-tools}
\CHAPTER{PETSc topics}{petsc}

\part{Other programming models}
\addcontentsline{loftn}{fortrannote}{Other}
\addcontentsline{lopy}{fortrannote}{Other}

\input otherblurb

\lstset{language=Fortran}
\CHAPTER{Co-array Fortran}{caf}
\lstset{language=C++}
\CHAPTER{Kokkos}{kokkos}
\CHAPTER{Sycl, OneAPI, DPC++}{dpcpp}
\CHAPTER{CUDA}{cuda}
\CHAPTER{Python multiprocessing}{multiprocessing}

\part{The Rest}

%\CHAPTER{Ruminations on parallelism}{patterns}
\CHAPTER{Exploring computer architecture}{architecture}
%% merged into next \CHAPTER{Process and thread affinity}{affinity}
\CHAPTER{Hybrid computing}{hybrid}
%% \CHAPTER{Parallel I/O}{io}
\CHAPTER{Support libraries}{libraries}

\part{Class projects}

\PROJECT{A Style Guide to Project Submissions}{projectstyle}
\PROJECT{Warmup Exercises}{warmup}
\PROJECT{Mandelbrot set}{mandelbrot}
\PROJECT{Data parallel grids}{grid}
\PROJECT{N-body problems}{nbody}

%pyskipbegin
\part{Didactics}

\CHAPTER{Teaching guide}{mpi-course}
\CHAPTER{Teaching from mental models}{mpi-mental}
%\CHAPTER{Parallel Programming Explained through Conway's Game Of Life}{conwaysection}
%pyskipend



\part {Bibliography, index, and list of acronyms}

\Level 0 {Bibliography}

\bibliography{vle}
\bibliographystyle{plain}
\vfill\pagebreak

\Level 0 {List of acronyms}

\def\acitem#1#2{\item[#1] #2}
\def\acitemi#1#2#3{\item[#1]{#2}\index{#1|see{#3}}}

\begin{multicols}{2}
\begin{description}
\input acronyms
\end{description}
\end{multicols}

\Level 0 {General Index}

\index{parallel!prefix|see{prefix}}

\printindex

\Level 0 {Lists of notes}

\Level 1 {MPI-4 notes}

\listofmpifournote
\vfill\hbox{}

\Level 1 {Fortran notes}

\listoffortrannote
\vfill\hbox{}

\Level 1 {C++ notes}

\listofcppnote
\vfill\hbox{}

\Level 1 {The MPL C++ interface}
\label{sec:idx:mpl}

\listofmplnote
\vfill\hbox{}

\Level 1 {Python notes}

\listofpythonnote

\Level 0 {Index of MPI commands and keywords}

\begin{multicols*}{2}
\printindex[mpi]
\end{multicols*}

\Level 1 {From the standard document}

This is an automatically generated list of every
function, type, and constant in the MPI standard document.
Where these appear in this book, a page reference is given.

\Level 2 {List of all functions}
\input{standard/standard-functions}
\Level 2 {List of all dtypes}
%%%% empty list?
%\input{standard/standard-dtypes}
\Level 2 {List of all ctypes}
\input{standard/standard-ctypes}
\Level 2 {List of all ftypes}
\input{standard/standard-ftypes}
\Level 2 {List of all constants}
\input{standard/standard-consts}
\Level 2 {List of all callbacks}
\input{standard/standard-callbacks}

\Level 1 {MPI for Python}

\Level 2 {Buffer specifications}

{\small
%\begin{multicols}{2}
  \verbatiminput{standardp/bufspec.tex}
  \verbatiminput{standardp/bufspecb.tex}
  \verbatiminput{standardp/bufspecv.tex}
  \verbatiminput{standardp/bufspecw.tex}
%\end{multicols}
}

\Level 2 {Listing of python routines}

\begin{multicols}{3}
\small
\pylist{Comm}
\pylist{Cartcomm}
\pylist{Distgraphcomm}
\pylist{Graphcomm}
\pylist{Intercomm}
\pylist{Intracomm}
\pylist{Topocomm}
\pylist{Group}

\pylist{Request}
\pylist{Grequest}
\pylist{Prequest}
\pylist{Status}

\pylist{Win}

\pylist{Datatype}
\pylist{File}
\pylist{Info}
\pylist{Op}

\pylist{Errhandler}
\pylist{Message}
%\pylist{_InPlace}
%\pylist{Exception}
%\pylist{_Pickle}
\end{multicols}

\Level 0 {Index of OpenMP keywords}

\begin{multicols*}{2}
\printindex[omp]
\end{multicols*}

\Level 0 {Index of PETSc keywords}

\begin{multicols*}{2}
\printindex[petsc]
\end{multicols*}

\Level 0 {Index of KOKKOS keywords}

\begin{multicols*}{2}
\printindex[kokkos]
\end{multicols*}

\Level 0 {Index of SYCL keywords}

\begin{multicols*}{2}
\printindex[sycl]
\end{multicols*}

\hbox{}\vfill
\includegraphics{isbnbarcode}

\closeout\chapterlist
\end{document}
