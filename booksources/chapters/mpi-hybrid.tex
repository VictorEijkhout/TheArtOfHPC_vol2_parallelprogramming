% -*- latex -*-
%%%%%%%%%%%%%%%%%%%%%%%%%%%%%%%%%%%%%%%%%%%%%%%%%%%%%%%%%%%%%%%%
%%%%%%%%%%%%%%%%%%%%%%%%%%%%%%%%%%%%%%%%%%%%%%%%%%%%%%%%%%%%%%%%
%%%%
%%%% This text file is part of the source of
%%%% `Parallel Programming in MPI and OpenMP'
%%%% by Victor Eijkhout, copyright 2012-2021
%%%%
%%%% hybrid.tex : about hybrid computing
%%%%
%%%%%%%%%%%%%%%%%%%%%%%%%%%%%%%%%%%%%%%%%%%%%%%%%%%%%%%%%%%%%%%%
%%%%%%%%%%%%%%%%%%%%%%%%%%%%%%%%%%%%%%%%%%%%%%%%%%%%%%%%%%%%%%%%

While the MPI standard itself makes no mention of threads
--~process being the primary unit of computation~--
the use of threads is allowed.
Below we will discuss what provisions exist for doing so.

Using threads and other shared memory models in combination with MPI
leads of course to the question how
\indexterm{race condition}s are handled.
Example of a code with a
\emph{data race}\index{race condition!in MPI/OpenMP}
that pertains to MPI:
\begin{lstlisting}
#pragma omp sections
#pragma omp section
  MPI_Send( x, /* to process 2 */ )
#pragma omp section
  MPI_Recv( x, /* from process 3 */ )
\end{lstlisting}
The MPI standard here puts the burden on the user:
this code is not legal, and behavior is not defined.

\Level 0 {MPI support for threading}
\label{sec:init-thread}
\label{sec:ref:mpi-thread}


In hybrid execution, the main question is whether all threads
are allowed to make MPI calls. To determine this,
replace the \indexmpishow{MPI_Init} call by
%
\indexmpiref{MPI_Init_thread}
%
Here the \n{required} and \n{provided} parameters can take the following
(monotonically increasing) values:
\begin{itemize}
\item \indexmpidef{MPI_THREAD_SINGLE}: Only a single thread will
  execute.
\item\indexmpidef{MPI_THREAD_FUNNELED}: The program may use multiple
  threads, but only the main thread will make MPI calls.

    The main thread is usually the one selected by the
    \indexomppragma{master} directive, but technically it is the only that
    executes \indexmpishow{MPI_Init_thread}. If you call this routine in
    a parallel region, the main thread may be different from the master.
\item\indexmpidef{MPI_THREAD_SERIALIZED}: The program may use multiple
  threads, all of which may make MPI calls, but there will never be
  simultaneous MPI calls in more than one thread.
\item\indexmpidef{MPI_THREAD_MULTIPLE}: Multiple threads may issue MPI
  calls, without restrictions.
\end{itemize}

After the initialization call, you can query the support level
with \indexmpiref{MPI_Query_thread}.

In case more than one thread performs communication, 
\indexmpiref{MPI_Is_thread_main}
can determine whether a thread is the main thread.

\begin{pythonnote}{Thread level}
  The thread level can be set through the \lstinline{mpi4py.rc} object
  (section~\ref{sec:mpi-init}):
\begin{lstlisting}
mpi4py.rc.threads # default True
mpi4py.rc.thread_level # default "multiple"
\end{lstlisting}
Available levels are
\n{multiple}, \n{serialized}, \n{funneled}, \n{single}.
\end{pythonnote}

\begin{mplnote}{Threading support}
  \ac{MPL} always calls \indexmpishow{MPI_Init_thread}
  requesting the highest level \indexmpishow{MPI_THREAD_MULTIPLE}.
\begin{lstlisting}
enum mpl::threading_modes {
  mpl::threading_modes::single = MPI_THREAD_SINGLE, 
  mpl::threading_modes::funneled = MPI_THREAD_FUNNELED,
  mpl::threading_modes::serialized = MPI_THREAD_SERIALIZED,
  mpl::threading_modes::multiple = MPI_THREAD_MULTIPLE
};
threading_modes mpl::environment::threading_mode ();
bool mpl::environment::is_thread_main ();
\end{lstlisting}
\end{mplnote}

\begin{tacc}
  The \indexterm{mvapich} implementation of MPI
  does have the required threading support, but you need to set this environment variable:  
\begin{verbatim}
export MV2_ENABLE_AFFINITY=0
\end{verbatim}
  Another solution is to run your code like this:
\begin{verbatim}
  ibrun tacc_affinity <my_multithreaded_mpi_executable
\end{verbatim}
  Intel MPI uses an environment variable to turn on thread support:
\begin{verbatim}
I_MPI_LIBRARY_KIND=<value>
where
release : multi-threaded with global lock
release_mt : multi-threaded with per-object lock for thread-split  
\end{verbatim}
\end{tacc}

The \emph{mpiexec}\index{mpiexec!and environment variables}
program usually propagates \indexterm{environment variables},
so the value of \indextermtt{OMP_NUM_THREADS} when you call \indextermtt{mpiexec}
will be seen by each MPI process.

\begin{itemize}
\item It is possible to use blocking sends in threads, and let the
  threads block. This does away with the need for polling.
\item You can not send to a thread number: use the MPI
  \indextermbus{message}{tag} to send to a specific thread.
\end{itemize}

\begin{exercise}
Consider the 2D heat equation and explore the mix of MPI/OpenMP
parallelism:
\begin{itemize}
\item Give each node one MPI process that is fully multi-threaded.
\item Give each core an MPI process and don't use multi-threading.
\end{itemize}
Discuss theoretically why the former can give higher performance.
Implement both schemes as special cases of the general hybrid case,
and run tests to find the optimal mix.
\end{exercise}

\cverbatimsnippet[examples/mpi/c/thread.c]{thread}

