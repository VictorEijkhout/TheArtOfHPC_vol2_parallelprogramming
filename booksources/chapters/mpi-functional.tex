% -*- latex -*-
%%%%%%%%%%%%%%%%%%%%%%%%%%%%%%%%%%%%%%%%%%%%%%%%%%%%%%%%%%%%%%%%
%%%%%%%%%%%%%%%%%%%%%%%%%%%%%%%%%%%%%%%%%%%%%%%%%%%%%%%%%%%%%%%%
%%%%
%%%% This text file is part of the source of 
%%%% `Parallel Programming in MPI and OpenMP'
%%%% by Victor Eijkhout, copyright 2012-2021
%%%%
%%%% mpi-functional.tex : about functional parallelism
%%%%
%%%%%%%%%%%%%%%%%%%%%%%%%%%%%%%%%%%%%%%%%%%%%%%%%%%%%%%%%%%%%%%%
%%%%%%%%%%%%%%%%%%%%%%%%%%%%%%%%%%%%%%%%%%%%%%%%%%%%%%%%%%%%%%%%

\Level 0 {The SPMD model}
\label{sec:spmd}

MPI programs conform largely
to the \acf{SPMD} model, where each processor runs the same executable.
This running executable we call a \indexterm{process}.

When MPI was first written, 20 years ago, it was clear what a processor
was: it was what was in a computer on someone's desk, or in a rack.
If this computer was part of a networked cluster, you called it a \indextermdef{node}.
So if you ran an MPI program, each node would have one MPI process;
%
\begin{figure}[ht]
  \includegraphics[scale=.11]{mpi-node1}
  \caption{Cluster structure as of the mid 1990s}
  \label{fig:oldmpi}
\end{figure}
%
figure~\ref{fig:oldmpi}.
%
You could of course run more than one
process, using the \indexterm{time slicing} of the \ac{OS}, but that
would give you no extra performance.

These days the situation is more complicated.
You can still talk about a node in a cluster, but now a node can contain
more than one processor chip (sometimes called a \indextermdef{socket}),
and each processor chip probably has multiple
\emph{cores}\index{core}.
%
\begin{figure}[ht]
  \includegraphics[scale=.11]{mpi-node3}
  \caption{Hybrid cluster structure}
  \label{fig:hybridmpi}
\end{figure}
%
Figure~\ref{fig:hybridmpi} shows how you could explore this using a mix
of MPI between the nodes, and a shared memory programming system on the nodes.

However, since each core can act like an independent processor,
you can also have multiple MPI processes per node. To MPI, the cores look
like the old completely separate processors. This is the `pure MPI'
model of figure~\ref{fig:purempi}, which we will use in most of this part
of the book. (Hybrid computing will be discussed in chapter~\ref{ch:hybrid}.)
%
\begin{figure}[ht]
  \includegraphics[scale=.11]{mpi-node2}
  \caption{MPI-only cluster structure}
  \label{fig:purempi}
\end{figure}
%

This is somewhat confusing: the old processors needed MPI programming, because
they were physically separated. The cores on a modern processor, on the other hand,
share the same memory, and even some caches. In its basic mode MPI
seems to ignore all
of this: each core receives an MPI process and the programmer writes the same send/receive call no matter
where the other process is located. In fact, you can't immediately see
whether two cores are on the same node or different nodes. Of course,
on the implementation level MPI uses a different communication
mechanism depending on whether  cores are on the same socket or on
different nodes, so you don't have to worry about lack of efficiency.

\begin{remark}
  In some rare cases you may want to run in an \indexac{MPMD} mode, rather
  than \ac{SPMD}. This can be achieved either on the \ac{OS} level
  (see section~\ref{sec:mpmd-start}),
  using options of the \indextermtt{mpiexec} mechanism, or you can use
  MPI's built-in process management; chapter~\ref{ch:mpi-proc}. Like
  I~said, this concerns only rare cases.
\end{remark}

%% \Level 0 {Starting and running MPI processes}
\input chapters/mpi-running

%% \Level 0 {Processor identification}
\input chapters/mpi-rank

\Level 0 {Functional parallelism}

Now that processes can distinguish themselves from each other,
they can decide to engage in different activities.
In an extreme case you could have a code that looks like
\begin{lstlisting}
// climate simulation:
if (procid==0)
  earth_model();
else if (procid=1)
  sea_model();
else
  air_model();
\end{lstlisting}
Practice is a little more complicated than this. But we will start
exploring this notion of processes deciding on their activity
based on their process number.

Being able to tell processes apart is already enough to write some
applications, without knowing any other MPI.
We will look at a simple parallel search algorithm:
based on its rank, a processor can find its section of
a search space.  For instance, in \indexterm{Monte Carlo codes} a
large number of random samples is generated and some computation
performed on each. (This particular example requires each MPI process
to run an independent random number generator, which is not entirely
trivial.)

\begin{exercise}
  \label{ex:primetest}
  Is the number $N=2,000,000,111$ prime?  Let each process test a
  disjoint set of integers, and print out any factor they find.  You don't
  have to test all integers~$<N$: any factor is at
  most~$\sqrt N\approx 45,200$.

  (Hint: \verb+i%0+ probably gives a runtime error.)

  Can you find more than one solution?
  \skeleton{prime}
\end{exercise}

\begin{remark}
  Normally, we expect parallel algorithms to be faster than sequential.
  Now consider the above exercise. Suppose the number we are testing
  is divisible by some small prime number, but every process has a
  large block of numbers to test. In that case the sequential algorithm would
  have been faster than the parallel one. Food for thought.
\end{remark}

As another example, in \indexterm{Boolean satisfiability} problems
a number of points in a search space needs to be evaluated. Knowing
a process's rank is enough to let it generate its own portion of the
search space. The computation of the \indexterm{Mandelbrot set} can
also be considered as a case of functional parallelism. However, the
image that is constructed is data that needs to be kept on one
processor, which breaks the symmetry of the parallel run.

Of course, at the end of a functionally parallel run you need to
summarize the results, for instance printing out some total.
The mechanisms for that you will learn next.

\Level 0 {Distributed computing and distributed data}
\label{sec:distributed-array}

One reason for using MPI is that sometimes you need to work on
a single object,
say a vector or a matrix,
with a  data size larger than can fit in the memory of a single processor.
With distributed memory, each processor then gets a part
of the whole data structure and only works on that.

So let's say we have a large array, and we want to
distribute the data over the processors.
That means that, with \n{p} processes and \n{n}~elements
per processor, we have a total of $\mathtt{n}\cdot\mathtt{p}$
elements.

\begin{figure}[ht]
  \includegraphics[scale=.1]{mpi-array}
  \caption{Local parts of a distributed array}
  \label{fig:mpi-array}
\end{figure}

We sometimes say that \n{data} is the local part
of a \indexterm{distributed array} with a total size of
$\mathtt{n}\cdot\mathtt{p}$
elements.
However, this array only exists
conceptually: each processor has an array with lowest index zero,
and you have to translate that yourself to an index in the global
array.
In other words, you have to write your code in such a way that
it acts like you're working with a large array that is distributed
over the processors, while
actually manipulating only the local arrays on the processors.

Your typical code then looks like

\begin{lstlisting}
int myfirst = .....;
for (int ilocal=0; ilocal<nlocal; ilocal++) {
   int iglobal = myfirst+ilocal;
   array[ilocal] = f(iglobal);
}
\end{lstlisting}

\begin{exercise}
  \label{ex:array-ints}
  Allocate on each process an array:
\begin{lstlisting}
int my_ints[10];    
\end{lstlisting}
and fill it so that process~0 has the integers $0\cdots 9$, process~1 has $10\cdots 19$,
et cetera.

It may be hard to print the output in a non-messy way.
\end{exercise}

If the array size is not perfectly divisible by the number of processors,
we have to come up with a division that is uneven, but not too much.
You could for instance, write

\begin{lstlisting}
int Nglobal, // is something large
    Nlocal = Nglobal/ntids,
    excess = Nglobal%ntids;
if (mytid==ntids-1) 
  Nlocal += excess;
\end{lstlisting}

\begin{exercise}
  Argue that this strategy is not optimal. Can you come up with a
  better distribution?
  Load balancing is further discussed in~\HPSCref{sec:load}.
\end{exercise}

\begin{comment}
  \begin{exercise}
    \label{ex:inproduct}
    Implement an inner product routine: let $x$ be a
    distributed vector of size~$N$ with elements $x[i]=i$,
    and compute~$x^tx$.
    As before, the right value is $(2N^3+3N^2+N)/6$.

    Use the inner product value to scale to vector so that it has
    norm~1.
    Check that your computation is correct.
  \end{exercise}
\end{comment}

\Level 0 {Review questions}

For all true/false questions, if you answer that a statement is false,
give a one-line explanation.

\begin{exercise}
  \label{ex:m1}
  True or false: \n{mpicc} is a compiler.
\end{exercise}

\begin{exercise}
  \label{ex:mpi01rank}
  T/F?
  \begin{enumerate}
  \item In C, the result of \lstinline{MPI_Comm_rank} is a number
    from zero to number-of-processes-minus-one, inclusive.
  \item In Fortran, the result of \lstinline{MPI_Comm_rank} is a number
    from one to number-of-processes, inclusive.
  \end{enumerate}
\end{exercise}

\begin{exercise}
  What is the function of a hostfile?
\end{exercise}

\begin{pcse}
\begin{exercise}
    An architecture is called `symmetric' or `uniform' if the
    relation between any pair of processes is essentially the same.
    In what way are MPI processes run on stampede symmetric; in what way
    not?
\end{exercise}
\end{pcse}

