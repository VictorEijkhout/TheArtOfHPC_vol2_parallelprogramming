% -*- latex -*-
%%%%%%%%%%%%%%%%%%%%%%%%%%%%%%%%%%%%%%%%%%%%%%%%%%%%%%%%%%%%%%%%
%%%%%%%%%%%%%%%%%%%%%%%%%%%%%%%%%%%%%%%%%%%%%%%%%%%%%%%%%%%%%%%%
%%%%
%%%% This text file is part of the source of
%%%% `Parallel Programming in MPI and OpenMP'
%%%% by Victor Eijkhout, copyright 2012-2025
%%%%
%%%% mpi-ptp.tex : about point-to-point communication
%%%%
%%%%%%%%%%%%%%%%%%%%%%%%%%%%%%%%%%%%%%%%%%%%%%%%%%%%%%%%%%%%%%%%
%%%%%%%%%%%%%%%%%%%%%%%%%%%%%%%%%%%%%%%%%%%%%%%%%%%%%%%%%%%%%%%%

%% \Level 0 {Blocking point-to-point operations}
\input chapters/mpi-blocksend

%% \Level 0 {Nonblocking point-to-point operations}
\input chapters/mpi-nonblock

\Level 0 {The Status object and wildcards}
\label{sec:mpi-wildcard}
\label{sec:mpi-status}
\index{message!status|(textbf}

In section~\ref{sec:mpi-send-recv}
you saw that \indexmpishow{MPI_Recv} has a `status' argument
of type \indexmpishow{MPI_Status} that \indexmpishow{MPI_Send} lacks.
(The various \indexmpishow{MPI_Wait...}  routines also have a status
argument; see section~\ref{sec:nonblocking}.)
Often you specify \indexmpishow{MPI_STATUS_IGNORE}
for this argument: commonly you know
what data is coming in and where it is coming from.

However, in some circumstances the recipient may not know all details of a
message when you make the receive call, so MPI has a way of querying
the \emph{status}\index{status!of received message} of the message:
\begin{itemize}
\item If you are expecting multiple incoming messages, it may be most
  efficient to deal with them in the order in which they arrive. So,
  instead of waiting for a specific message, you would specify
  \indexmpishow{MPI_ANY_SOURCE} or \indexmpishow{MPI_ANY_TAG} in
  the description of the receive message. 
  Now you have to be able to ask `who did this message come from,
  and what is in it'.
\item Maybe you know the sender of a message, but the amount of data
  is unknown. In that case you can overallocate your receive buffer,
  and after the message is received ask how big it was, or you can
  `probe' an incoming message
  (see section~\ref{sec:mpi-probe})
  and allocate enough data when you find
  out how much data is being sent.
\end{itemize}

To do this, the receive call has a \indexmpidef{MPI_Status} parameter.
The \indexmpishow{MPI_Status} object
is a structure (in~C a \lstinline{struct}, in~F90 an array, in~F2008 a derived type)
with freely accessible members:
\begin{itemize}
\item \indexmpishow{MPI_SOURCE} gives the source ofthe message;
  see section~\ref{sec:mpi-source}.
\item \indexmpishow{MPI_TAG} gives the tag with which the message was received;
  see section~\ref{sec:mpi-tag}.
\item \indexmpishow{MPI_ERROR} gives the error status of the receive call;
  see section~\ref{sec:mpi-status-error}.
\item The number of items in the message can be deduced from the status object,
  not as a structure member,
  but through a function call to \indexmpishow{MPI_Get_count};  
  see section~\ref{sec:mpi-status-count}.
\end{itemize}

\begin{fortrannote}{Status object in f08}
  \label{f:status-object}
  The \indextermtt{mpi_f08} module turns many handles
  (such as communicators)
  from Fortran \lstinline{Integer}s into \lstinline{Type}s.
  Retrieving the integer from the type is usually done
  through the \lstinline+%val+ member,
  but for the status object this is more difficult.
  The routines \indexmpidef{MPI_Status_f2f08} and \indexmpidef{MPI_Status_f082f}
  convert between these.
  (Remarkably, these routines are even available in~C,
  where they operate on \indexmpishow{MPI_Fint}, \indexmpishow{MPI_F08_status} arguments.)
\end{fortrannote}

\begin{mpifournote}{Status access}
  The \lstinline{SOURCE,TAG,ERROR} fields can also be accessed,
  both to get and to set,
  with routines 
  \begin{itemize}
    \item \indexmpishow{MPI_Status_get_source}, \indexmpishow{MPI_Status_set_source};
    \item \indexmpishow{MPI_Status_get_tag}, \indexmpishow{MPI_Status_set_tag};
    \item \indexmpishow{MPI_Status_get_error}, \indexmpishow{MPI_Status_set_error};
  \end{itemize}
\end{mpifournote}

\begin{pythonnote}{Status object}
  The status object is explicitly created before being passed
  to the receive routine. It has the usual query method
  for the message count:
  %
  \pverbatimsnippet{statuscountp}

  (The count function without argument returns a result in bytes.)

  However, unlike in C/F where the fields of the status object
  are directly accessible, Python has query methods for these too:
  \begin{lstlisting}
    status.Get_source()
    status.Get_tag()
    status.Get_elements()
    status.Get_error()
    status.Is_cancelled()
  \end{lstlisting}
  Should you need them, there are even \lstinline{Set}
  variants of these.
  \url{https://mpi4py.readthedocs.io/en/stable/reference/mpi4py.MPI.Status.html}
\end{pythonnote}

\begin{mplnote}{Status object}
  The \lstinline+mpl::status_t+ object is created by the receive
  (or wait) call:
  %
  \mplverbatimsnippet[examples/mpi/mpl/vector.cxx]{mplstatuscreate}
\end{mplnote}

\Level 1 {Source}
\label{sec:mpi-source}

In some applications it makes sense that a message can come from 
one of a number of processes. In this case, it is possible to specify
\indexmpishow{MPI_ANY_SOURCE} as the source.
%
To find out the \emph{source}\index{message!status!source}
where the message actually came from,
you would use the \indexmpidef{MPI_SOURCE} field of the status object
that is delivered by \indexmpishow{MPI_Recv}
or the \indexmpishow{MPI_Wait...} call after an \indexmpishow{MPI_Irecv}.
\begin{lstlisting}
MPI_Recv(recv_buffer+p,1,MPI_INT, MPI_ANY_SOURCE,0,comm,
         &status);
sender = status.MPI_SOURCE;
\end{lstlisting}

There are various scenarios where receiving from `any source' makes sense.
One is that of the \indexterm{manager-worker} model. The manager task would first send
data to the worker tasks, then issues a blocking wait for the data of whichever process
finishes first.

In \fstandard{2008} style, the source is a member of the \flstinline{Status} type.

\fverbatimsnippet[examples/mpi/f08/anysource.F90]{anysource-f08}

In \fstandard{90} style, the source is an index in the \flstinline{Status} array.

\fverbatimsnippet[examples/mpi/f/anysource.F90]{anysource-f}

\begin{mplnote}{Status querying}
The \indexmplshow{status} object can be queried:
\begin{lstlisting}
int source = recv_status.source();
\end{lstlisting}
Likewise the source:
\begin{lstlisting}
mpl::tag_t t = recv_status.tag();
\end{lstlisting}

\end{mplnote}

\Level 1 {Tag}
\label{sec:mpi-tag}

In some circumstances, a tag wildcard can be used on the receive operation:
\indexmpishow{MPI_ANY_TAG}.
The actual
\emph{tag}\index{message!status!tag}
of a message can be retrieved as the
%
\indexmpidef{MPI_TAG}
%
member in the status structure.

There are not many cases where this is needed.
\begin{itemize}
\item Messages from a single source, even non-blocking,
  are \emph{non-overtaking}\index{message!non-overtaking}.
  This means that messages can be distinguished by their order.
\item Messages from multiple sources can be distinguished by the source field.
\item Retrieving the message tag might be needed if information is encoded in it.
\item The non-overtaking argument does not apply in the case of hybrid
  computing: two threads may send messages that do not have an MPI-imposed order.
  See the example in section~\ref{sec:hybrid-order}.
\end{itemize}

\begin{mplnote}{Message tag}
  \ac{MPL} differs from other \acp{API} in its treatment of tags:
  a~tag is not directly an integer, but an object of class \indexmplshow{tag_t}.
  %
  \mplverbatimsnippet{mplsendrecv} % in an exercise
  %
  The \indexmplshow{tag_t} class has a couple of methods such as
  \lstinline+mpl::+\indexmplshow{tag_t::any}\lstinline+()+
  (for the \indexmpishow{MPI_ANY_TAG} wildcard in receive calls)
  and
  \lstinline+mpl::+\indexmplshow{tag_t::up}\lstinline+()+
  (maximal tag, found from the \indexmpishow{MPI_TAG_UB} attribute).
\end{mplnote}

\begin{mplnote}{Tag types}
  Tag are \lstinline{int} or an \lstinline{enum} typ:
  \begin{lstlisting}
    template<typename T >
 	tag_t (T t);
 	tag_t (int t);
  \end{lstlisting}
  Example:
  \mplverbatimsnippet{mplinttag}
\end{mplnote}

\Level 1 {Error}
\label{sec:mpi-status-error}

For functions that return a single status, any error is returned
as the function result.
For a function returning multiple statuses,
such as \indexmpishow{MPI_Waitall},
the presence of an error in one of the receives
is indicated by a result of \indexmpidef{MPI_ERR_IN_STATUS}.
Any \emph{errors}\index{message!status!error}
during the receive operation can be found as the
\indexmpidef{MPI_ERROR}
member of the status structure.


\Level 1 {Count}
\label{sec:mpi-status-count}

If the amount of data received is not known a~priori, the
\emph{count}\index{message!count} of elements received
can be found by 
\indexmpiref{MPI_Get_count}:
%
\cverbatimsnippet[examples/mpi/c/count.c]{mpigetcount}
%
\fsnippetwithoutput{mpigetcountf08}{examples/mpi/f08}{count}
% [/count.F90]

This may be necessary since the \n{count} argument to \indexmpishow{MPI_Recv} is 
the buffer size, not an indication of the actually received number of
data items.

Remarks.
\begin{itemize}
\item Unlike the source and tag, the message count is not directly a member of the status
  structure.
\item The `count' returned is the number of elements of the specified
  datatype. If this is a derived type
  (section~\ref{sec:derived-types}) this is not the same as the number
  of predefined datatype elements. For that, use
  \indexmpiref{MPI_Get_elements} or
  \indexmpishow{MPI_Get_elements_x}
  which returns the number of basic elements.
\end{itemize}

\begin{mplnote}{Receive count}
  The \lstinline+get_count+ function is a method of the status object.
  The argument type is handled through templating:
  %
  \mplverbatimsnippet[examples/mpi/mpl/recvstatus.cxx]{mplrecvcount}
\end{mplnote}

\Level 1 {Example: receiving from any source}

Consider an example where the last process receives from every other process.
We could implement this as a loop
\begin{lstlisting}
for (int p=0; p<nprocs-1; p++)
  MPI_Recv( /* from source= */ p );
\end{lstlisting}
but this may incur idle time if the messages arrive out of order.

Instead, we use the  \indexmpishow{MPI_ANY_SOURCE} specifier to give a wildcard
behavior to the receive call: using this value for the `source' value
means that we accept mesages from any source within the communicator,
and messages are only matched by tag value.
(Note that size and type of the receive buffer are not used for message matching!)

We then retrieve the
actual source from the \indexmpishow{MPI_Status} object through the
\indexmpishow{MPI_SOURCE} field.
%
\cverbatimsnippet[examples/mpi/c/anysource.c]{anysource}
%
\pverbatimsnippet[examples/mpi/p/anysource.py]{anysourcep}

The \indexterm{manager-worker} model is a
design patterns that offers an opportunity for inspecting the
\indexmpishow{MPI_SOURCE} field of the \indexmpishow{MPI_Status}
object describing the data that was received.
All workers processes model their work by waitin a random amount of time,
and the manager process accepts messages from any source.

\cverbatimsnippet[examples/mpi/c/anysource.c]{anysource}

In chapter~\ref{prj:mandelbrot} you can do programming project with
this model.

\index{message!status|)}

\Level 0 {More about point-to-point communication}

\Level 1 {Message probing}
\label{sec:mpi-probe}

MPI receive calls specify a receive buffer, and its size has to be
enough for any data sent. In case you really have no idea how much data
is being sent, and you don't want to overallocate the receive buffer,
you can use a `probe' call.

The routines \indexmpiref{MPI_Probe} and \indexmpiref{MPI_Iprobe}
(for which see also section~\ref{sec:progress})
accept a message but do not copy the data.
Instead, when probing tells you that there is a
message, you can use \indexmpishow{MPI_Get_count}
(section~\ref{sec:mpi-status-count})
to determine its size,
allocate a large enough receive buffer, and do a regular receive to
have the data copied.

\cverbatimsnippet[examples/mpi/c/probe.c]{probe}

There is a problem with the \indexmpishow{MPI_Probe} call in a
multithreaded environment: the following scenario can happen.
\begin{enumerate}
\item A thread determines by probing that a certain message has come
  in.
\item It issues a blocking receive call for that message\dots
\item But in between the probe and the receive call another thread
  has already received the message.
\item \dots~Leaving the first thread in a blocked state with no
  message to receive.
\end{enumerate}
This is solved by \indexmpiref{MPI_Mprobe}, which after a successful
probe removes the message from the \indexterm{matching queue}: the
list of messages that can be matched by a receive call. The thread
that matched the probe now issues an \indexmpiref{MPI_Mrecv} call on
that message through an object of type \indexmpidef{MPI_Message}.

\Level 1 {Errors}

MPI routines return \indexmpishow{MPI_SUCCESS} upon succesful completion.
The following error codes can be returned
(see section~\ref{sec:mpi-err-codes} for details)
for completion with error by both send and receive operations:
  \indexmpishow{MPI_ERR_COMM},
  \indexmpishow{MPI_ERR_COUNT},
  \indexmpishow{MPI_ERR_TYPE},
  \indexmpishow{MPI_ERR_TAG},
  \indexmpishow{MPI_ERR_RANK}.


\Level 1 {Message envelope}
\label{sec:mpi-envelope}
\index{message!envelope|(}
\index{envelope|see{message, envelope}}

Apart from its bare data, each message has a \emph{message envelope}.
This has enough information to distinguish messages from each other:
the source, destination, tag, communicator.

\index{message!envelope|)}

\Level 0 {Review questions}

For all true/false questions, if you answer that a statement is false,
give a one-line explanation.

\begin{review}
  Describe a deadlock scenario involving three processors.
\end{review}

\begin{review}
  True or false: a message sent with \lstinline{MPI_Isend} from one processor can be
  received with an \lstinline{MPI_Recv} call on another processor.
\end{review}

\begin{review}
  True or false: a message sent with \lstinline{MPI_Send} from one processor can be
  received with an \lstinline{MPI_Irecv} on another processor.
\end{review}

\begin{review}
  Why does the \lstinline{MPI_Irecv} call not have an \lstinline{MPI_Status} argument?
\end{review}

\begin{review}
  Suppose you are testing ping-pong timings.
  Why is it generally not a good idea to use processes 0 and~1 for the
  source and target processor?  Can you come up with a better guess?
\end{review}

%% one-sided
\begin{review}
  What is the relation between the concepts of `origin', `target', `fence',
  and `window' in one-sided communication.
\end{review}

\begin{review}
  What are the three routines for one-sided data transfer?
\end{review}

\lstset{
  language=C,
}

\begin{review}
  In the following fragments % in figures~\ref{fig:qblockc},\ref{fig:qblockf},
  assume that all buffers have been
  allocated with sufficient size. For each fragment note whether it
  deadlocks or not. Discuss performance issues.
\end{review}

    \lstset{language=C,basicstyle=\footnotesize\ttfamily}
    %% \lstinputlisting{qblock1}
\begin{lstlisting}
for (int p=0; p<nprocs; p++)
  if (p!=procid)
    MPI_Send(sbuffer,buflen,MPI_INT,p,0,comm);
for (int p=0; p<nprocs; p++)
  if (p!=procid)
    MPI_Recv(rbuffer,buflen,MPI_INT,p,0,comm,MPI_STATUS_IGNORE);
\end{lstlisting}
˜    %% \lstinputlisting{qblock2}
\begin{lstlisting}
for (int p=0; p<nprocs; p++)
  if (p!=procid)
    MPI_Recv(rbuffer,buflen,MPI_INT,p,0,comm,MPI_STATUS_IGNORE);
for (int p=0; p<nprocs; p++)
  if (p!=procid)
    MPI_Send(sbuffer,buflen,MPI_INT,p,0,comm);
\end{lstlisting}
    %% \lstinputlisting{qblock3}
\begin{lstlisting}
int ireq = 0;
for (int p=0; p<nprocs; p++)
  if (p!=procid)
    MPI_Isend(sbuffers[p],buflen,MPI_INT,p,0,comm,&(requests[ireq++]));
for (int p=0; p<nprocs; p++)
  if (p!=procid)
    MPI_Recv(rbuffer,buflen,MPI_INT,p,0,comm,MPI_STATUS_IGNORE);
MPI_Waitall(nprocs-1,requests,MPI_STATUSES_IGNORE);
\end{lstlisting}
    %% \lstinputlisting{qblock4}
\begin{lstlisting}
int ireq = 0;
for (int p=0; p<nprocs; p++)
  if (p!=procid)
    MPI_Irecv(rbuffers[p],buflen,MPI_INT,p,0,comm,&(requests[ireq++]));
for (int p=0; p<nprocs; p++)
  if (p!=procid)
    MPI_Send(sbuffer,buflen,MPI_INT,p,0,comm);
MPI_Waitall(nprocs-1,requests,MPI_STATUSES_IGNORE);
\end{lstlisting}
    %% \lstinputlisting{qblock5}
\begin{lstlisting}
int ireq = 0;
for (int p=0; p<nprocs; p++)
  if (p!=procid)
    MPI_Irecv(rbuffers[p],buflen,MPI_INT,p,0,comm,&(requests[ireq++]));
MPI_Waitall(nprocs-1,requests,MPI_STATUSES_IGNORE);
for (int p=0; p<nprocs; p++)
  if (p!=procid)
    MPI_Send(sbuffer,buflen,MPI_INT,p,0,comm);
\end{lstlisting}

    \pagebreak
    Fortran codes:
    
    \lstset{language=Fortran}

  \lstset{language=Fortran,basicstyle=\footnotesize\ttfamily}
  %% \lstinputlisting{qblock1f}
\begin{lstlisting}
do p=0,nprocs-1
   if (p/=procid) then
      call MPI_Send(sbuffer,buflen,MPI_INT,p,0,comm,ierr)
   end if
end do
do p=0,nprocs-1
   if (p/=procid) then
      call MPI_Recv(rbuffer,buflen,MPI_INT,p,0,comm,MPI_STATUS_IGNORE,ierr)
   end if
end do
\end{lstlisting}
  %% \lstinputlisting{qblock2f}
\begin{lstlisting}
do p=0,nprocs-1
   if (p/=procid) then
      call MPI_Recv(rbuffer,buflen,MPI_INT,p,0,comm,MPI_STATUS_IGNORE,ierr)
   end if
end do
do p=0,nprocs-1
   if (p/=procid) then
      call MPI_Send(sbuffer,buflen,MPI_INT,p,0,comm,ierr)
   end if
end do
\end{lstlisting}
  %% \lstinputlisting{qblock3f}
\begin{lstlisting}
ireq = 0
do p=0,nprocs-1
   if (p/=procid) then
      call MPI_Isend(sbuffers(1,p+1),buflen,MPI_INT,p,0,comm,&
           requests(ireq+1),ierr)
      ireq = ireq+1
   end if
end do
do p=0,nprocs-1
   if (p/=procid) then
      call MPI_Recv(rbuffer,buflen,MPI_INT,p,0,comm,MPI_STATUS_IGNORE,ierr)
   end if
end do
call MPI_Waitall(nprocs-1,requests,MPI_STATUSES_IGNORE,ierr)
\end{lstlisting}
  %% \lstinputlisting{qblock4f}
\begin{lstlisting}
ireq = 0
do p=0,nprocs-1
   if (p/=procid) then
      call MPI_Irecv(rbuffers(1,p+1),buflen,MPI_INT,p,0,comm,&
           requests(ireq+1),ierr)
      ireq = ireq+1
   end if
end do
do p=0,nprocs-1
   if (p/=procid) then
      call MPI_Send(sbuffer,buflen,MPI_INT,p,0,comm,ierr)
   end if
end do
call MPI_Waitall(nprocs-1,requests,MPI_STATUSES_IGNORE,ierr)
\end{lstlisting}
  %% \lstinputlisting{qblock5f}
\begin{lstlisting}
// block5.F90
ireq = 0
do p=0,nprocs-1
   if (p/=procid) then
      call MPI_Irecv(rbuffers(1,p+1),buflen,MPI_INT,p,0,comm,&
           requests(ireq+1),ierr)
      ireq = ireq+1
   end if
end do
call MPI_Waitall(nprocs-1,requests,MPI_STATUSES_IGNORE,ierr)
do p=0,nprocs-1
   if (p/=procid) then
      call MPI_Send(sbuffer,buflen,MPI_INT,p,0,comm,ierr)
   end if
end do
\end{lstlisting}

\begin{review}
  Consider a ring-wise communication where
  \begin{lstlisting}
    int
    next = (mytid+1) % ntids,
    prev = (mytid+ntids-1) % ntids;
  \end{lstlisting}
  and each process sends to \lstinline{next}, and receives from \lstinline{prev}.

  The normal solution for preventing deadlock is to use both
  \indexmpishow{MPI_Isend} and \indexmpishow{MPI_Irecv}. The send and
  receive complete at the wait call. But does it matter in what
  sequence you do the wait calls?
\end{review}

\begin{multicols}{2}
    \cverbatimsnippet{ring3}
    \columnbreak
    \cverbatimsnippet{ring4}
  \end{multicols}

  Can we have one nonblocking and one blocking call?
  Do these scenarios block?
  \begin{multicols}{2}
    \cverbatimsnippet{ring1}
    \columnbreak
    \cverbatimsnippet{ring2}
  \end{multicols}


\index{communication!two-sided|)}
