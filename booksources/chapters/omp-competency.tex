% -*- latex -*-
%%%%%%%%%%%%%%%%%%%%%%%%%%%%%%%%%%%%%%%%%%%%%%%%%%%%%%%%%%%%%%%%
%%%%%%%%%%%%%%%%%%%%%%%%%%%%%%%%%%%%%%%%%%%%%%%%%%%%%%%%%%%%%%%%
%%%%
%%%% This text file is part of the source of 
%%%% `Parallel Programming in MPI and OpenMP'
%%%% by Victor Eijkhout, copyright 2012-2021
%%%%
%%%% omp-competency.tex : goals of the OpenMP part
%%%%
%%%%%%%%%%%%%%%%%%%%%%%%%%%%%%%%%%%%%%%%%%%%%%%%%%%%%%%%%%%%%%%%
%%%%%%%%%%%%%%%%%%%%%%%%%%%%%%%%%%%%%%%%%%%%%%%%%%%%%%%%%%%%%%%%

This section of the book teaches OpenMP (`Open Multi Processing'),
the dominant model for shared memory programming in science and engineering.
It will instill the following competencies.

Basic level:
\begin{itemize}
\item Threading model: the student will understand the threading model of OpenMP,
  and the relation between threads and cores (chapter~\ref{ch:omp-basics});
  the concept of a parallel region and private versus shared data
  (chapter~~\ref{ch:omp-parallel}).
\item Loop parallelism: the student will be able to parallelize loops,
  and understand the impediments to parallelization, and
  iteration scheduling (chapter~\ref{ch:omp-loop};
  reductions (chapter~\ref{ch:omp-reduction}).
\item The student will understand the concept of worksharing constructs,
  and its implications for synchronization (chapter~\ref{ch:omp-share}).
\end{itemize}

Intermediate level:
\begin{itemize}
\item The student will understand the abstract notion of synchronization,
  its implementations in OpenMP, and implications for performabnce
  (chapter~\ref{ch:omp-sync}).
\item The student will understand the task model as underlying the thread model,
  be able to write code that spawns tasks, and be able to distinguish
  when tasks are needed versus simpler worksharing constructs
  (chapter~\ref{ch:omp-task}).
\item The student will understand thread/code affinity, how to control it,
  and possible implications for performance (chapter~\ref{ch:omp-affinity}).
\end{itemize}

Advanced level:
\begin{itemize}
\item The student will understand the OpenMP memory model,
  and sequential consistency (chapter~\ref{ch:omp-memory}).
\item The student will understand \acs{SIMD} processing, the extent to which
  compilers do this outside of OpenMP,
  and how OpenMP can specify further opportunities for SIMD-ization
  (chapter~\ref{ch:omp-simd}).
\item The student will understand offloading to \acp{GPU},
  and the OpenMP directives for effecting this (chapter~\ref{ch:omp-gpu}).
\end{itemize}
