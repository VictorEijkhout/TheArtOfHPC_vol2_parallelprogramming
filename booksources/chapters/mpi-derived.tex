% -*- latex -*-
%%%%%%%%%%%%%%%%%%%%%%%%%%%%%%%%%%%%%%%%%%%%%%%%%%%%%%%%%%%%%%%%
%%%%%%%%%%%%%%%%%%%%%%%%%%%%%%%%%%%%%%%%%%%%%%%%%%%%%%%%%%%%%%%%
%%%%
%%%% This text file is part of the source of
%%%% `Parallel Programming in MPI and OpenMP'
%%%% by Victor Eijkhout, copyright 2012-2024
%%%%
%%%% mpi-derived.tex : derived datatypes
%%%%
%%%%%%%%%%%%%%%%%%%%%%%%%%%%%%%%%%%%%%%%%%%%%%%%%%%%%%%%%%%%%%%%
%%%%%%%%%%%%%%%%%%%%%%%%%%%%%%%%%%%%%%%%%%%%%%%%%%%%%%%%%%%%%%%%

\Level 0 {Derived datatypes}
\label{sec:derived-types}
\index{datatype!derived|(}

MPI allows you to create your own data types, somewhat (but not completely\ldots)
analogous to defining
structures in a programming language. MPI data types are mostly of use
if you want to send multiple items in one message.

There are two problems with using only predefined datatypes
as you have seen so far.
\begin{itemize}
\item MPI communication routines can only send multiples of a
  single data type: it is not possible to send items of different
  types, even if they are contiguous in memory. It would be possible
  to use the \indexmpishow{MPI_BYTE} data type, but this is not advisable.
\item It is also ordinarily not possible to send items of one type if they are
  not contiguous in memory. You could of course send a contiguous memory area
  that contains the items you want to send, but that is wasteful of
  bandwidth, and of memory space on the receiving side.
\end{itemize}
With MPI data types you can solve these problems in several ways.
\begin{itemize}
\item You can create a new \indextermbus{contiguous}{data type}
  consisting of an array of elements of another data type. There is no
  essential difference between sending one element of such a type
  and multiple elements of the
  component type.
\item You can create a \indextermbus{vector}{data type} consisting of
  regularly spaced blocks of elements of a component type. This is a first
  solution to the problem of sending noncontiguous data.
\item For not regularly spaced data, there is the
  \indextermbus{indexed}{data type}, where you specify an array of
  index locations for blocks of elements of a component type.
  The blocks can each be of a different size.
\item The \indextermbus{struct}{data type} can accomodate multiple
  data types.
\end{itemize}
And you can combine these mechanisms to get irregularly spaced
heterogeneous data, et cetera.

\Level 1 {Basic calls}
\label{sec:data-commit}

The typical sequence of calls for creating a new datatype is as follows:
\begin{itemize}
\item You need a variable for the datatype;
  this is of type \indexmpishow{MPI_Datatype}.
\item There is a create call, followed by a `commit' call where MPI
  performs internal bookkeeping and optimizations;
  we will discuss this in great detail below.
\item The type needs to be `committed'. After this:
\item The datatype is used, possibly multiple times;
\item When the datatype is no longer needed, it must be freed to prevent memory leaks;
  see section~\ref{sec:mpi-type-commit-free}.
\end{itemize}
In code:
\begin{lstlisting}
MPI_Datatype newtype;
MPI_Type_something( < oldtype specifications >, &newtype );
MPI_Type_commit( &newtype );
/* code that uses your new type */
MPI_Type_free( &newtype );
\end{lstlisting}

In \fstandard{2008}:
\lstset{language=Fortran}
\begin{lstlisting}
Type(MPI_Datatype) :: newvectortype
call MPI_Type_something( <oldtype specification>, &
          newvectortype)
call MPI_Type_commit(newvectortype)
!! code that uses your type
call MPI_Type_free(newvectortype)
\end{lstlisting}
\lstset{language=C}

\begin{pythonnote}{Derived type handling}
  The various type creation routines are methods
  of the datatype classes, after which commit and free are
  methods on the new type.
  %
  \pverbatimsnippet[examples/mpi/p/vector.py]{vectorp}
\end{pythonnote}

\begin{mplnote}{Derived type handling}
  \label{mpl:derived}
  In \ac{MPL} type creation routines are in the main namespace,
  templated over the datatypes.
  %
  \mplverbatimsnippet[examples/mpi/mpl/vector.cxx]{vectormpl}
  %
  The commit call is part of the type creation,
  and freeing is done in the destructor.
\end{mplnote}

\begin{mplnote}{Layouts}
  \begin{lstlisting}
namespace mpl {
  template <typename T> class layout; // Basisklasse
  template <typename T> class null_layout; // MPI_DATATYPE_NULL
  template <typename T> class empty_layout; // leere Nachricht
  template <typename T> class contiguous_layout; // MPI_Type_contiguous
  template <typename T> class vector_layout; // MPI_Type_contiguous
  template <typename T> class strided_vector_layout; // MPI_Type_vector
  template <typename T> class indexed_layout; // MPI_Type_indexed
  template <typename T> class hindexed_layout; // MPI_Type_create_hindexed
  template <typename T> class indexed_block_layout; // MPI_Type_create_indexed_block
  template <typename T> class hindexed_block_layout;  // MPI_Type_create_hindexed_block 
  template <typename T> class iterator_layout; // MPI_Type_create_hindexed_block 
  template <typename T> subarray_layout; // MPI_Type_create_subarray
  class heterogeneous_layout; // MPI_Type_create_struct
}
  \end{lstlisting}
\end{mplnote}

\Level 2 {Create calls}

The \indexmpishow{MPI_Datatype} variable gets its value by a call to
one of the following routines:
\begin{itemize}
\item \indexmpishow{MPI_Type_contiguous} for contiguous blocks of
  data; section~\ref{sec:data:contiguous};
\item \indexmpishow{MPI_Type_vector} for regularly strided data;
  section~\ref{sec:data:vector};
\item \indexmpishow{MPI_Type_create_subarray} for subsets out higher
  dimensional block; section~\ref{sec:type_subarray};
\item \indexmpishow{MPI_Type_create_struct} for heterogeneous irregular data;
  section~\ref{sec:data:struct};
\item \indexmpishow{MPI_Type_indexed} and
  \indexmpishow{MPI_Type_hindexed} for irregularly strided data;
  section~\ref{sec:data:indexed}.
\end{itemize}
These calls take an existing type, whether predefined or also derived,
and produce a new type.

\Level 2 {Commit and free}
\label{sec:mpi-type-commit-free}

It is necessary to call \indexmpiref{MPI_Type_commit} on a new data
type, which makes MPI do the indexing calculations for the data type.

When you no longer
need the data type, you call \indexmpiref{MPI_Type_free}.
(This is typically not needed in \ac{OO} \acp{API}.)
This has the following effects:
\begin{itemize}
\item The definition of the datatype identifier will be changed to
  \indexmpidef{MPI_DATATYPE_NULL}.
\item Any communication using this data type, that was already
  started, will be completed succesfully.
\item Datatypes that are defined in terms of this data type will still
  be usable.
\end{itemize}

\Level 1 {Contiguous type}
\label{sec:data:contiguous}

The simplest derived type is the `contiguous' type,
constructed with \indexmpiref{MPI_Type_contiguous}.

A~contigous type describes an array of items
of an predefined or earlier defined type. There is no difference between sending
one item of a contiguous type and multiple items of the constituent type.
\begin{figure}[t]
  \includegraphics[scale=.08]{data-contiguous}
  \caption{A contiguous datatype is built up out of elements of a constituent type}
  \label{fig:data-contiguous}
\end{figure}
This is illustrated in figure~\ref{fig:data-contiguous}.

\cverbatimsnippet[examples/mpi/c/contiguous.c]{contiguous}

\fverbatimsnippet[examples/mpi/f08/contiguous.F90]{contiguous-f08}

\pverbatimsnippet[examples/mpi/p/contiguous.py]{contiguousp}

\begin{mplnote}{Contiguous type}
  The \ac{MPL} interface makes extensive use of
  \indexmpldef{contiguous_layout},
  as it is the main way to declare a nonscalar buffer.
\end{mplnote}

\begin{mplnote}{Contiguous composing}
  \label{mpl:cont_vector}
  Contiguous layouts can only use predefined types or other contiguous layouts
  as their `old' type.
  To make a contiguous type for other layouts, use \indexmpidef{vector_layout}:
  %
  \mplverbatimsnippet{mplcontvector}
  %
  (Contrast this with \indexmplshow{strided_vector_layout}; note~\ref{mpl:strided_vector}.)
\end{mplnote}

\Level 1 {Vector type}
\label{sec:data:vector}

The simplest noncontiguous datatype is the `vector' type, constructed with
\indexmpiref{MPI_Type_vector}.

A~vector type describes a series of blocks, all 
of equal size, spaced with a constant stride.
\begin{figure}[t]
  \includegraphics[scale=.12]{data-vector}
  \caption{A vector datatype is built up out of strided blocks of elements of a constituent type}
  \label{fig:data-vector}
\end{figure}
This is illustrated in figure~\ref{fig:data-vector}.

The vector datatype gives the first nontrivial illustration that
datatypes can be \emph{different on the sender and
  receiver}\index{datatype!different on sender and receiver}. If the
sender sends \n{b}~blocks of length~\n{l} each, the receiver can
%
\begin{figure}[t]
  \includegraphics[scale=.12]{data-vector-to-contiguous}
  \caption{Sending a vector datatype and receiving it as predefined or
    contiguous}
  \label{fig:data-vector-to-contiguous}
\end{figure}
%
receive them as \n{bl} contiguous elements, either as a contiguous
datatype, or as a contiguous buffer of an predefined type; see
figure~\ref{fig:data-vector-to-contiguous}.
The receiver has no knowledge of the stride of the datatype on the sender.

In this example a vector type is created only on the sender, in order to send
a strided subset of an array; the receiver receives the data as a contiguous block.
\cverbatimsnippet[examples/mpi/c/vector.c]{vector}

We illustrate \fstandard{2008}:
%
\fverbatimsnippet[examples/mpi/f08/vector.F90]{vector-f08}

In legacy mode \fstandard{90}, code stays the same except that the type is
declared as \lstinline{Integer}:
%
\fverbatimsnippet[examples/mpi/f/vector.F90]{vector-f}

\begin{pythonnote}{Vector type}
  The vector creation routine is a method of the datatype class.
  For the general discussion, see section~\ref{sec:data-commit}.
\end{pythonnote}

\pverbatimsnippet[examples/mpi/p/vector.py]{vectorp}

\begin{mplnote}{Vector type}
  \label{mpl:strided_vector}
  MPL has the \indexmpldef{strided_vector_layout} class as equivalent
  of the vector type:
  %
  \mplverbatimsnippet[examples/mpi/mpl/vector.cxx]{vectormpl}

  (See note~\ref{mpl:cont_vector} for nonstrided vectors.)
\end{mplnote}

\Level 2 {Two-dimensional arrays}

Figure~\ref{fig:blasmatrix} indicates one source of irregular
data:
%
\begin{figure}[t]
  \includegraphics[scale=.1]{blasmatrix}
  \caption{Memory layout of a row and column of a matrix in column-major storage}
  \label{fig:blasmatrix}
\end{figure}
%
with a matrix on \indexterm{column-major storage}, a column is
stored in contiguous memory. However, a row of such a matrix
is not contiguous; its elements being separated by a \indexterm{stride}
equal to the column length.

\begin{exercise}
  \label{ex:submatrix}
  How would you describe the memory layout of a submatrix,
  if the whole matrix has size $M\times N$ and the submatrix $m\times n$?
\end{exercise}

As an example of this datatype, consider the example of transposing
a matrix, for instance to convert between
C and Fortran arrays.
Suppose that 
a processor has a matrix stored in~C, row-major, layout, and it needs
to send a column to another processor. If the matrix is declared as
\begin{lstlisting}
  int M,N; double mat[M][N]
\end{lstlisting}
then a column has $M$ blocks of one element, spaced $N$~locations apart.
In other words:
\begin{lstlisting}
MPI_Datatype MPI_column;
MPI_Type_vector( 
    /* count= */ M, /* blocklength= */ 1, /* stride= */ N,
    MPI_DOUBLE, &MPI_column );
\end{lstlisting}
Sending the first column is easy:
\begin{lstlisting}
MPI_Send( mat, 1,MPI_column, ... );
\end{lstlisting}
The second column is just a little trickier: you now need to pick out 
elements with the same stride, but starting at \n{A[0][1]}.
\begin{lstlisting}
MPI_Send( &(mat[0][1]), 1,MPI_column, ... );
\end{lstlisting}
You can make this marginally more efficient (and harder to read)
by replacing the index expression by \n{mat+1}.

\begin{exercise}
  Suppose you have a matrix of size $4N\times 4N$, and you want to
  send the elements \n{A[4*i][4*j]} with $i,j=0,\ldots,N-1$. How would
  you send these elements with a single transfer?
\end{exercise}

\begin{exercise}
  \label{ex:col-to-row}
  Allocate a matrix on processor zero, using Fortran column-major storage.
  Using $P$ sendrecv calls, distribute the rows of this matrix among the
  processors.
\end{exercise}

\begin{pythonnote}{Sending from the middle of a matrix}
  In C and Fortran it's easy to apply a derived type
  to data in the middle of an array,
  for instance to extract an arbitrary column out of a C~matrix,
  or row out of a Fortran matrix.
  While Python has no trouble describing sections from an array,
  usually it copies these instead of taking the address.
  Therefore, it is necessary to convert the matrix
  to a buffer and compute an explicit offset in bytes:
  \pverbatimsnippet{pmatcolsend}
\end{pythonnote}

\begin{figure}[t]
  \includegraphics[scale=.5]{stridesend}
  \caption{Send strided data from process zero to all others}
  \label{fig:stridesend}
\end{figure}

\begin{exercise}
  \label{ex:stridesend}
  Let processor~0 have an array~$x$ of length $10P$, where $P$~is the number of processors.
  Elements $0,P,2P,\ldots,9P$ should go to processor zero, $1,P+1,2P+1,\ldots$ to processor~1,
  et cetera.
  \begin{itemize}
  \item
    Code this as a sequence of send/recv calls, using a vector datatype
    for the send, and a contiguous buffer for the receive.
  \item 
    For simplicity, skip the send to/from zero. What is the most elegant
    solution if you want to include that case?
  \item
    For testing, define the array as $x[i]=i$.
  \end{itemize}

  \skeleton{stridesend}
\end{exercise}

\begin{exercise}
  Write code to compare the time it takes to send a strided subset
  from an array: copy the elements by hand to a smaller buffer, or use
  a vector data type. What do you find? You may need to test on fairly
  large arrays.
\end{exercise}

\Level 1 {Subarray type}
\label{sec:type_subarray}

The vector datatype can be used for blocks in an array of dimension
more than~2 by using it recursively. However, this gets
tedious. Instead, there is an explicit subarray type
%
\indexmpiref{MPI_Type_create_subarray}.
%
This describes the dimensionality and extent of the array, and
the starting point (the `upper left corner') and extent of the
subarray.

\begin{mplnote}{Subarray layout}
  The templated \indexmpldef{subarray_layout} class is constructed
  from a vector of triplets of global size~/ subblock size~/ first coordinate.
\begin{lstlisting}
mpl::subarray_layout<int>( 
 { {ny, ny_l, ny_0}, {nx, nx_l, nx_0} }
);  
\end{lstlisting}
\end{mplnote}

\begin{exercise}
  \label{ex:cubegather}
  Assume that your number of processors is $P=Q^3$, and that each
  process has an array of identical size. Use
  \indexmpishow{MPI_Type_create_subarray} to gather all data onto a root process.
  Use a sequence of send and receive calls;
  \indexmpishow{MPI_Gather} does not work here.

  \skeleton{cubegather}

\begin{tacc}
If you haven't started \n{idev} with the right number of processes,
use
\begin{verbatim}
ibrun -np 27 cubegather
\end{verbatim}
Normally you use \n{ibrun} without process count argument.
\end{tacc}

\end{exercise}

\begin{fortrannote}{Subarrays}
  Subarrays are naturally supported in Fortran through array sections.
  %
  \fverbatimsnippet[examples/mpi/f08/section.F90]{fsectionsend}
  %
  However, there is a subtlety with non-blocking operations:
  for a non-contiguous buffer a temporary is created, which is released
  after the MPI call. This is correct for blocking sends, but for non-blocking
  the temporary has to stay around till the wait call.
  %
  \fverbatimsnippet[examples/mpi/f08/sectionisend.F90]{fsectionisend}
  %
  In \mpistandard{3} the variable \indexmpidef{MPI_SUBARRAYS_SUPPORTED}
  indicates support for this mechanism:
  %
  \fverbatimsnippet[examples/mpi/f08/sectionisend.F90]{fsectiontest}
\end{fortrannote}

The possibilities for the \n{order} parameter are
\indexmpidef{MPI_ORDER_C} and \indexmpidef{MPI_ORDER_FORTRAN}.
However, this has nothing to do with the order of traversal of elements;
it determines how the bounds of the subarray are interpreted.
As an example, we fill a $4\times 4$ array in C~order with the numbers~$0\cdots15$,
and send the $[0,1]\times [0\cdots4]$ slice two ways, 
first C~order, then Fortran order:
%
\cverbatimsnippet[code/mpi/c/row2col.c]{row2colsub}

The receiver receives the following, formatted to bring out
where the numbers originate:
\begin{verbatim}
Received C order:
 0.000 1.000 2.000 3.000
 4.000 5.000 6.000 7.000
Received F order:
 0.000 1.000
 4.000 5.000
 8.000 9.000
 12.000 13.000
\end{verbatim}

\Level 1 {Distributed array type}

Each dimension can independently be distributed as
\indexmpidef{MPI_DISTRIBUTE_BLOCK},
\indexmpidef{MPI_DISTRIBUTE_CYCLIC},
\indexmpidef{MPI_DISTRIBUTE_NONE},

With the cyclic distribution, the amount of cyclicity
can be indicated by setting \lstinline{dargs[id]} to a certain number.

With the block distribution,
blocks can be set explicitly in \lstinline{dargs[id]},
but \indexmpidef{MPI_DISTRIBUTE_DFLT_DARG} causes an even distribution
to be found.

Ordering can be \indexmpishow{MPI_ORDER_C} or \indexmpishow{MPI_ORDER_FORTRAN}.

\Level 1 {Indexed type}
\label{sec:data:indexed}

The indexed datatype, constructed with \indexmpiref{MPI_Type_indexed}
can send arbitrarily located elements from an array of a single datatype.
You need to supply an array of index locations, plus an array of blocklengths
with a separate blocklength for each index. The total number of elements sent
is the sum of the blocklengths.

\begin{figure}[t]
  \includegraphics[scale=.11]{data-indexed}
  \caption{The elements of an MPI Indexed datatype}
  \label{fig:data-indexed}
\end{figure}

The following example picks items that are on prime number-indexed
locations.
\cverbatimsnippet[examples/mpi/c/indexed.c]{indexed}
%
For Fortran we show the legacy syntax for once:
%
\fverbatimsnippet[examples/mpi/f/indexed.F90]{indexed-f}
%
\pverbatimsnippet[examples/mpi/p/indexed.py]{indexp}

\begin{mplnote}{Indexed type}
  In \ac{MPL}, the \indexmpldef{indexed_layout} is based on a vector
  of 2-tuples denoting block length~/ block location.
  %
  \mplverbatimsnippet[examples/mpi/mpl/indexed.cxx]{indexedmpl}
\end{mplnote}

\begin{mplnote}{Layouts for gatherv}
  The size/displacement arrays for \indexmpishow{MPI_Gatherv}~/
  \indexmpishow{MPI_Alltoallv} are handled through a
  \indexmpldef{layouts} object, which is basically a vector of
  \indexmplshow{layout} objects.
  %
  \mplverbatimsnippet{mpllayouts}
\end{mplnote}

\begin{mplnote}{Indexed block type}
  For the case where all block lengths are the same,
  use \indexmpldef{indexed_block_layout}:
  %
  \mplverbatimsnippet[examples/mpi/mpl/indexedblock.cxx]{indexedbmpl}
\end{mplnote}

You can also \indexmpishow{MPI_Type_create_hindexed} which describes blocks
of a single old type, but with index locations in bytes, rather than
in multiples of the old type.
\begin{lstlisting}
int MPI_Type_create_hindexed
 (int count, int blocklens[], MPI_Aint indices[],
  MPI_Datatype old_type,MPI_Datatype *newtype)
\end{lstlisting}
A slightly simpler version,
\indexmpiref{MPI_Type_create_hindexed_block}
assumes constant block length.

There is an important difference between the \n{hindexed} and the
above \indexmpishow{MPI_Type_indexed}: that one described offsets from
a base location; these routines describes absolute memory addresses.
You can use this to send for instance the elements of a linked list.
You would traverse the list, recording the addresses of the
elements with \indexmpiref{MPI_Get_address}.
(The routine \indexmpidepr{MPI_Address} is deprecated.)

In C++ you can use this to
send an \lstinline{std::<vector>}\index{C++!standard library!vector}, that
is, a vector object from the \indextermbus{C++}{standard library}, if
the component type is a pointer.

\Level 1 {Struct type}
\label{sec:data:struct}

The structure type, created with \indexmpiref{MPI_Type_create_struct},
can contain multiple data types. (The routine
\indexmpishow{MPI_Type_struct} is deprecated with \mpistandard{3}.)
%
\begin{figure}[t]
  \includegraphics[scale=.09]{data-struct}
  \caption{The elements of an MPI Struct datatype}
  \label{fig:data-struct}
\end{figure}
%
The specification contains a `count' parameter that specifies how many blocks
there are in a single structure. For instance,
\begin{lstlisting}
struct {
 int i;
 float x,y;
} point;
\end{lstlisting}
has two blocks, one of a single integer, and one of two floats.
This is illustrated in figure~\ref{fig:data-struct}.

\begin{description}
\item[\texttt{count}] The number of blocks in this
  datatype. The \n{blocklengths}, \n{displacements}, \n{types}
  arguments have to be at least of this length.
\item[\texttt{blocklengths}] array containing the lengths of the blocks of each datatype.
\item[\texttt{displacements}] array describing the relative location
  of the blocks of each datatype.
\item[\texttt{types}] array containing the datatypes; each block in
  the new type is of a single datatype; there can be multiple
  blocks consisting of the same type.
\end{description}
In this example, unlike the previous ones, both sender and receiver
create the structure type. With structures it is no longer possible to
send as a derived type and receive as a array of a simple type.
(It would be possible to send as one structure type and receive as another, 
as long as they have the same \indextermbus{datatype}{signature}.)

\cverbatimsnippet[examples/mpi/c/struct.c]{structure}

Note the \n{displacement} calculations in this example,
which involve some not so elegant pointer arithmetic.
The following Fortran code uses \indexmpishow{MPI_Get_address},
which is more elegant, and in fact
the only way address calculations can be done in Fortran.

\fverbatimsnippet[examples/mpi/f08/struct.F90]{fstructure}

It would have been incorrect to write
\begin{lstlisting}
displacement[0] = 0;
displacement[1] = displacement[0] + sizeof(char);
\end{lstlisting}
since you do not know the way the \indexterm{compiler} lays out the
structure in memory\footnote{Homework question: what does the language
  standard say about this?}.

If you want to send more than one structure, you have to worry more
about padding in the structure. You can solve this by adding an extra
type \indexmpishow{MPI_UB} for the `upper bound' on the structure:
\begin{lstlisting}
displacements[3] = sizeof(myobject); types[3] = MPI_UB;
MPI_Type_create_struct(struclen+1,.....);
\end{lstlisting}

\begin{mplnote}{Struct type scalar}
  One could describe the MPI struct type as a collection of
  displacements, to be applied to any set of items that conforms
  to the specifications.
  An \ac{MPL} \indexmplshow{heterogeneous_layout} on the other hand,
  incorporates the actual data. Thus you could write
  %
  \mplverbatimsnippet[examples/mpi/mpl/struct.cxx]{structuremplscalar}
  %
  Here, the \indexmplshow{absolute} indicates the lack of an implicit buffer:
  the layout is absolute rather than a relative description.
\end{mplnote}

\begin{mplnote}{Struct type general}
  More complicated data than scalars takes more work:
  %
  \mplverbatimsnippet[examples/mpi/mpl/struct.cxx]{structurempl}
  %
  Note the \indexmplshow{make_absolute} in addition to
  \indexmplshow{absolute} mentioned above.
\end{mplnote}
