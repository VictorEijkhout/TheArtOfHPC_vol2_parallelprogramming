% -*- latex -*-
%%%%%%%%%%%%%%%%%%%%%%%%%%%%%%%%%%%%%%%%%%%%%%%%%%%%%%%%%%%%%%%%
%%%%%%%%%%%%%%%%%%%%%%%%%%%%%%%%%%%%%%%%%%%%%%%%%%%%%%%%%%%%%%%%
%%%%
%%%% This text file is part of the source of 
%%%% `Parallel Computing'
%%%% by Victor Eijkhout, copyright 2012-2022
%%%%
%%%% introduction.tex : general blah
%%%%
%%%%%%%%%%%%%%%%%%%%%%%%%%%%%%%%%%%%%%%%%%%%%%%%%%%%%%%%%%%%%%%%
%%%%%%%%%%%%%%%%%%%%%%%%%%%%%%%%%%%%%%%%%%%%%%%%%%%%%%%%%%%%%%%%

The term `parallel computing' means different things
depending on the application area. In this book we focus
on parallel computing
--~and more specifically parallel \emph{programming};
we will not discuss a lot of theory~--
in the context of scientific computing.

Two of the most common software systems for parallel programming
in scientific computing are MPI and OpenMP.
They target different types of parallelism,
and use very different constructs. 
Thus, by covering both of them in one book
we can offer a treatment of parallelism that
spans a large range of possible applications.

Finally, we also discuss the PETSc
(Portable Toolkit for Scientific Computing)
library, which offers an abstraction level higher than
MPI or OpenMP, geared specifically towards
parallel linear algebra,
and very specifically the sort of linear algebra
computations arising from Partial Differential Equation modeling.

The main languages in scientific computing are C/C++ and Fortran.
We will discuss both MPI and OpenMP with many examples in these two languages.
For MPI and the PETSc library we will also discuss the Python interfaces.

\begin{quotation}
  \textbf{Comments} This book is in perpetual state of revision and refinement.
  Please send comments of any kind to \texttt{eijkhout@tacc.utexas.edu}.
\end{quotation}
