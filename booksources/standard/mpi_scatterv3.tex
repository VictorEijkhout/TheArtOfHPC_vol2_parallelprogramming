% -*- latex -*-
%%%%%%%%%%%%%%%%%%%%%%%%%%%%%%%%
%%%%%%%%%%%%%%%%%%%%%%%%%%%%%%%%
%%
%% This text file is part of the source of 
%% `Parallel Computing'
%% by Victor Eijkhout, copyright 2012-2022
%%
%% MPI API file for MPI_Scatterv
%%
%% THIS FILE IS AUTO-GENERATED
%%
%%%%%%%%%%%%%%%%%%%%%%%%%%%%%%%%
%%%%%%%%%%%%%%%%%%%%%%%%%%%%%%%%

\begingroup
\ttfamily\bfseries
\catcode`\_=12
\begin{tabular}{
l       % name
l       % param name
p{\mpiparmtextsize}  % explanation
p{.7in} % ctype
p{.9in} % ftype
l       % inout
}
\toprule
\mdseries\textrm{Name}&
\mdseries\textrm{Param name}&
\mdseries\textrm{Explanation}&
\mdseries\textrm{C type}&
\mdseries\textrm{F type}&
\mdseries\textrm{inout}\\
\midrule
\hbox to 18pt{MPI_Scatterv (\hss} \\
 & sendbuf & address of send buffer & const void* & TYPE(*), \hbox{}\kern10pt{}DIMENSION(..)  & IN \\ & sendcounts & non-negative integer array (of length group size) specifying the number of elements to send to each rank & const int[] & INTEGER\discretionary{}{\kern10pt}{}(*)  & IN \\ & displs & integer array (of length group size). Entry \mpicode{i} specifies the displacement (relative to \mpiarg{sendbuf}) from which to take the outgoing data to process \mpicode{i} & const int[] & INTEGER\discretionary{}{\kern10pt}{}(*)  & IN \\ & sendtype & datatype of send buffer elements & MPI_Datatype & TYPE\discretionary{}{\kern10pt}{}(MPI_Datatype)  & IN \\ & recvbuf & address of receive buffer & void* & TYPE(*), \hbox{}\kern10pt{}DIMENSION(..)  & OUT \\ & recvcount & number of elements in receive buffer & int & INTEGER  & IN \\ & recvtype & datatype of receive buffer elements & MPI_Datatype & TYPE\discretionary{}{\kern10pt}{}(MPI_Datatype)  & IN \\ & root & rank of sending process & int & INTEGER  & IN \\ & comm & communicator & MPI_Comm & TYPE\discretionary{}{\kern10pt}{}(MPI_Comm)  & IN \\
&)\\

\bottomrule
\end{tabular}
\endgroup

