% -*- latex -*-
%%%%%%%%%%%%%%%%%%%%%%%%%%%%%%%%%%%%%%%%%%%%%%%%%%%%%%%%%%%%%%%%
%%%%%%%%%%%%%%%%%%%%%%%%%%%%%%%%%%%%%%%%%%%%%%%%%%%%%%%%%%%%%%%%
%%%%
%%%% This text file is part of the lecture slides for
%%%% `Parallel Programming in MPI and OpenMP'
%%%% by Victor Eijkhout, copyright 2012-2024
%%%%
%%%% Affinity-slides.tex : remaining OpenMP topics
%%%%
%%%%%%%%%%%%%%%%%%%%%%%%%%%%%%%%%%%%%%%%%%%%%%%%%%%%%%%%%%%%%%%%
%%%%%%%%%%%%%%%%%%%%%%%%%%%%%%%%%%%%%%%%%%%%%%%%%%%%%%%%%%%%%%%%

\begin{numberedframe}{Random issues}
  \begin{itemize}
  \item If you have two threads and two cores, how do you place the threads?
  \item The OS can `migrate' threads. Is that good?
  \end{itemize}
\end{numberedframe}

\begin{numberedframe}{Frontera}
\includegraphics[scale=.12]{lstopo/frontera}
\end{numberedframe}

\begin{numberedframe}{Lonestar6}
\includegraphics[scale=.2]{lstopo/ls6}
\end{numberedframe}

\begin{numberedframe}{Affinity}
  \begin{itemize}
  \item
    \indexompshow{OMP_PROC_BIND}: either \lstinline{true/false} or \lstinline{close/spread}
  \item 
    \indexompshow{OMP_PLACES}: \lstinline{sockets/cores/threads} or explicit list.
  \end{itemize}
\end{numberedframe}

\begin{numberedframe}{Affinity utility}    
\includepdf{amask.pdf}
\end{numberedframe}

\begin{numberedframe}{NUMA}
  \includegraphics[scale=.13]{stampede-node}
\end{numberedframe}

\begin{numberedframe}{NUMA}
  \includegraphics[scale=.5]{ranger-numa}
\end{numberedframe}

\begin{numberedframe}{Virtual memory}
\begin{itemize}
\item
  Memory is organized in `pages', typically~4K, or 1M~for `large' pages.
\item User code uses `logical addresses' into these pages
\item Black magic translates logical addresses to physical.
\end{itemize}
\end{numberedframe}

\begin{numberedframe}{First-touch}
\begin{lstlisting}
double *x = (double*) malloc( lots );
// no memory has been reserved by malloc
for ( lots )
  // now the pages get created
  x[i] = something
\end{lstlisting}

  \begin{itemize}
  \item Pages get `instantiated' (placed in memory) when they are first touched.
  \item In multi-socket designs, pages are instantiated
    in the memory of the touching thread.
  \end{itemize}
\end{numberedframe}

\begin{numberedframe}{First touch}
\begin{lstlisting}
double *x = (double*) malloc( lots );
for ( lots )
  x[i] = 0.;
#pragma omp parallel for
for ( lots )
  x[i] = f(i);
\end{lstlisting}
  \begin{itemize}
  \item first loop instantiates pages on the socket
    of the primary thread
  \item second loop: half the threads need to pull
    data from the other socket.
  \item Solution?
  \end{itemize}
\end{numberedframe}

%% \begin{numberedframe}{False sharing}
%%   Used to be a problem, much less the
%% \end{numberedframe}

\endinput

\begin{numberedframe}{}
\begin{lstlisting}
\end{lstlisting}
  \begin{itemize}
  \item 
  \end{itemize}
\end{numberedframe}

