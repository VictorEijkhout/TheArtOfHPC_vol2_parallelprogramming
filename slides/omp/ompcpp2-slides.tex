% -*- latex -*-
%%%%%%%%%%%%%%%%%%%%%%%%%%%%%%%%%%%%%%%%%%%%%%%%%%%%%%%%%%%%%%%%
%%%%%%%%%%%%%%%%%%%%%%%%%%%%%%%%%%%%%%%%%%%%%%%%%%%%%%%%%%%%%%%%
%%%%
%%%% This text file is part of the source of 
%%%% `Parallel Programming in MPI and OpenMP'
%%%% by Victor Eijkhout, copyright 2022-2024
%%%%
%%%% ompcpp2-slides.tex : course in C++ aspects of OpenMP
%%%%
%%%%%%%%%%%%%%%%%%%%%%%%%%%%%%%%%%%%%%%%%%%%%%%%%%%%%%%%%%%%%%%%
%%%%%%%%%%%%%%%%%%%%%%%%%%%%%%%%%%%%%%%%%%%%%%%%%%%%%%%%%%%%%%%%

\begin{numberedframe}{Questions}
  \begin{enumerate}
  \item Do regular OpenMP loops look different in C++?
  \item Is there a relation between OpenMP parallel loops and iterators?
  \item OpenMP parallel loops vs parallel execution policies on algorithms.
  \end{enumerate}
\end{numberedframe}

\begin{numberedframe}{Range syntax}
  \input{cppnote-range-syntax.cut}
\end{numberedframe}

\begin{numberedframe}{General idea}
  OpenMP can parallelize any loop over a C++ construct
  that has a `random-access' iterator.
\end{numberedframe}

\begin{numberedframe}{C++ ranges header}
  \input{cppnote-c++20-ranges-header.cut}
\end{numberedframe}

\begin{numberedframe}{C++ ranges speedup}
  \input{cppnote-c++20-ranges-speedup.cut}
\end{numberedframe}

\begin{numberedframe}{Ranges and indices}
  \input{cppnote-ranges-and-indices.cut}  
\end{numberedframe}

\begin{numberedframe}{Custom iterators, 0}
  Recall that
\begin{multicols}{2}
Short hand:
\begin{lstlisting}
vector<float> v;
for ( auto e : v )
   ... e ...
\end{lstlisting}
\columnbreak for:
\begin{lstlisting}
for ( vector<float>::iter
      e=v.begin();
      e!=v.end(); e++ )
  ... *e ...
\end{lstlisting}
\end{multicols}
If we want 
\begin{lstlisting}
for ( auto e : my_object )
    ... e ...
\end{lstlisting}
we need a sub-class for the iterator with methods such as
\lstinline{begin}, \lstinline{end},
\lstinline|*| and \lstinline|++|.\\
Probably also \lstinline|+=| and \lstinline|-|
\end{numberedframe}

\begin{numberedframe}{Custom iterators, 1}
  OpenMP can parallelize any range-based loop with a random-access iterator.
  \begin{multicols}{2}
    Class:
    \cxxverbatimsnippet{classwithiter}
    \columnbreak
    Main:
    \cxxverbatimsnippet{ompcustompar}    
  \end{multicols}
  %%  \input{cppnote-custom-iterators.cut}
\end{numberedframe}

\begin{numberedframe}{Custom iterators, 2}
  Required iterator methods:
  \cxxverbatimsnippet{omprandaccess}  
  This is a little short of a full random-access iterator;
  the difference depends on the OpenMP implementation.
\end{numberedframe}

\begin{numberedframe}{Custom iterators, exercise}
  Write the missing iterator methods.\\
  Here's something to get you started.
  \cxxverbatimsnippet{classwithiteriter}  
\end{numberedframe}

\begin{numberedframe}{Custom iterators, solution}
  \cxxverbatimsnippet{classwithitersol1}
\end{numberedframe}

\begin{numberedframe}{Custom iterators, solution}
  \cxxverbatimsnippet{classwithitersol2}
\end{numberedframe}

\begin{numberedframe}{OpenMP vs standard parallelism}
  Application: prime number marking
  (load unbalanced)

  \cxxverbatimsnippet{markprimeomp}
  \cxxverbatimsnippet{markprimecpp}

  Standard parallelism uses \ac{TBB} as backend
\end{numberedframe}

\begin{numberedframe}{Timing}
  \lstinputlisting{examples/tbb/cxx/primepolicy-scaling-ls6.runout}
\end{numberedframe}

\begin{numberedframe}{Reductions vs standard parallelism}
  Application: prime number counting
  (load unbalanced)

  \cxxverbatimsnippet{reduceprimeomp}
  \cxxverbatimsnippet{reduceprimecpp}

\end{numberedframe}

\begin{numberedframe}{Timing}
  \lstinputlisting{examples/tbb/cxx/reducepolicy-scaling-ls6.runout}
\end{numberedframe}

