% -*- latex -*-
%%%%%%%%%%%%%%%%%%%%%%%%%%%%%%%%%%%%%%%%%%%%%%%%%%%%%%%%%%%%%%%%
%%%%%%%%%%%%%%%%%%%%%%%%%%%%%%%%%%%%%%%%%%%%%%%%%%%%%%%%%%%%%%%%
%%%%
%%%% This text file is part of the source of 
%%%% `Parallel Programming in MPI and OpenMP'
%%%% by Victor Eijkhout, copyright 2022-2024
%%%%
%%%% ompcpp1-slides.tex
%%%%
%%%%%%%%%%%%%%%%%%%%%%%%%%%%%%%%%%%%%%%%%%%%%%%%%%%%%%%%%%%%%%%%
%%%%%%%%%%%%%%%%%%%%%%%%%%%%%%%%%%%%%%%%%%%%%%%%%%%%%%%%%%%%%%%%

\begin{numberedframe}{Questions}
  \begin{enumerate}
  \item Are simple reductions the same as in C?
  \item Can you reduce \lstinline{std::vector} like an array?
  \item Precisely \emph{what} can you reduce?
  \item Any interesting examples?
  \item Compare reductions to native C++ mechanisms.
  \end{enumerate}
\end{numberedframe}

\begin{numberedframe}{Scalar reductions}
  Same as in C,\\
  you can now use range syntax for the loop.
  %
  \cxxverbatimsnippet{omprangereduct}
\end{numberedframe}

\begin{numberedframe}{Reductions on vectors}
  \input{cppnote-reductions-on-vectors.cut}
\end{numberedframe}

\begin{numberedframe}{Reduction on class objects}
  \input{cppnote-reduction-on-class-objects.cut}
\end{numberedframe}

\begin{numberedframe}{Reduction illustrated}
  \label{fig:omp-reduct}

  Reduction of four items on two threads.
  \n{i}~is the OpenMP initialization, and \n{u}~is the
  user initialization; each \n{p}~stands for a partial reduction value.
  \includegraphics[scale=.1]{omp-reduct.jpeg}
\end{numberedframe}

\begin{numberedframe}{User-defined reductions, syntax}
  \begin{lstlisting}
    #pragma omp declare reduction 
    ( identifier : typelist : combiner )
    [initializer(initializer-expression)]
  \end{lstlisting}
\end{numberedframe}

\begin{numberedframe}{Reduction over iterators}
  \input{cppnote-reduction-over-iterators.cut}
\end{numberedframe}

\begin{numberedframe}{Lambda expressions in declared reductions}
  \input{cppnote-lambda-expressions-in-declared-reductions.cut}
\end{numberedframe}

%% \begin{numberedframe}{Example: reduction over a map}
%%   \input{cppnote-example:-reduction-over-a-map.cut}
%% \end{numberedframe}

\begin{numberedframe}{Example category: histograms}
  Count which elements fall into what bin:
  \begin{lstlisting}
    for ( auto e : some_range )
      histogram[ value(e)]++;
  \end{lstlisting}
  Collisions are possible, but unlikely, so critical section is very inefficient
\end{numberedframe}

\begin{numberedframe}{Histogram: intended main program}
  Declare a reduction on a histogram object;\\
  each thread gets a local map:
  \cxxverbatimsnippet{ompbinreduce}
  Q: why does the \lstinline{inc} not have to be atomic?
\end{numberedframe}

\begin{numberedframe}{Histogram solution: reduction operator}
  Give the class a \lstinline{+=} operator to do the combining:
  \footnotesize
  \cxxverbatimsnippet{ompbincounter}
\end{numberedframe}

\begin{numberedframe}{Histogram in native C++}
  Use atomics because there is no reduction mechanism:
  \cxxverbatimsnippet{atomicbincounter}
\end{numberedframe}

\begin{numberedframe}{Histogram in native C++, comparison}
  OpenMP reduction on \lstinline{array<int,26>}:
  \footnotesize
  \lstinputlisting{code/omp/cxx/mapatomic-frontera.runout}
\end{numberedframe}

\begin{numberedframe}{Exercise: mapreduce}
  Make an OpenMP parallel version of:
\begin{lstlisting}
intcounter primecounter;
for ( auto n : numbers )
  if ( isprime(n) )
    primecounter.add(n);
\end{lstlisting}
  where \lstinline{primecounter} contains a \lstinline|map<int,int>|.

  Use skeleton: \n{mapreduce.cxx}  
\end{numberedframe}

\begin{numberedframe}{Example category: list filtering}
The sequential code is as follows:
\begin{lstlisting}
vector<int> data(100);
// fil the data
vector<int> filtered;
for ( auto e : data ) {
  if ( f(e) )
    filtered.push_back(e);
}
\end{lstlisting}
\end{numberedframe}

\begin{numberedframe}{List filtering, solution 1}
  Let each thread have a local array,
  and then to concatenate these:
\begin{lstlisting}
#pragma omp parallel
{
  vector<int> local;
# pragma omp for
  for ( auto e : data )
    if ( f(e) ) local.push_back(e);
  filtered += local;
}
\end{lstlisting}
where we have used an append operation on vectors:
\cxxverbatimsnippet{cppvectorplusis}
\end{numberedframe}

\begin{numberedframe}{List filtering, not quite solution 2}
We could use the plus-is operation to declare a reduction:
\cxxverbatimsnippet{cppvectorplusreduct}

Problem: OpenMP reductions can not be declared non-commutative,
so the contributions from the threads
may not appear in order.

\cxxsnippetwithoutput{cppvectordoreduct}{code/omp/cxx}{filterreduct}
\end{numberedframe}

\begin{numberedframe}{List filtering, task-based solution}
  Parallel region, without for:
  %
\cxxsnippetwithoutput{cppvectordomain}{code/omp/cxx}{filtertask}
\end{numberedframe}

\begin{numberedframe}{List filtering, task-based solution'}
  The task spins until it's its turn:
\cxxsnippetwithoutput{cppvectordotask}{code/omp/cxx}{filtertask}
\end{numberedframe}

\begin{numberedframe}{Templated reductions}
  \input{cppnote-templated-reductions.cut}
\end{numberedframe}

