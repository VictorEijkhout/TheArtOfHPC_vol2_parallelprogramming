% -*- latex -*-
%%%%%%%%%%%%%%%%%%%%%%%%%%%%%%%%%%%%%%%%%%%%%%%%%%%%%%%%%%%%%%%%
%%%%%%%%%%%%%%%%%%%%%%%%%%%%%%%%%%%%%%%%%%%%%%%%%%%%%%%%%%%%%%%%
%%%%
%%%% This text file is part of the lecture slides for
%%%% `Parallel Computing'
%%%% by Victor Eijkhout, copyright 2012-2025
%%%%
%%%% Cluster-slides.tex : logistics about clusters
%%%%
%%%%%%%%%%%%%%%%%%%%%%%%%%%%%%%%%%%%%%%%%%%%%%%%%%%%%%%%%%%%%%%%
%%%%%%%%%%%%%%%%%%%%%%%%%%%%%%%%%%%%%%%%%%%%%%%%%%%%%%%%%%%%%%%%

\begin{numberedframe}{Cluster setup}
  \small
  Typical cluster:
  \begin{itemize}
  \item Login nodes, where you ssh into; usually shared with 100 (or
    so) other people. You don't run your parallel program there!
  \item Compute nodes: where your job is run. They are often exclusive
    to you: no other users getting in the way of your program.
  \end{itemize}
  Hostfile: the description of where your job runs. Usually generated
  by a \indexterm{job scheduler}.
\end{numberedframe}

\begin{numberedframe}{Interactive run}
  \begin{itemize}
  \item Do not run your programs on a login node.
  \item Acquire compute nodes with \indextermtt{idev}
  \item Caveat: only small short jobs; nodes may not be available.
  \end{itemize}
  \includegraphics[scale=.15]{job_idev}
\end{numberedframe}

\begin{numberedframe}{Batch run}
  \begin{itemize}
  \item Submit batch job with \indextermtt{sbatch} or \indextermtt{qsub}
  \item Your job will be executed~\ldots Real Soon Now.
  \item See userguide for details about queues, sizes, runtimes,~\ldots
  \end{itemize}
  \includegraphics[scale=.15]{job_sbatch}
\end{numberedframe}

\begin{exerciseframe}
  \begin{itemize}
  \item Connect to your favorite cluster\\
    what is the hostname? how many users are logged in?
  \item Start an interactive session with \n{idev};\\
    what is the hostname? how many users are logged in?
  \item Run: \n{ibrun hostname}\\
    also \n{ibrun -n 3 hostname}
  \item Same, but \n{idev} on two nodes.
  \item Create a job script that will run on 10 nodes;\\
    again let it run the \n{hostname} command.
  \end{itemize}
\end{exerciseframe}
