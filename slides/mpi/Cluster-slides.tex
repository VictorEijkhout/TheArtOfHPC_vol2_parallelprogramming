% -*- latex -*-
%%%%%%%%%%%%%%%%%%%%%%%%%%%%%%%%%%%%%%%%%%%%%%%%%%%%%%%%%%%%%%%%
%%%%%%%%%%%%%%%%%%%%%%%%%%%%%%%%%%%%%%%%%%%%%%%%%%%%%%%%%%%%%%%%
%%%%
%%%% This text file is part of the lecture slides for
%%%% `Parallel Computing'
%%%% by Victor Eijkhout, copyright 2012-2026
%%%%
%%%% Cluster-slides.tex : logistics about clusters
%%%%
%%%%%%%%%%%%%%%%%%%%%%%%%%%%%%%%%%%%%%%%%%%%%%%%%%%%%%%%%%%%%%%%
%%%%%%%%%%%%%%%%%%%%%%%%%%%%%%%%%%%%%%%%%%%%%%%%%%%%%%%%%%%%%%%%

\begin{numberedframe}{Cluster setup}
  \small
  Typical cluster:
  \begin{itemize}
  \item Login nodes, where you ssh into; usually shared with 100 (or
    so) other people. You don't run your parallel program there!
  \item Compute nodes: where your job is run. They are often exclusive
    to you: no other users getting in the way of your program.
  \end{itemize}
\end{numberedframe}

\begin{numberedframe}{Login nodes}
  Shared between many users\\
  (how many right now?)\\
  You are allowed to do:
  \begin{itemize}
  \item Compilation
  \item Post-processing
  \item Run very short programs (but not MPI)
  \item Submit jobs for batch execution (\indextermtt{sbatch})
  \item Connect for interactive job (\indextermtt{idev})
  \end{itemize}
\end{numberedframe}

\begin{numberedframe}{Batch run}
  \begin{itemize}
  \item Submit batch job with \indextermtt{sbatch} \\
    (on other clusters: \indextermtt{qsub})
  \item Your job will be executed~\ldots Real Soon Now.
  \item See userguide for details about queues, sizes, runtimes,~\ldots
  \end{itemize}
  \includegraphics[scale=.15]{job_sbatch}
\end{numberedframe}

\begin{numberedframe}{Interactive run}
  \begin{itemize}
  \item Do not run your programs on a login node.
  \item Acquire compute nodes with \indextermtt{idev}
  \item Caveat: only small short jobs; nodes may not be available.\\
    During training there will be a `reservation':\\
    carefully look at messages from \lstinline{idev}
  \end{itemize}
  \includegraphics[scale=.15]{job_idev}
\end{numberedframe}

\begin{numberedframe}{{\tt idev} command}
\begin{lstlisting}
idev -t hh:mm:ss -N nodes -n cores -p queue
\end{lstlisting}
  \begin{itemize}
  \item \lstinline{-t}: time
  \item \lstinline{-N}: number of nodes
  \item \lstinline{-n}: total number of cores
  \item \lstinline{-p}: partition~/ queue
  \end{itemize}
\end{numberedframe}

\begin{numberedframe}{Batch job}
\begin{lstlisting}
sbatch batchfile.slurm
\end{lstlisting}
  \begin{itemize}
  \item \lstinline{sbatch}: submit
  \item \lstinline{squeue}: job status
  \end{itemize}
\end{numberedframe}

\begin{exerciseframe}
  \begin{itemize}
  \item Connect to your favorite cluster\\
    what is the hostname? how many users are logged in?
  \item Start an interactive session with \n{idev};\\
    what is the hostname? how many users are logged in?
  \item Run: \n{ibrun hostname}\\
    also \n{ibrun -n 3 hostname}
  \item Same, but \n{idev} on two nodes.
  \item Create a job script that will run on 10 nodes;\\
    again let it run the \n{hostname} command.\\
    Submit with \lstinline{sbatch}
  \end{itemize}
\end{exerciseframe}

\Level 1 {Course practicalities}

\begin{numberedframe}{Languages}
  You can program in:
  \begin{itemize}
  \item C: use \lstinline{mpicc} as compiler
  \item C++: use \lstinline{mpicxx} as compiler\\
    with MPL library: additionally \lstinline{module load mpl}
  \item Fortran: use \lstinline{mpif90} as compiler
  \item Python: \lstinline{module load python} or \lstinline{python3}
  \end{itemize}
\end{numberedframe}

\begin{numberedframe}{Lab setup}
  \begin{itemize}
  \item Clone the repository\\
    \url{https://github.com/VictorEijkhout/TheArtOfHPC_vol2_parallelprogramming}
  \item Directory: \n{exercises-mpi-c} or \n{cxx} or \n{f} or \n{f08}
    or \n{p} or \n{mpl}
  \item Open a terminal window on a TACC cluster.
  \item Type \n{idev -N 2 -n 10 -t 2:0:0 } which gives
    you an interactive session of 2~nodes, 10~cores, for the next
    2~hours.
  \item Type \n{make exercisename} to compile it
  \item Run with \n{ibrun} or \n{mpiexec} 
  \end{itemize}
\end{numberedframe}

\endinput

\begin{numberedframe}{}
  \begin{itemize}
  \item 
  \end{itemize}
\end{numberedframe}

\begin{numberedframe}{}
  \begin{itemize}
  \item 
  \end{itemize}
\end{numberedframe}

  Hostfile: the description of where your job runs. Usually generated
  by a \indexterm{job scheduler}.
