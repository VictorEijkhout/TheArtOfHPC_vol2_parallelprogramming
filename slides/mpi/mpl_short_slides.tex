% -*- latex -*-
%%%%%%%%%%%%%%%%%%%%%%%%%%%%%%%%%%%%%%%%%%%%%%%%%%%%%%%%%%%%%%%%
%%%%%%%%%%%%%%%%%%%%%%%%%%%%%%%%%%%%%%%%%%%%%%%%%%%%%%%%%%%%%%%%
%%%%
%%%% This text file is part of the source of 
%%%% `Parallel Computing'
%%%% by Victor Eijkhout, copyright 2012-2022
%%%%
%%%% mpl_short_slides.tex : short include file for MPL
%%%%
%%%%%%%%%%%%%%%%%%%%%%%%%%%%%%%%%%%%%%%%%%%%%%%%%%%%%%%%%%%%%%%%
%%%%%%%%%%%%%%%%%%%%%%%%%%%%%%%%%%%%%%%%%%%%%%%%%%%%%%%%%%%%%%%%

\begin{frame}{Justification}
  While the C API to MPI is usable from C++, it feels very unidiomatic
  for that language.
  \acf{MPL} is a modern C++11 interface to MPI.
  It is both idiomatic and elegant, simplifying many calling sequences.
  It is very low overhead.

  \url{https://github.com/rabauke/mpl}

\end{frame}

\Level 0 {Basics}

\begin{numberedframe}{Rank and size}
  \input{mplnote-rank-and-size.cut}
\end{numberedframe}
%% \referenceframe{Comm_size}
%% \referenceframe{Comm_rank}

\begin{numberedframe}{Scalar buffers}
  \input{mplnote-scalar-buffers.cut}
\end{numberedframe}
\begin{numberedframe}{Vector buffers}
  \input{mplnote-vector-buffers.cut}
\end{numberedframe}

\Level 1 {Collectives}

\begin{numberedframe}{Reduce on non-root processes}
  \input{mplnote-reduce-on-non-root.cut}
\end{numberedframe}
\begin{numberedframe}{User defined operators}
  \input{mplnote-user-defined-operators.cut}
\end{numberedframe}
\begin{numberedframe}{Lambda reduction operators}
  \input{mplnote-lambda-operator.cut}
\end{numberedframe}
\begin{numberedframe}{Nonblocking collectives}
  \input{mplnote-nonblocking-collectives.cut}
\end{numberedframe}

\Level 1 {Point-to-point communication}

\begin{numberedframe}{Blocking send and receive}
  \input{mplnote-blocking-send-and-receive.cut}
\end{numberedframe}
\begin{numberedframe}{Sending arrays}
  \input{mplnote-sending-arrays.cut}
\end{numberedframe}

\begin{numberedframe}{Requests from nonblocking calls}
  \input{mplnote-requests-from-nonblocking-calls.cut}
\end{numberedframe}
\begin{numberedframe}{Request pools}
  \input{mplnote-request-pools.cut}
\end{numberedframe}
\begin{numberedframe}{Request handling}
  \input{mplnote-request-handling.cut}
\end{numberedframe}

\Level 1 {Derived Datatypes}

\begin{numberedframe}{Vector type}
  \input{mplnote-vector-type.cut}
\end{numberedframe}

\Level 1 {Communicator manipulations}

\begin{numberedframe}{Communicator splitting}
  \input{mplnote-communicator-splitting.cut}
\end{numberedframe}


