% -*- latex -*-
%%%%%%%%%%%%%%%%%%%%%%%%%%%%%%%%%%%%%%%%%%%%%%%%%%%%%%%%%%%%%%%%
%%%%%%%%%%%%%%%%%%%%%%%%%%%%%%%%%%%%%%%%%%%%%%%%%%%%%%%%%%%%%%%%
%%%%
%%%% This text file is part of the source of 
%%%% `Parallel Computing'
%%%% by Victor Eijkhout, copyright 2012-7
%%%%
%%%% mpi_course.tex : master file for an MPI course
%%%%
%%%%%%%%%%%%%%%%%%%%%%%%%%%%%%%%%%%%%%%%%%%%%%%%%%%%%%%%%%%%%%%%
%%%%%%%%%%%%%%%%%%%%%%%%%%%%%%%%%%%%%%%%%%%%%%%%%%%%%%%%%%%%%%%%

\documentclass[dvipsnames,aspectratio=169,10pt]{beamer}
%\documentclass[11pt,headernav]{beamer}

\usepackage[T1]{fontenc}
\usepackage{xparse}
\usepackage{themes/beamerthemeTACC16-169}
\usepackage{fancybox, fancyvrb, calc}
\definecolor{blue}{HTML}{002868}
\definecolor{red}{HTML}{BF0A30}
\definecolor{brightnavyblue}{RGB}{0,102,204}

%% \beamertemplatenavigationsymbolsempty
%% \usetheme{Madrid}%{Montpellier}
%% \usecolortheme{seahorse}
\setcounter{tocdepth}{1}
\AtBeginSection[]
{
  \begin{frame}
    \frametitle{Table of Contents}
    \tableofcontents[currentsection]
  \end{frame}
}

\setbeamertemplate{footline}{\hskip1em Eijkhout: MPI intro\hfill
  \hbox to 0in {\hss \includegraphics[scale=.1]{tacclogonew}}%
  \hbox to 0in {\hss \arabic{page}\hskip 1in}}

\parskip=1pt plus 2pt 

\input coursemacs

\input tacc-condensed.inex

%%%%
%%%% Where is this course?
%%%%


\def\Location{TACC Supercomputing Institute, 2017}


%%%%
%%%% What level are we teaching?
%%%%

%\def\TitleExtra{, Summer Supercomputing Institute}

%%%%%%%%%%%%%%%%
%%%%%%%%%%%%%%%% Document
%%%%%%%%%%%%%%%%

\begin{document}
\parskip=10pt plus 5pt minus 3pt

\title{MPI Tutorial}
\author{Victor Eijkhout {\tt eijkhout@tacc.utexas.edu}}
\date{\Location}

\begin{frame}
  \titlepage
\end{frame}

\begin{frame}{Justification}
  The MPI library is the main tool
  for parallel programming on a large scale.
  This course introduces the main concepts
  through lecturing and exercises.
\end{frame}

\Level 0 {The SPMD model}
\input SPMD-slides

\Level 0 {Collectives}
\input Collective-slides
\sectionframe{Advanced collectives}
\input Highercollective-slides

\Level 0 {Point-to-point communication}
\input PTP-slides

\Level 0 {One-sided communication}
\input Onesided-slides

\begin{comment}
    \Level 0 {Complicated data}
    \input Data-slides

    \Level 0 {Sub-computations}
    \input Subcomm-slides

    \Level 0 {MPI File I/O}
    \input MPIO-slides


    \Level 0 {Process management}
    \input Spawn-slides
\end{comment}

\end{document}

\newenvironment
    {theindex}
    {\begin{itemize}\setlength\itemsep{0pt}\baselineskip=8pt}
    {\end{itemize}}
\let\indexspace\par
\def\subitem{\par\indent}
%
\begin{frame}{Index}
\small
\begin{multicols}{2}
\printindex  
\end{multicols}
\end{frame}

\end{document}

\begin{frame}{Bibliography}
  \bibliographystyle{plain}
  \bibliography{vle}
\end{frame}

