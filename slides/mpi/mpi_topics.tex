\documentclass[11pt,headernav]{beamer}

\beamertemplatenavigationsymbolsempty
\usetheme{Madrid}%{Montpellier}
\usecolortheme{seahorse}
\setcounter{tocdepth}{1}
\AtBeginSection[]
{
  \begin{frame}
    \frametitle{Table of Contents}
    \tableofcontents[currentsection]
  \end{frame}
}

\usepackage{pslatex}
\usepackage{amsmath,arydshln,multirow,multicol,verbatim,wrapfig}
\def\verbatimsnippet#1{\begingroup\small \verbatiminput{snippets/#1}\endgroup}

\setbeamertemplate{footline}{\hskip1em Eijkhout: MPI intro\hfill
  \arabic{page}\hskip1em \includegraphics[scale=.2]{tacc_logo}\hskip1em}

\newdimen\unitindent \unitindent=20pt
\usepackage[algo2e,noline,noend]{algorithm2e}
\newenvironment{displayalgorithm}
 {\par
  \begin{algorithm2e}[H]\leftskip=\unitindent \parskip=0pt\relax
  \DontPrintSemicolon
  \SetKwInOut{Input}{Input}\SetKwInOut{Output}{Output}
 }
 {\end{algorithm2e}\par}
\newenvironment{displayprocedure}[2]
 {\everymath{\strut}
  \begin{procedure}[H]\leftskip=\unitindent\caption{#1(#2)}}
 {\end{procedure}}
\def\sublocal{_{\mathrm\scriptstyle local}}

\usepackage{comment}
\specialcomment{exercise}{
  \def\CommentCutFile{exercise.cut}
  \def\PrepareCutFile{%
        \immediate\write\CommentStream{\noexpand\begin{frame}{Exercise}}}
  \def\FinalizeCutFile{%
        \immediate\write\CommentStream{\noexpand\end{frame}}}
  }{}

\usepackage{outliner}
\OutlineLevelStart 0{\frame{\part{#1}\Huge\bf #1}}
\OutlineLevelStart 1{\section{#1}}%{\frame{\section{#1}\Large\bf#1}}
\def\sectionframe#{\Level 1 }
\usepackage{framed}
\colorlet{shadecolor}{blue!15}
\OutlineLevelStart 2{\subsection{#1}
  \frame{\begin{shaded}\large #1\end{shaded}}}
%% \OutlineLevelCont 2{\end{frame}\begin{frame}{#1}}
%% \OutlineLevelEnd 2{\end{frame}}
%\OutlineTracetrue

\input scimacs
\input idxmacs

\newcounter{excounter}
\newenvironment{exerciseframe}
               {\begin{frame}\frametitle{Exercise \arabic{excounter}}
                   \refstepcounter{excounter}}
               {\end{frame}}

%\advance\textwidth by 1in
%\advance\oddsidemargin by -.5in

\usepackage{makeidx}
\makeindex

\begin{document}
\parskip=10pt plus 5pt minus 3pt

\title{Topics in parallel program design}
\author{Victor Eijkhout}
\date{2015}

\begin{frame}
  \titlepage
\end{frame}

\Level 0 {High performance linear algebra}

\begin{frame}{Justification}
  Bringing architecture-awareness to linear algebra,
  we discuss how high performance results from
  using the right formulation and implementation of algorithms.
\end{frame}

\input Collectives
\input DenseMVP
\input SparseMVP
\input ImplicitParallel
\input MulticoreBlock

\Level 0 {Applications}

\begin{frame}{Justification}
  We briefly discuss two applications that,
  while at first glance not linear-algebra like, surprisingly 
  can be covered by the foregoing concepts.
\end{frame}

\Level 1 {N-body problems: naive and equivalent formulations}
%\input BH-slides
\input Nbody-slides

\Level 1 {Graph analytics, interpretation as sparse matrix problems}

\input GraphAlgorithms-slides

\newenvironment{theindex}{\begin{itemize}}{\end{itemize}}
\let\indexspace\par
\def\subitem{\par\indent}

\begin{frame}{Index}
\small
\begin{multicols}{2}
\printindex  
\end{multicols}
\end{frame}

\end{document}
