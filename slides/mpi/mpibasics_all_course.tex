% -*- latex -*-
%%%%%%%%%%%%%%%%%%%%%%%%%%%%%%%%%%%%%%%%%%%%%%%%%%%%%%%%%%%%%%%%
%%%%%%%%%%%%%%%%%%%%%%%%%%%%%%%%%%%%%%%%%%%%%%%%%%%%%%%%%%%%%%%%
%%%%
%%%% This text file is part of the source of 
%%%% `Parallel Computing'
%%%% by Victor Eijkhout, copyright 2012-2025
%%%%
%%%% mpibasics_course.tex : basic part of the MPI course
%%%%
%%%%%%%%%%%%%%%%%%%%%%%%%%%%%%%%%%%%%%%%%%%%%%%%%%%%%%%%%%%%%%%%
%%%%%%%%%%%%%%%%%%%%%%%%%%%%%%%%%%%%%%%%%%%%%%%%%%%%%%%%%%%%%%%%

\documentclass[11pt]{beamer}

\usepackage{comment}
\input lang
\input courseformat
%% define MPI &c keywords
\input pcselistingmacs
\def\qrcode{qrvol2}

%%%%%%%%%%%%%%%%
%%%%%%%%%%%%%%%% Document
%%%%%%%%%%%%%%%%
\begin{document}

\author[Eijkhout]{Victor Eijkhout}
\date[2025]{2025 TACC APPI}
\title[MPI]{MPI Basics}
\maketitle

\begin{frame}[containsverbatim]{Materials}
    Textbooks and repositories:\\
    \url{https://theartofhpc.com}
\end{frame}

\begin{frame}{Justification}
  The MPI library is the main tool
  for parallel programming on a large scale.
  This course introduces the main concepts
  through lecturing and exercises.
\end{frame}

%% \begin{frame}{Table of Contents}
%%   \def\contentsline####1####2{\item{} ####2}
%%   \IfFileExists{\jobname.toc}
%%                {\begin{itemize}
%%                    \tableofcontents
%%                \end{itemize}
%%                }{}
%% \end{frame}

%% WHY? \renewcommand\standardversion{3}

\Level 0 {Supercomputer clusters}
\input Cluster-slides

\Level 0 {The SPMD model}
\label{sec:spmd}
\input SPMD-slides

\Level 0 {Collectives}
\label{sec:collectives}
\input Collective-slides

\Level 0 {Point-to-point communication}
\input PTP-slides

%% \begin{exerciseframe}[serialsend]
%%   \input ex:serialsend
%% \end{exerciseframe}

\Level 0 {Other topics}

\begin{numberedframe}{Where to go from here\ldots}
  \begin{itemize}
  \item Derived data types: send strided/irregular/inhomogeneous data
  \item Sub-communicators: work with subsets of \indexmpishow{MPI_COMM_WORLD}
  \item I/O: efficient file operations
  \item One-sided communication: `just' put/get the data somewhere
  \item Process management
  \item Non-blocking collectives
  \item Graph topology and neighborhood collectives
  \item Shared memory
  \end{itemize}
\end{numberedframe}

\input CMake-slides

%% \renewcommand\standardversion{}

\end{document}

