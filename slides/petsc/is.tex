% -*- latex
%%%%%%%%%%%%%%%%%%%%%%%%%%%%%%%%%%%%%%%%%%%%%%%%%%%%%%%%%%%%%%%%
%%%%
%%%% This TeX file is part of the tutorial
%%%% `Introduction to the PETSc library'
%%%% Victor Eijkhout, eijkhout@tacc.utexas.edu
%%%% copyright Victor Eijkhout 2012-2022
%%%%
%%%%%%%%%%%%%%%%%%%%%%%%%%%%%%%%%%%%%%%%%%%%%%%%%%%%%%%%%%%%%%%%

\sectionframe{IS and VecScatter: irregular grids}

\begin{numberedframe}{Irregular data movement}

  Example: collect distributed boundary onto a single processor
  (this happens in the matrix-vector product)

  \includegraphics[scale=.07]{vecscatter}

  Problem: figuring out communication is hard, actual communication is
  cheap
\end{numberedframe}

\begin{numberedframe}{VecScatter}
Preprocessing: determine mapping between input vector and output:
\begin{verbatim}
VecScatterCreate(Vec,IS,Vec,IS,VecScatter*) 
// also Destroy
\end{verbatim}
Application to specific vectors:
\begin{verbatim}
VecScatterBegin(VecScatter,
  Vec,Vec, InsertMode mode, ScatterMode direction)
VecScatterEnd  (VecScatter,
  Vec,Vec, InsertMode mode, ScatterMode direction)
\end{verbatim}
\end{numberedframe}

\begin{numberedframe}{IS: index set}
Index Set is a set of indices
\begin{verbatim}
ISCreateGeneral(comm,n,indices,PETSC_COPY_VALUES,&is);
    /* indices can now be freed */
ISCreateStride (comm,n,first,step,&is);
ISCreateBlock  (comm,bs,n,indices,&is);

ISDestroy(is);
\end{verbatim}
Use \lstinline{MPI_COMM_SELF} most of the time.

Various manipulations: \lstinline{ISSum}, \lstinline{ISDifference}, \lstinline{ISInvertPermutations}
et cetera.\\
Also \lstinline{ISGetIndices} / \lstinline{ISRestoreIndices} / \lstinline{ISGetSize}
\end{numberedframe}

\begin{numberedframe}{Example: split odd and even}
Input:
\begin{multicols}{2}
\tiny
\begin{verbatim}
Process [0]
0.
1.
2.
3.
4.
5.
\end{verbatim}
\begin{verbatim}
Process [1]
6.
7.
8.
9.
10.
11.
\end{verbatim}
\end{multicols}
Output:
\begin{multicols}{2}
\tiny
\begin{verbatim}
Process [0]
0.
2.
4.
6.
8.
10.
\end{verbatim}
\begin{verbatim}
Process [1]
1.
3.
5.
7.
9.
11.
\end{verbatim}
\end{multicols}
\end{numberedframe}

\begin{numberedframe}{index sets for this example}
\cverbatimsnippet{isoddeven}
\end{numberedframe}

\begin{numberedframe}{scatter for this example}
\cverbatimsnippet{vsoddeven}
\end{numberedframe}

\begin{exerciseframe}[oddeven]
Now alter the \lstinline{IS} objects so that the output becomes:

\begin{multicols}{2}
\tiny
\begin{verbatim}
Process [0]
10.
8.
6.
4.
2.
0.
\end{verbatim}
\begin{verbatim}
Process [1]
11.
9.
7.
5.
3.
1.
\end{verbatim}
\end{multicols}
\end{exerciseframe}

\begin{numberedframe}{Example: simulate allgather}
\small
\begin{verbatim}
/* create the distributed vector with one element per processor */
ierr = VecCreate(MPI_COMM_WORLD,&global); 
ierr = VecSetType(global,VECMPI); 
ierr = VecSetSizes(global,1,PETSC_DECIDE); 

/* create the local copy */
ierr = VecCreate(MPI_COMM_SELF,&local); 
ierr = VecSetType(local,VECSEQ); 
ierr = VecSetSizes(local,ntids,ntids); 
\end{verbatim}
\end{numberedframe}

\begin{numberedframe}
\small
\begin{verbatim}
IS indices;
ierr = ISCreateStride(MPI_COMM_SELF,ntids,0,1, &indices); 
/* create a scatter from the source indices to target */
ierr = VecScatterCreate
  (global,indices,local,indices,&scatter); 
/* index set is no longer needed */
ierr = ISDestroy(&indices); 
\end{verbatim}
\end{numberedframe}

\begin{numberedframe}{Example: even and odd indices}
\small
\cverbatimsnippet{isoddeven}
\end{numberedframe}

\begin{numberedframe}{scattering odd and even}
\small
\cverbatimsnippet{vsoddeven}
\end{numberedframe}

