% -*- latex -*-
%%%%%%%%%%%%%%%%%%%%%%%%%%%%%%%%%%%%%%%%%%%%%%%%%%%%%%%%%%%%%%%%
%%%%
%%%% This TeX file is part of the tutorial
%%%% `Introduction to the PETSc library'
%%%% Victor Eijkhout, eijkhout@tacc.utexas.edu
%%%% copyright Victor Eijkhout 2012-2022
%%%%
%%%%%%%%%%%%%%%%%%%%%%%%%%%%%%%%%%%%%%%%%%%%%%%%%%%%%%%%%%%%%%%%

\sectionframe{Grid manipulation}

\begin{numberedframe}{Regular grid: DMDA}
DMDAs are for storing vector field, not matrix.

Support for different stencil types:

\includegraphics[scale=.07]{starbox}
\end{numberedframe}

\begin{numberedframe}{Ghost regions around processors}
A DMDA defines a global vector, which contains the elements of the grid,
and a local vector for each processor which
has space for "ghost points".

\includegraphics[scale=.6]{ghost}
\end{numberedframe}

\begin{numberedframe}{DMDA construction}
\begin{lstlisting}
DMDACreate2d(comm, bndx,bndy, type, M, N, m, n, 
  dof, s, lm[], ln[], DMDA *da)
\end{lstlisting}
\footnotesize
\n{bndx,bndy} boundary behaviour: none/ghost/periodic

type: Specifies stencil\\
\lstinline{DMDA_STENCIL_BOX} or \lstinline{DMDA_STENCIL_STAR}

M/N: Number of grid points in x/y-direction\\
m/n: Number of processes in x/y-direction\\
dof: Degrees of freedom per node\\
s: The stencil width (for instance, 1 for 2D five-point stencil)\\
lm/n: array of local sizes (optional; 
Use \lstinline{PETSC_NULL} for the default)
\end{numberedframe}

\begin{numberedframe}{Grid info}
  Divide $100\times 100$ grid over 4 processes,\\
  stencil width$=1$:

  \def\codefontsize{\tiny}
  \csnippetwithoutput{dm2dcreate}{examples/petsc/c}{dmcreate}
\end{numberedframe}

\begin{numberedframe}{Associated vectors}
  \begin{itemize}
  \item Global vector: based on grid partitioning.
  \item Local vector: including halo regions
  \end{itemize}
  \cverbatimsnippet[examples/petsc/c/dmrhs.c]{dmghostvec}
\end{numberedframe}

\begin{numberedframe}{Grid info}
\begin{lstlisting}
typedef struct {
  PetscInt     dim,dof,sw;
  PetscInt     mx,my,mz;    /* grid points in x,y,z */
  PetscInt     xs,ys,zs;    /* starting point, excluding ghosts */
  PetscInt     xm,ym,zm;    /* grid points, excluding ghosts */
  PetscInt     gxs,gys,gzs; /* starting point, including ghosts */
  PetscInt     gxm,gym,gzm; /* grid points, including ghosts */
  DMBoundaryType   bx,by,bz;    /* type of ghost nodes */
  DMDAStencilType  st;
  DM               da;
} DMDALocalInfo;
\end{lstlisting}
\end{numberedframe}

\begin{numberedframe}{Range over local subdomain}
\begin{lstlisting}
for (int j=info.ys; j<info.ys+info.ym; j++) {
  for (int i=info.xs; i<info.xs+info.xm; i++) {
    // actions on point i,j
  }
}
\end{lstlisting}
\end{numberedframe}

\begin{numberedframe}{Arrays of vectors}
  \cverbatimsnippet[examples/petsc/c/dmrhs.c]{dmghostvec}
\end{numberedframe}

\begin{numberedframe}{Operating on arrays}
  \cverbatimsnippet[examples/petsc/c/dmrhs.c]{dmghostsweep}  
\end{numberedframe}

\begin{numberedframe}{Associated matrix}
Matrix that has knowledge of the grid:
\begin{lstlisting}
DMSetUp(DM grid);
DMCreateMatrix(DM grid,Mat *J)
\end{lstlisting}
Set matrix values based on stencil:
\begin{lstlisting}
MatSetValuesStencil(Mat mat,
  PetscInt m,const MatStencil idxm[],
  PetscInt n,const MatStencil idxn[],
  const PetscScalar v[],InsertMode addv)
\end{lstlisting}
(ordering of row/col variables too complicated for \lstinline{MatSetValues})
\end{numberedframe}

\begin{numberedframe}{Set values by stencil}
  \footnotesize
  \cverbatimsnippet{grid2d5pt}
\end{numberedframe}

\begin{numberedframe}{DMPlex}
  Support for unstructured grids and node/edge/cell relations.

  This is complicated and under-documented.
\end{numberedframe}

