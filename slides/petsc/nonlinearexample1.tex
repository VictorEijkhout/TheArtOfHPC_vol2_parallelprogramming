% -*- latex -*-
%%%%%%%%%%%%%%%%%%%%%%%%%%%%%%%%%%%%%%%%%%%%%%%%%%%%%%%%%%%%%%%%
%%%%
%%%% This TeX file is part of the tutorial
%%%% `Introduction to the PETSc library'
%%%% Victor Eijkhout, eijkhout@tacc.utexas.edu
%%%% copyright Victor Eijkhout 2012-2022
%%%%
%%%%%%%%%%%%%%%%%%%%%%%%%%%%%%%%%%%%%%%%%%%%%%%%%%%%%%%%%%%%%%%%

\sectionframe{Nonlinear example}

\frame{\frametitle{Nonlinear problems}

Basic equation
\[ f(u) = 0 \]
where $u$ can be big, for instance nonlinear PDE.

Typical solution method:
\[ u_{n+1} = u_n- J(u_n)\inv f(u_n) \]
Newton iteration.

Needed: function and Jacobian.
\end{numberedframe}

\begin{numberedframe}{Basic SNES usage}

User supplies function and Jacobian:
\begin{verbatim}
SNES           snes;

SNESCreate(PETSC_COMM_WORLD,&snes)

VecCreate(PETSC_COMM_WORLD,&r)
SNESSetFunction(snes,r,FormFunction,PETSC_NULL)

MatCreate(PETSC_COMM_WORLD,&J)
SNESSetJacobian(snes,J,J,FormJacobian,PETSC_NULL)

SNESSolve(snes,PETSC_NULL,x)
SNESGetIterationNumber(snes,&its)
\end{verbatim}
\end{numberedframe}

\begin{numberedframe}{Target function}
\begin{verbatim}
PetscErrorCode FormFunction
    (SNES snes,Vec x,Vec f,void *dummy)
{
VecGetArray(x,&xx); VecGetArray(f,&ff);

ff[0] = PetscSinScalar(3.0*xx[0]) + xx[0];
ff[1] = xx[1];

VecRestoreArray(x,&xx); VecRestoreArray(f,&ff);
return 0;
}
\end{verbatim}
\end{numberedframe}

\begin{numberedframe}{Jacobian}
\begin{verbatim}
PetscErrorCode FormJacobian
 (SNES snes,Vec x,Mat *jac,Mat *prec,MatStructure *flag,void *dummy)
{
PetscScalar    A[];
VecGetArray(x,&xx)
A[0] = ... ; /* et cetera */
MatSetValues(*jac,...,INSERT_VALUES)
MatSetValues(*prec,..,INSERT_VALUES)
*flag = SAME_NONZERO_PATTERN;
VecRestoreArray(x,&xx)
MatAssemblyBegin(*prec,MAT_FINAL_ASSEMBLY)
MatAssemblyEnd(*prec,MAT_FINAL_ASSEMBLY)
MatAssemblyBegin(*jac,MAT_FINAL_ASSEMBLY)
MatAssemblyEnd(*jac,MAT_FINAL_ASSEMBLY)
return 0;
}
\end{verbatim}
\end{numberedframe}

\begin{numberedframe}{Solve customization}
\begin{verbatim}
SNESSetType(snes,SNESTR); /* newton with trust region */
SNESGetKSP(snes,&ksp)
KSPGetPC(ksp,&pc)
PCSetType(pc,PCNONE)
KSPSetTolerances(ksp,1.e-4,PETSC_DEFAULT,PETSC_DEFAULT,20)
\end{verbatim}
\end{numberedframe}

\endinput

PetscMPIInt    size,rank;
PetscScalar    pfive = .5,*xx;
PetscTruth     flg;
AppCtx         user;         /* user-defined work context */
IS             isglobal,islocal;

/* ------------------------------------------------------------------- */
#undef __FUNCT__
#define __FUNCT__ "FormJacobian2"

